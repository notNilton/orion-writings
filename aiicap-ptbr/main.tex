%%
%% Arquivo principal para:
%% - trabalhos de conclusão de curso
%%
%% NOTA: ESSE MODELO DE TCC FOI BASEADO NO MODELO DE TESE E DISSERTAÇÃO DO PPGEEC,
%%       DA UFRN, COM AUTORIZAÇÃO DO SEU AUTOR, O PROFESSOR ADELARDO MEDEIROS
%%
%% Criado por Adelardo Medeiros, professor do Curso de Engenharia de Computação da UFRN, dezembro de 2005. 
%% Revisado pelos alunos de Metodologia da Pesquisa Científica de 2016.1, do curso de Engenharia de Computação da UFRN.
%% Adaptado por Diogo Henrique Duarte Bezerra, professor do Curso de Engenharia de Computação da UFMT, dezembro de 2020.

%%%%%%%%%%%%%%%%%%%%%%%%%%%%%%%%%%%%%%%%%%%%%%%%%%%%%%%%%%%%%%%%%%%%%%%%%%%%%%%
%% DEFINIÇÕES GLOBAIS
%%%%%%%%%%%%%%%%%%%%%%%%%%%%%%%%%%%%%%%%%%%%%%%%%%%%%%%%%%%%%%%%%%%%%%%%%%%%%%%

% Esta primeira linha define a classe do documento e suas opções básicas
% PARA A VERSÃO FINAL:
% - a4paper: tamanho do papel
% - 12pt: tamanho da fonte
% - openright: capítulos só iniciam em página direita (verso)
% - twoside: documento para impressão frente e verso
% PARA QUALIFICAÇÃO E VERSÃO INICIAL:
% - openany: capítulos iniciam em qualquer página
% - oneside: documento impresso apenas em um lado
\documentclass[a4paper,12pt,openany,oneside]{book}

%%%%%%%%%%%%%%%%%%%%%%%%%%%%%%%%%%%%%%%%%%%%%%%%%%%%%%%%%%%%%%%%%%%%%%%%%%%%%%%
%% PACOTES OBRIGATÓRIOS
%%%%%%%%%%%%%%%%%%%%%%%%%%%%%%%%%%%%%%%%%%%%%%%%%%%%%%%%%%%%%%%%%%%%%%%%%%%%%%%

% Pacote para codificação de caracteres (permite acentos diretamente)
\usepackage[utf8]{inputenc}
% Pacote para melhor renderização de fontes
\usepackage{ae}
% Define fonte padrão como sans-serif
\renewcommand{\familydefault}{\sfdefault}
% Melhora a qualidade das fontes no PDF
\usepackage{pslatex}
% Configurações de idioma e hifenização
\usepackage[portuges, brazil]{babel}
% Indenta o primeiro parágrafo de cada seção
\usepackage{indentfirst}
% Para inclusão de figuras
\usepackage{graphicx}
% Para criação de lista de símbolos/glossário
\usepackage[portuguese,noprefix]{nomencl}
% Controle de espaçamento entre linhas
\usepackage{setspace}
% Pacote matemático avançado
\usepackage{amsmath}
% Ambiente para comentários no código
\usepackage{verbatim}
% Tabelas com colunas de largura flexível
\usepackage{tabularx}
% Executar comandos após a página atual
\usepackage{afterpage}
% Formatação de URLs
\usepackage{url}
% Reduz espaçamento em listas
\usepackage{lib/noitemsep}
% Estilo de citações bibliográficas
\usepackage[abbr]{lib/harvard}
% Tabelas longas (multipágina)
\usepackage{longtable}
% Destaque de código fonte
\usepackage{minted}
% Suporte a cores
\usepackage{color}
% Células que abrangem múltiplas linhas em tabelas
\usepackage{multirow}
% Texto em múltiplas colunas
\usepackage{multicol}
% Listagens de código
\usepackage{listings}
% Pseudocódigos e algoritmos
\usepackage[portuguese, ruled, linesnumbered]{algorithm2e}
% Estilo personalizado para capítulos
\usepackage{lib/capitulos}
% Links e referências no PDF
\usepackage[breaklinks]{hyperref}
% Correção de links para figuras/tabelas
\usepackage[all]{hypcap}
% Cores em tabelas
\usepackage[table]{xcolor}
% Formatação de apêndices
\usepackage[titletoc]{appendix}
% Inclusão de PDFs externos
\usepackage{pdfpages}
% Tabelas com notas
\usepackage{threeparttable}

%%%%%%%%%%%%%%%%%%%%%%%%%%%%%%%%%%%%%%%%%%%%%%%%%%%%%%%%%%%%%%%%%%%%%%%%%%%%%%%
%% CONFIGURAÇÕES PERSONALIZADAS
%%%%%%%%%%%%%%%%%%%%%%%%%%%%%%%%%%%%%%%%%%%%%%%%%%%%%%%%%%%%%%%%%%%%%%%%%%%%%%%

% Carrega comandos personalizados definidos em um arquivo separado
\input{lib/comandos.tex}

%%%%%%%%%%%%%%%%%%%%%%%%%%%%%%%%%%%%%%%%%%%%%%%%%%%%%%%%%%%%%%%%%%%%%%%%%%%%%%%
%% DEFINIÇÃO DAS MARGENS
%%%%%%%%%%%%%%%%%%%%%%%%%%%%%%%%%%%%%%%%%%%%%%%%%%%%%%%%%%%%%%%%%%%%%%%%%%%%%%%

% Configuração de margens horizontais:
% - Margem esquerda para páginas ímpares: 3.5cm
% - Margem esquerda para páginas pares: 2.5cm
% - Largura do texto: 15cm
\setlength{\oddsidemargin}{2.5cm}
\setlength{\evensidemargin}{2.5cm}
\setlength{\textwidth}{16cm}
% Ajuste para compensar margem padrão do LaTeX (1 polegada)
\addtolength{\oddsidemargin}{-1in}
\addtolength{\evensidemargin}{-1in}

% Configuração de margens verticais:
% - Margem superior: 2.0cm
% - Altura do cabeçalho: 1.0cm
% - Espaço entre cabeçalho e texto: 1.0cm
\setlength{\topmargin}{2.0cm}
\setlength{\headheight}{1.0cm}
\setlength{\headsep}{1.0cm}
% Altura do texto principal
\setlength{\textheight}{22.7cm}
% Espaço entre texto e rodapé
\setlength{\footskip}{1.0cm}
% Ajuste para compensar margem padrão do LaTeX (1 polegada)
\addtolength{\topmargin}{-1in}

%%%%%%%%%%%%%%%%%%%%%%%%%%%%%%%%%%%%%%%%%%%%%%%%%%%%%%%%%%%%%%%%%%%%%%%%%%%%%%%
%% CONFIGURAÇÕES ADICIONAIS
%%%%%%%%%%%%%%%%%%%%%%%%%%%%%%%%%%%%%%%%%%%%%%%%%%%%%%%%%%%%%%%%%%%%%%%%%%%%%%%

% Define o estilo bibliográfico
\bibliographystyle{bibliografia/ppgee}

% Espaçamento entre linhas inicial
\singlespacing

% Habilita a criação de glossário/lista de símbolos
\makenomenclature

%%%%%%%%%%%%%%%%%%%%%%%%%%%%%%%%%%%%%%%%%%%%%%%%%%%%%%%%%%%%%%%%%%%%%%%%%%%%%%%
%% INÍCIO DO DOCUMENTO
%%%%%%%%%%%%%%%%%%%%%%%%%%%%%%%%%%%%%%%%%%%%%%%%%%%%%%%%%%%%%%%%%%%%%%%%%%%%%%%
\begin{document}

% Páginas iniciais sem numeração
\pagestyle{empty}

% Capa do trabalho
\begin{titlepage}

\begin{center}

\small

\begin{tabularx}{\linewidth}{@{}l@{}C@{}r@{}}
\parbox[c]{2cm}{\includegraphics[width=2cm]{pre-textuais/figuras/ufmt}} &
\begin{center}
\textsf{\textsc{Universidade Federal de Mato Grosso\\
Campus Universitário de Várzea Grande\\
Faculdade de Engenharia}}
\end{center} &
\parbox[c]{2cm}{\includegraphics[width=2.75cm]{pre-textuais/figuras/faeng}}
\end{tabularx}

\vfill

\LARGE

\textbf{RADARE: Desenvolvimento de uma aplicação \textit{web} de reconciliação de dados em processos industriais com recursos para análise de dados utilizando métodos e tecnologias atuais}

\vfill

\Large

\textbf{Nilton Aguiar dos Santos}

\vfill

\normalsize

Orientador: Prof. Dr. João Gustavo Coelho Pena

\vfill

\hfill
\parbox{0.5\linewidth}{\textbf{Trabalho de Conclusão de Curso}
apresentado ao Curso de Engenharia de Computação da FAENG/CUVG/UFMT,
na área de concentração de Engenharia de Computação,
como requisito parcial para a obtenção do título de Bacharel
em Engenharia de Computação.}

\vfill

\large

Cuiabá, MT, DIA de MES de ANO

\end{center}

\end{titlepage}


% Ficha catalográfica (apenas na versão final)
% \include{pre-textuais/catalograficos}

% Páginas preliminares (assinaturas, dedicatória, agradecimentos)
% %
% ********** Página de assinaturas
%

\begin{titlepage}

\begin{center}

\LARGE

\textbf{DRD - \textit{Dashboard of Data Reconciliation}: Desenvolvimento de uma aplicação \textit{web} de reconciliação de dados de processos industriais com recursos para análise de dados}

\vfill

\Large

\textbf{Nilton Aguiar dos Santos}

\end{center}

\vfill

% O \noindent é para eliminar a tabulação inicial que normalmente é
% colocada na primeira frase dos parágrafos
\noindent
% Descomente a opção que se aplica ao seu caso
% Note que propostas de tema de qualificação nunca têm preâmbulo.
Trabalho de Conclusão de Curso
%Tese de Doutorado
aprovado em XX de XXXXXXX de 202X pela banca examinadora composta
pelos seguintes membros:

% Os nomes dos membros da banca examinadora devem ser listados
% na seguinte ordem: orientador, co-orientador (caso haja),
% examinadores externos, examinadores internos. Dentro de uma mesma
% categoria, por ordem alfabética

\begin{center}

\vspace{1.5cm}\rule{0.95\linewidth}{1pt}
\parbox{0.9\linewidth}{%
Prof. Dr. Sicrano Matosinho de Melo (orientador) \dotfill\ FAENG/CUVG/UFMT}

\vspace{1.5cm}\rule{0.95\linewidth}{1pt}
\parbox{0.9\linewidth}{%
Prof. Dr. Beltrano Catandura do Amaral (co-orientador) \dotfill\ FAENG/CUVG/UFMT}

\vspace{1.5cm}\rule{0.95\linewidth}{1pt}
\parbox{0.9\linewidth}{%
Prof. Dr. Clint Stallone da Silva \dotfill\ IC/UFMT}

%\vspace{1.5cm}\rule{0.95\linewidth}{1pt}
%\parbox{0.9\linewidth}{%
%Profª Drª Florisbela do Amaral \dotfill\ DCA/UFRN}

\end{center}

\end{titlepage}

%
% ********** Dedicatória
%

% A dedicatória não é obrigatória. Se você tem alguém ou algo que teve
% uma importância fundamental ao longo do seu curso, pode dedicar a ele(a)
% este trabalho. Geralmente não se faz dedicatória a várias pessoas: para
% isso existe a seção de agradecimentos.
% Se não quiser dedicatória, basta excluir o texto entre
% \begin{titlepage} e \end{titlepage}

\begin{titlepage}

\vspace*{\fill}

\hfill
\begin{minipage}{0.5\linewidth}
\begin{flushright}
\large\it
``Não é o conhecimento, mas o ato de aprender.
Não a posse, mas o ato de chegar lá
Que garante o maior prazer''
(Carl Friedrich Gauss)
\end{flushright}
\end{minipage}

\vspace*{\fill}

\end{titlepage}

\chapter*{Agradecimentos}
\thispagestyle{empty}

\begin{trivlist}  \itemsep 2ex

\item Aos meus pais, José Roberto dos Santos e Teresa de Jesus de Souza Aguiar, por cada sacrifício, por cada gesto de amor, por cada exemplo de dedicação. 

\item Aos meus irmãos por acreditaram em meu potencial e caminharam ao meu lado nesta jornada acadêmica. 

\item À minha namorada por ter me motivado a cada segundo do processo.

\item Ao meu orientador, João Gustavo Coelho Pena, pela inspiração e dedicação.

\item Ao corpo docente da UFMT por todo esforço durante todo o meu curso para me dar um ótimo ensino.

\end{trivlist}


% Define espaçamento 1.5 para o texto principal
\doublespacing

% % Inclui resumo e abstract | O resumo já está incluído na página inicial
% %
% ********** Resumo
%

% Usa-se \chapter*, e não \chapter, porque este "capítulo" não deve
% ser numerado.
% Na maioria das vezes, ao invés dos comandos LaTeX \chapter e \chapter*,
% deve-se usar as nossas versões definidas no arquivo comandos.tex,
% \mychapter e \mychapterast. Isto porque os comandos LaTeX têm um erro
% que faz com que eles sempre coloquem o número da página no rodapé na
% primeira página do capítulo, mesmo que o estilo que estejamos usando
% para numeração seja outro.
\mychapterast{Resumo}

Neste trabalho de conclusão de curso, foi desenvolvido o \textbf{RADARE (Reconciliation and Data Analysis in a Responsive Environment)}, um \textit{software online} voltado para reconciliação e qualidade de dados, utilizando técnicas de minimização de funções multivariáveis pelo método dos multiplicadores de Lagrange. A solução prioriza uma abordagem baseada na \textit{web}, oferecendo ao usuário a capacidade de realizar a análise e reconciliação de dados de forma remota e eficiente, com foco nos conceitos computacionais modernos, para proporcionar uma experiência facilitada. O \textit{software} aplica cálculos matemáticos e estatísticos ao longo de todo o processo, especificamente voltados para problemas de reconciliação de dados.

Por meio do RADARE, é possível modelar todo um processo industrial, alimentá-lo com dados oriundos da planta em questão e, a partir disso, reconciliar e verificar a qualidade dos dados em tempo real. O \textit{software} se destaca como uma solução inovadora, pois não há concorrente direto que ofereça as mesmas funcionalidades em um ambiente mais acessível e ágil no estado de Mato Grosso. Ao longo deste trabalho, são detalhados o processo filosófico de desenvolvimento, os cálculos matemáticos, os conceitos estatísticos e computacionais, e a lógica do \textit{software} aplicada na solução. Exemplos do código funcional e considerações finais sobre o trabalho realizado são apresentados nos capítulos finais.


\vspace{1.5ex}

{\bf Palavras-chave}: \textit{Dashboard}, Desenvolvimento de \textit{Software}, Desenvolvimento \textit{Web}, Indústria 4.0, Multiplicadores de Lagrange, Qualidade de Dados, Reconciliação de Dados.


%
% ********** Abstract
%
\mychapterast{Abstract}

In this graduation thesis, the \textbf{RADARE (Reconciliation and Data Analysis in a Responsive Environment)} was developed, an \textit{online software} focused on data reconciliation and quality, using multivariable function minimization techniques through the Lagrange multipliers method. The solution prioritizes a \textit{web}-based approach, offering users the ability to perform data analysis and reconciliation remotely and efficiently, focusing on modern computational concepts to provide a facilitated experience. The \textit{software} applies mathematical and statistical calculations throughout the entire process, specifically aimed at data reconciliation problems.

Through RADARE, it is possible to model an entire industrial process, feed it with data from the plant in question, and then reconcile and verify the quality of the data in real-time. The software stands out as an innovative solution, as there is no direct competitor offering the same functionalities in a more accessible and agile environment in the state of Mato Grosso. Throughout this work, the philosophical development process, the mathematical calculations, the statistical and computational concepts, and the logic of the software applied as a solution are detailed. Examples of functional code and final considerations about the work performed are presented in the final chapters.

\vspace{1.5ex}

{\bf Keywords}: Dashboard, Data Quality, Data Reconciliation, Industry 4.0, Lagrange Multipliers, Software Development, Web Development.


%%%%%%%%%%%%%%%%%%%%%%%%%%%%%%%%%%%%%%%%%%%%%%%%%%%%%%%%%%%%%%%%%%%%%%%%%%%%%%%
%% ELEMENTOS PRÉ-TEXTUAIS
%%%%%%%%%%%%%%%%%%%%%%%%%%%%%%%%%%%%%%%%%%%%%%%%%%%%%%%%%%%%%%%%%%%%%%%%%%%%%%%

% Inicia numeração romana para páginas pré-textuais
\frontmatter

% % Sumário
% \phantomsection
% \addcontentsline{toc}{chapter}{Sumário}
% \tableofcontents

% % Lista de figuras
% \cleardoublepage
% \phantomsection
% \addcontentsline{toc}{chapter}{Lista de Figuras}
% \listoffigures

% % Lista de tabelas
% \cleardoublepage
% \phantomsection
% \addcontentsline{toc}{chapter}{Lista de Tabelas}
% \listoftables

%%%%%%%%%%%%%%%%%%%%%%%%%%%%%%%%%%%%%%%%%%%%%%%%%%%%%%%%%%%%%%%%%%%%%%%%%%%%%%%
%% TEXTO PRINCIPAL
%%%%%%%%%%%%%%%%%%%%%%%%%%%%%%%%%%%%%%%%%%%%%%%%%%%%%%%%%%%%%%%%%%%%%%%%%%%%%%%

% Inicia numeração arábica e estilo com cabeçalhos
\mainmatter
\pagestyle{headings}

% Capítulos do trabalho:
\mychapter{Introdução}
\label{Cap:Introducao}

O setor industrial brasileiro, responsável por cerca de 20\% do PIB e 70\% das exportações \cite{cni2024}, enfrenta desafios constantes, sendo crucial para a economia nacional. A necessidade de inovação e investimentos em tecnologia apresenta ser um dos pilares da evolução do setor industrial brasileiro \cite{produtividadeindustria}. Nesse cenário, o \textit{software} de reconciliação de dados \textit{online}, denominado RADARE, foi desenvolvido como uma proposta inovadora para aprimorar a qualidade e confiabilidade dos dados nos processos industriais, enquanto oferta uma interface gráfica de fácil uso e com funcionalidades de uma API (\textit{Application Programming Interface}) que permite adaptações para atender a diferentes aplicações industriais.

A precisão dos dados de processo nas indústrias químicas/petroquímicas desempenha um papel crucial no desempenho operacional e nos ganhos financeiros provenientes da utilização de diversos \textit{softwares} para monitoramento de processos, otimização \textit{online} e controle \cite{datarecshakar}. Entretanto, as medições realizadas na planta frequentemente apresentam imprecisões causadas por erros aleatórios ou sistemáticos que comprometem o modelo de processo utilizado para otimização e controle. E, para resolver problemas dessa natureza, foi desenvolvida a área de conhecimento da reconciliação de dados que faz uso de equações de processos, relações de equilíbrio e leis de conservação de massa e energia \cite{reformulationdatarecon}. De todos os possíveis métodos a serem aplicados no \textit{software} RADARE, o escolhido foi o de minimização de funções multivariáveis utilizando o método dos multiplicadores de Lagrange.

Além disso, o desenvolvimento \textit{web} tem experimentado um notável crescimento na última década \cite{webusage}. As vantagens distintas do ambiente \textit{web}, como a implementação de sistemas de API \cite{apirest}, possibilitam interoperabilidade eficiente entre diferentes sistemas, acesso remoto, manuseio simplificado \cite{apiimportance} e maior acessibilidade, inclusive para indivíduos com deficiências visuais. Esses atributos tornam o ambiente \textit{web} uma escolha estratégica para o desenvolvimento do RADARE, destacando-o em relação a alternativas menos dinâmicas e explorando áreas ainda não amplamente investidas no mercado.

Portanto, a criação de uma ferramenta que unifique esses aspectos tecnológicos torna-se imperativa. Ao integrar a teoria de reconciliação de dados com as técnicas modernas de desenvolvimento de \textit{software}, o RADARE não só representa um avanço significativo no enfrentamento dos desafios industriais, mas também se beneficia das vantagens proporcionadas pelo ambiente \textit{web}, oferecendo uma plataforma flexível e adaptável. Este cenário fomenta a excelência operacional e impulsiona o desenvolvimento tecnológico no Brasil \cite{industry4status}.

\section{Objetivos}
\subsection{\textit{Objetivo geral}}

Desenvolver e implementar um \textit{software web} para a reconciliação de dados oriundos de processos industriais, utilizando o método dos multiplicadores de Lagrange, com o objetivo de aumentar a qualidade, a confiabilidade das informações e otimizar a tomada de decisões nos processos industriais.

\subsection{\textit{Objetivos específicos}}

Para atingir o objetivo geral, os seguintes objetivos específicos foram definidos:

\begin{itemize}
    \item Pesquisar o estado da arte sobre as principais técnicas e ferramentas de validação, limpeza e reconciliação de dados, identificando práticas e tendências recentes.
    \item Analisar metodologias e algoritmos de reconciliação de dados, destacando suas aplicações, vantagens e limitações, para uma escolha fundamentada dos métodos a serem implementados.
    \item Selecionar e adaptar as tecnologias mais adequadas para a implementação da ferramenta \textit{online}, considerando linguagens de programação, \textit{frameworks}, escalabilidade, segurança e eficiência.
    \item Projetar a arquitetura da ferramenta, definindo os componentes principais, suas interações e lógica de funcionamento para garantir a eficácia e facilitar futuras atualizações.
    \item Desenvolver a ferramenta \textit{online}, implementando os algoritmos e funcionalidades identificados, assegurando a usabilidade, a integridade dos dados e a eficiência do sistema.
    \item Realizar testes abrangentes de integração, segurança e simulações práticas, a fim de validar a eficácia e confiabilidade da ferramenta em diferentes cenários e volumes de dados, garantindo sua aplicabilidade em ambientes reais de produção.
    \item Elaborar uma documentação completa, cobrindo o processo de desenvolvimento e instruções de uso, para facilitar a manutenção e futuras implementações da ferramenta.
\end{itemize}


\section{Estado da Arte}

Em meio ao processo de elaboração do trabalho de conclusão de curso, um aspecto fundamental que merece destaque é a investigação das tendências tecnológicas emergentes nas áreas relevantes para o estudo. Essa pesquisa não apenas fornece uma visão abrangente do cenário atual, mas também ajuda a identificar oportunidades para inovação e melhoria.

As tendências tecnológicas são indicadores poderosos do progresso em qualquer campo de estudo e podem ser representadas tanto na área comercial direta, como na indústria química, nas áreas adjacentes, como em conceitos de contabilidade, além de pesquisas cientificas. Elas refletem os avanços mais recentes e as direções futuras que a tecnologia pode tomar. Portanto, é essencial estar ciente dessas tendências ao realizar qualquer pesquisa acadêmica ou científica.

\subsection{Aplicação na indústria química}

No contexto da indústria química, há várias empresas com soluções similares à proposta neste trabalho, e a investigação das tendências tecnológicas emergentes é um aspecto crucial durante o processo de elaboração dessa ferramenta (RADARE). 

\subsubsection{Software RECON da empresa ChemPlant Technology}

A ChemPlant Technology, s.r.o, fundada em 1991 é uma empresa situada na Republica Checa, especializada em fornecer soluções tecnológicas avançadas para indústrias de processos (principalmente, processos químico, petróleo e gás, geração e distribuição de energia) \cite{reconset}. Uma de suas inovações mais notáveis é a ferramenta de reconciliação de dados chamada RECON. Esta ferramenta é fundamentada nos sólidos princípios físicos de balanço de massa e energia, um \textit{software} interativo abrangente que oferece uma plataforma robusta para modelagem de complexas plantas industriais químicas e energéticas. Ele realiza uma variedade de cálculos, incluindo balanceamento de massa, energia e momento, bem como cálculos termodinâmicos. O principal objetivo da solução é validar dados que já foram obtidos de processos operacionais. No entanto, a ferramenta, exposta na Figura \ref{fig:RECONSET} também pode ser utilizada para simular o comportamento da planta sob diferentes condições.

\begin{figure}[htbp!] 
    \centering
    \includegraphics[width=0.6\textwidth]{figuras/RECONSET.jpg}
    \caption{Tela Principal da ferramenta RECONSET. Fonte: Chemplant Techonology (2024).}
    \label{fig:RECONSET}
\end{figure}

O \textit{software} é orientado para PC e possui uma interface de usuário gráfica interativa, tornando-o fácil de usar. Os usuários definem problemas (ou tarefas) interativamente por meio desta interface. O RECON é capaz de equilibrar materiais e energia de componentes únicos ou múltiplos de sistemas complexos, seja em estado estacionário ou instável (dinâmico), com ou sem reações químicas (balanceamento de reatores). Além disso, ele pode realizar balanceamento de momento com base em cálculos hidráulicos de vazão em sistemas de dutos. O RECON reconcilia vazões medidas, concentrações, temperaturas e outras variáveis de processo, e calcula variáveis não medidas. Para definir um problema (ou tarefa), os usuários geralmente criam um fluxograma de processo e definem variáveis de processo, como taxas de fluxo, temperaturas, pressões, etc. O fluxograma inclui nós, fluxos de massa e energia, e trocadores de calor. Se necessário, os usuários também podem complementar (ou até mesmo substituir) o modelo de balanceamento com suas próprias equações.

\subsubsection{BILCO}

CASPEO é uma empresa francesa fundada em 2004, especializada em engenharia de processos e soluções tecnológicas. Originada do 
Departamento de Pesquisas Geológicas e Minerais da França. Ela foi criada para oferecer à indústria de mineração métodos e ferramentas computacionais resultantes de anos de pesquisa e tornou-se uma referência na indústria de processamento mineral, atendendo a vários mercados, como mineração e metalurgia, processamento de biomassa e alimentos, tratamento de resíduos sólidos e outras indústrias de processamento \cite{bilco}.

Uma das suas inovações é o \textit{software} de reconciliação de dados BILCO, projetado para derivar um balanço de material coerente e total a partir de todos os dados disponíveis (medições, análises, estimativas) para todas as correntes de processo. Ele é uma ferramenta poderosa que permite aos usuários reconciliar dados de qualquer planta de processamento, a Figura \ref{fig:BILCO}, demonstra a tela inicial do \textit{software}.

\begin{figure}[htbp!] 
    \centering
    \includegraphics[width=0.6\textwidth]{figuras/BILCOCASPEOP.png}
    \caption{Tela Principal da ferramenta BILCO (Fonte: BRGM - Pesquisa Geológica Francesa, 2024).}
    \label{fig:BILCO}
\end{figure}

Ele tem a capacidade de se incorporar à duas outras ferramentas facilmente, no caso, à um simulador de processos, denominado USIM PAC e à um \textit{software} de contabilidade metalúrgica, INVETEO; e, dessa forma, fornece cálculos de balanço precisos e gera um conjunto de estimadores coerentes que estão em conformidades com as restrições da lei de conservação de energia, além de calcular seus erros relativos. Ele também é capaz de determinar, em quantidade e qualidade, a composição de cada corrente de processo. É um dos poucos \textit{softwares} de balanço de massa capaz de calcular toda a composição da corrente (taxas de fluxo, classes de tamanho, tipos de partículas, massa molar, etc.) em um único cálculo.

Essa solução também oferece uma única interface para gerenciar todo o processo. Consiste numa interface gráfica fácil de usar, tornando-a acessível tanto para novos usuários quanto para os mais experientes. Uma das características mais úteis do BILCO é a possibilidade de exportar os resultados para o Excel, permitindo uma análise mais profunda dos dados. Ele fornece resultados detalhados, incluindo uma planilha de comparação e uma planilha global, para uma visão completa do balanço de material.

\subsubsection{Software PIMSOFT da empresa Sigmafine}

A Sigmafine, fundada na Itália, é uma empresa que desenvolve e integra soluções voltadas para a qualidade de dados e reconciliação em indústrias de processos. Seu principal produto, o \textit{PIMSOFT}, foi originalmente lançado em 2002 pela Sigmafine.

O \textit{PIMSOFT} é uma solução voltada para a reconciliação de dados e controle de qualidade, amplamente utilizada em setores como petróleo, gás e manufatura. A ferramenta utiliza princípios de conservação, estatísticas e padrões de engenharia para validar, corrigir e estruturar dados de processos industriais, oferecendo uma base confiável para análises e tomadas de decisão. A Sigmafine continua a expandir e aprimorar o PIMSOFT.

\begin{figure}[htbp!] 
    \centering
    \includegraphics[width=0.6\textwidth]{figuras/sigmafine-pimsoft.png}
    \caption{Tela Principal da ferramenta PIMSoft (Fonte: Sigmafine, 2023).}
    \label{fig:Sigmafine}
\end{figure}

\subsubsection{Software AspenTech da empresa Aspentech}

A AspenTech foi fundada a partir do Projeto ASPEN, uma iniciativa conjunta entre o MIT e o Departamento de Energia dos EUA nos anos 1980, com o objetivo de desenvolver tecnologia avançada de modelagem e simulação para a indústria de processos químicos. A AspenTech se consolidou como uma empresa líder em soluções de otimização industrial, oferecendo softwares especializados para diversos setores, incluindo geração de energia, mineração, química e farmacêutica, entre outros \cite{aspen}.

O software \textit{AspenTech} foi desenvolvido para auxiliar na reconciliação de dados e otimização de processos industriais, com funcionalidades que permitem o monitoramento em tempo real e a análise de desempenho dos ativos. Essa ferramenta facilita a redução de perdas de material, o monitoramento de emissões e a gestão de balanços de massa e energia, contribuindo para um uso mais eficiente dos recursos.

\begin{figure}[htbp!] 
    \centering
    \includegraphics[width=0.6\textwidth]{figuras/aspen-aspentech.png}
    \caption{Tela do software AspenTech (Fonte: AspenTech, 2024).}
    \label{fig:AspenTech}
\end{figure}


\subsection{Aplicação na contabilidade}

Paralelamente à disciplina de Reconciliação de Dados na área industrial química, há uma outra área na qual investe bastante nesse campo, a de soluções contábeis. E a investigação das tendências tecnológicas emergentes é um aspecto crucial durante o processo de desenvolvimento de uma solução mais dinâmica e que consegue resolver problemas já solucionados por outras áreas da ciência.

\subsubsection{Software AutoRec da empresa FloQast}

A FloQast é uma empresa fundada em 2013, especializada em contabilidade operacional na nuvem, com foco na automação e gestão contábil. Uma de suas soluções é o \textit{AutoRec}, uma ferramenta de reconciliação de dados voltada para contabilidade, parte da plataforma FloQast Reconciliation Management. Essa solução avançada de automação de fluxo de trabalho permite o gerenciamento de reconciliações de contas de ponta a ponta, otimizando o fechamento financeiro e reduzindo o risco de declarações incorretas \cite{floqast}.

O \textit{software} permite que controladores e suas equipes automatizem e gerenciem o processo de reconciliação em uma plataforma centralizada, amplamente confiável por contadores e auditores em todo o mundo.

\begin{figure}[htbp!] 
    \centering
    \includegraphics[width=0.6\textwidth]{figuras/floqast-autorec.png}
    \caption{Tela principal da ferramenta AutoRec (Fonte: FloQast, 2024).}
    \label{fig:AutoRec}
\end{figure}

\subsubsection{Software AURA da empresa Aspentech}

O \textit{AURA} (Aspen Unified Reconciliation Accounting), desenvolvido pela Aspentech, é uma solução destinada à reconciliação de dados em processos industriais, com foco na minimização de perdas materiais e na otimização de margens por meio de um balanceamento eficiente de massa e volume. A ferramenta fornece suporte à tomada de decisões baseadas em dados validados e reconciliados, apresentando uma arquitetura escalável que permite rápida implantação em ambientes de nuvem.

O sistema utiliza o \textit{Smart Solver}, um algoritmo proprietário que ajusta automaticamente erros de densidade detectados, aprimorando a precisão da reconciliação de dados. Além disso, o \textit{AURA} contribui para o monitoramento e a gestão de emissões de gases de efeito estufa, facilitando a geração de relatórios sobre a intensidade de carbono dos produtos.

Devido à indisponibilidade pública, imagens do \textit{AURA} não foram incluídas neste trabalho.

% ------
\subsection{Pesquisas sendo desenvolvidas}

O objetivo do RADARE é incorporar avanços tecnológicos de diversas áreas, com foco não apenas nas técnicas aplicadas diretamente à reconciliação de dados, mas também em áreas adjacentes que possam contribuir para o aprimoramento da ferramenta. Dessa forma, é fundamental estar alinhado com as pesquisas mais recentes em reconciliação de dados computacional, bem como explorar como elas podem ser aplicadas no contexto industrial brasileiro.

Pesquisas em outras áreas de aplicação, como o processamento mineral e a distribuição de gás, também oferecem contribuições valiosas para o desenvolvimento do RADARE, especialmente na aplicação de técnicas de reconciliação de dados em diferentes cenários industriais.

\subsubsection{Controle e Supervisão de Processamento Mineral}

A pesquisa apresentada por Daniel Hoduin \cite{danielhoduin} foca nas restrições de conservação de massa e energia, que servem como base para o desenvolvimento de estratégias de medição, atualização do valor medido por meio de técnicas de filtragem de erros de medição e estimativa de variáveis de processo não medidas. Nas unidades de processamento mineral, as principais variáveis são geralmente as vazões e concentrações, e a reconciliação desses dados com as leis de conservação de massa é essencial para manter a precisão dos processos.

Os métodos de reconciliação propostos utilizam técnicas clássicas, como os mínimos quadrados usuais, que minimizam os erros entre os valores observados e os valores estimados. Além disso, a filtragem de Kalman, um algoritmo de estimação recursiva, é empregada para estimar variáveis de estado de processos dinâmicos e com ruído. São sugeridas ferramentas para três tipos de regimes operacionais: regime estacionário, quase-estacionário e dinâmico.

As ferramentas propostas para diferentes regimes operacionais de processamento mineral podem ser adaptadas para o RADARE, permitindo que ele lide com cenários variáveis de fluxo de dados e oferecendo maior flexibilidade no tratamento de dados de processos industriais com características dinâmicas.

\subsubsection{Balanços de Materiais em Rede de Distribuição de Gás}

O desenvolvimento realizado por Jesús David Badillo Herrera, Arlex Chaves e José Augusto Fuentes Osorio \cite{balancecontrol}, foca em uma solução prática para enfrentar os desafios apresentados por erros em sistemas de distribuição de gás natural. A ferramenta combina técnicas numéricas e estatísticas, como a Reconciliação de Dados (DR) e a Detecção de Erros Brutos (GED), para melhorar a precisão das medições e garantir a conformidade com as leis de conservação de massa.

Essa abordagem não apenas aprimora a acurácia das medições, mas também identifica e corrige erros grosseiros, proporcionando maior confiabilidade nos dados. A validação da ferramenta com problemas da literatura e sua aplicação em uma rede de distribuição real demonstram sua eficácia em fornecer resultados reconciliados com alta precisão.

A abordagem combinada de Reconciliação de Dados (DR) e Detecção de Erros Brutos (GED), utilizada na distribuição de gás natural, foi ser integrada ao RADARE para garantir maior precisão nas medições e identificar rapidamente inconsistências ou erros em sistemas industriais.

% ---
\section{Justificativa}

O setor industrial brasileiro, responsável por uma parcela significativa da economia nacional, enfrenta desafios críticos relacionados à qualidade e confiabilidade dos dados gerados em processos produtivos. A falta de precisão nas medições e a ausência de ferramentas eficazes para a reconciliação de dados comprometem a eficiência operacional e a competitividade das indústrias. Dessa forma, torna-se urgente o desenvolvimento de soluções tecnológicas que possam superar essas limitações, sendo o \textit{software} RADARE uma proposta que visa preencher essa lacuna com inovação e eficácia.

A escolha por desenvolver o RADARE como uma aplicação \textit{web} é estratégica, pois o ambiente web oferece diversas vantagens essenciais para o setor industrial. Entre elas, a interoperabilidade eficiente entre sistemas permite integrar o \textit{software} com outras ferramentas já utilizadas nas plantas industriais. O acesso remoto possibilita que engenheiros e operadores possam monitorar e controlar os processos de qualquer localidade, reduzindo tempos de inatividade e melhorando a resposta a problemas. Além disso, a facilidade de uso da plataforma, associada à alta acessibilidade para indivíduos com necessidades especiais, amplia o alcance da solução, garantindo que o sistema possa ser adotado por uma gama diversa de usuários.

O projeto RADARE visa atender a uma demanda crescente por soluções no setor industrial brasileiro, onde a competitividade global exige maior eficiência e qualidade nas operações. O contexto brasileiro apresenta desafios específicos, como a falta de ferramentas acessíveis e adaptadas à realidade das indústrias locais. O RADARE se diferencia por ser uma solução criada para o contexto nacional, ao mesmo tempo em que incorpora tecnologias de ponta e metodologias comprovadas internacionalmente, como o método dos multiplicadores de Lagrange, aplicado de forma inovadora em uma plataforma web de fácil acesso e integração.

O RADARE representa uma inovação ao unir teorias avançadas de reconciliação de dados com tecnologias modernas de desenvolvimento \textit{web}. Ao criar uma solução que é tanto acessível quanto adaptável às necessidades do setor industrial, o RADARE tem o potencial de impactar diretamente a eficiência das operações industriais no Brasil, promovendo o desenvolvimento tecnológico e aumentando a competitividade das empresas brasileiras no cenário global.

\subsection{Organização do trabalho de conclusão de curso}

Este trabalho está estruturado da seguinte forma: no \textbf{Capítulo 1}, apresenta-se a introdução, contextualizando o problema e destacando os objetivos principais e secundários da pesquisa. O \textbf{Capítulo 2} oferece o referencial teórico, com definições e descrições das principais teorias que embasam a solução proposta. No \textbf{Capítulo 3}, descrevem-se os métodos e técnicas utilizados para o desenvolvimento do sistema, assim como o cronograma de atividades apresentado por meio do diagrama de Gantt. No \textbf{Capítulo 4}, são apresentados os resultados obtidos e uma análise dos dados obtidos ao longo do desenvolvimento. Por fim, no \textbf{Capítulo 5}, são apresentadas as conclusões, destacando o aprendizado adquirido, as reflexões sobre o desenvolvimento futuro do projeto e as contribuições gerais do trabalho.
       % Capítulo 1: Introdução
% \mychapter{Referencial teórico}
\label{Cap:ReferencialTeorico}

Este capítulo aborda a prática da reconciliação de dados, fundamental para garantir a consistência e precisão das informações coletadas em diferentes contextos científicos e industriais e utilizada como base no desenvolvimento do \textit{software}. É detalhada sua definição, evolução e aplicação na indústria, destacando sua contribuição para a otimização de processos e a análise proativa. 
    
Além disso, é discutido o método dos multiplicadores de Lagrange como uma abordagem matemática eficaz para resolver problemas complexos de reconciliação de dados. Por fim, a sinergia entre a indústria e a internet é contextualizada, evidenciando o papel crucial do \textit{software} na atualidade.

\section{Reconciliação de dados}
\subsection{Definição}

A reconciliação de dados é uma prática de tratamento de dados em diversos campos da ciência e da engenharia, visando garantir a consistência e a precisão dos dados coletados a partir de diferentes fontes \cite{datarecshakar}. Em sua essência, a reconciliação de dados consiste no processo de comparar e corrigir dados divergentes ou inconsistentes, a fim de garantir que todas as informações disponíveis estejam alinhadas e concordantes. Esse processo é essencial em contextos nos quais múltiplas fontes de dados são utilizadas, como em sistemas industriais, processos químicos, redes de sensores e modelagem ambiental \cite{datarecshakar}.
        
A reconciliação de dados envolve a aplicação de técnicas estatísticas e matemáticas para identificar e corrigir discrepâncias entre os dados observados e os valores esperados, com base em modelos e relações conhecidas \cite{datarecragnoli}. Isso pode incluir a detecção e correção de erros sistemáticos ou aleatórios, a estimativa de valores ausentes e a harmonização de diferentes tipos de dados. Ao garantir a consistência e a confiabilidade dos dados, a reconciliação de dados permite uma análise mais precisa e uma tomada de decisão mais fundamentada em uma variedade de contextos aplicados \cite{datarecragnoli}.
    
Em suma, a reconciliação de dados desempenha um papel crucial na garantia da qualidade e integridade dos dados em diferentes áreas de aplicação. Ao alinhar e corrigir dados discrepantes, essa prática permite uma análise mais confiável e uma interpretação mais precisa dos fenômenos estudados, contribuindo assim para avanços significativos em diversos campos da ciência e da engenharia \cite{datarecshakar}.
    
\subsection{Reconciliação de dados no âmbito industrial}
    
No começo da década de 1960 se entendeu a importância do controle de processos químicos industriais por computadores que aplicavam técnicas matemáticas \cite{computecontrol}, dessa forma surge a área da computação voltada à reconciliação de dados, na qual há a comparação, validação e correção de informações coletadas em diferentes pontos do processo, a fim de determinar a consistência dos dados, a qualidade dos mesmos ou até otimizar os processos \cite{datarecshakar}.
    
Ao longo das últimas décadas, os métodos de reconciliação de dados evoluíram significativamente, acompanhando os avanços tecnológicos e as demandas crescentes da indústria \cite{datarecsurvey}. Com o advento de sistemas de automação mais avançados, sensores inteligentes e a proliferação de dispositivos conectados, a quantidade de dados gerados nas operações industriais aumentou drasticamente \cite{datarecsurvey}. Nesse contexto, a reconciliação, análise e gestão de dados tornaram-se ferramentas indispensáveis para garantir a integridade e a confiabilidade das informações coletadas em tempo real \cite{aularecon}.
    
Na era contemporânea, a reconciliação de dados desempenha um papel crucial na otimização dos processos industriais, contribuindo para a eficiência operacional e a redução de custos. Sistemas avançados de reconciliação não apenas comparam e validam dados, mas também utilizam algoritmos sofisticados para identificar padrões, tendências e anomalias \cite{datarecragnoli}. Essa capacidade analítica permite que as indústrias ajam proativamente, antecipando-se a problemas potenciais, otimizando a produção e melhorando a qualidade dos produtos finais \cite{datarecshakar}.

\subsection{Reconciliação de dados pelo método de minimização de multivariáveis por multiplicadores de Lagrange}
    
No sistema em questão, a reconciliação de dados vai ser feita com a minimização de funções multivariáveis utilizando o método dos multiplicadores de Lagrange, desenvolvido pelo matemático Joseph Louis Lagrange (1739-1813), que desenvolveu um método de encontrar o mínimo ou máximo de uma função multivariável sujeita a uma ou várias condições de restrição \cite{lagrangebio}.
    
Nesse contexto, a aplicação do método dos multiplicadores de Lagrange destaca-se como uma abordagem matemática sofisticada e eficaz para resolver problemas complexos de reconciliação de dados \cite{optimizationlagrange}. A técnica proporciona uma estrutura robusta para lidar com situações em que é necessário otimizar uma função multivariável sujeita a restrições específicas. Ao utilizar os multiplicadores de Lagrange, o sistema ganha a capacidade de encontrar soluções que atendam simultaneamente às condições impostas, proporcionando uma reconciliação precisa e eficiente dos dados envolvidos \cite{aularecon}. Essa metodologia, fundamentada em princípios matemáticos sólidos, contribui para aprimorar a qualidade e a confiabilidade dos resultados obtidos, tornando-a uma ferramenta valiosa para a solução proposta no trabalhos.
        
\section{Desenvolvimento de um \textit{software online}}
\subsection{Desenvolvimento \textit{Front-End}}
    
A interface de usuário de um \textit{software}, também denominada de \textit{front-end}, desempenha um papel crucial no desenvolvimento de sistemas modernos \cite{eloquentjavascript}. É responsável por criar a experiência do usuário, tornando a interação com o \textit{software} intuitiva, eficiente e agradável. Compreende todos os elementos visíveis e interativos de uma aplicação, como botões, menus, formulários e \textit{layouts}, que são projetados e desenvolvidos para fornecer uma interface acessível e funcional aos usuários \cite{frontendperfomance}.
    
As principais tecnologias aplicadas a essa área dentro do desenvolvimento \textit{web} são linguagens de programação como HTML, CSS, JavaScript e TypeScript \cite{webdevlang}. Essas tecnologias permitem aos desenvolvedores criar interfaces de usuário dinâmicas e responsivas, capazes de se adaptar a diferentes dispositivos e tamanhos de tela. Além disso, o \textit{front-end} envolve a utilização de \textit{frameworks}, \textit{runtimes} e bibliotecas importantes e poderosas, como React, Angular e Vue.js, possibilitando um desenvolvimento rápido e eficiente de interfaces de usuário modernas \cite{frontendperfomance}. Desta forma, o \textit{front-end} é uma parte essencial do processo de desenvolvimento de \textit{software}, desempenhando um papel fundamental na criação de sistemas interativos e centrados na experiência do usuário \cite{reactjs}.

\subsection{Desenvolvimento \textit{Back-End}}
    
Muitas vezes referido como a parte não visível ou o "cérebro" por trás de uma aplicação, o \textit{back-end} é igualmente crucial para o funcionamento de sistemas modernos \cite{backend}. Enquanto o \textit{front-end} é responsável pela parte visual e interativa da aplicação, o \textit{back-end} lida com a lógica de negócios, o armazenamento de dados e a comunicação com o servidor. Isso inclui operações como autenticação de usuários, processamento de dados, gerenciamento de sessões e acesso a banco de dados \cite{backenddevroles}.
    
As tecnologias empregadas no \textit{back-end} são diversas, incluido linguagens de programação como Python, Ruby, Java, PHP e Node.js. Estas linguagens são utilizadas em conjunto com \textit{frameworks} e bibliotecas específicas para o desenvolvimento \textit{web}, como Django, Ruby on Rails, Spring, Laravel e Express.js \cite{eloquentjavascript}. O \textit{back-end} também envolve a criação de APIs (Interfaces de Programação de Aplicativos) para permitir a comunicação eficiente entre o \textit{front-end}, outros sistemas e o banco de dados. Desta forma, o \textit{back-end} é responsável por garantir que a aplicação \textit{web} seja funcional, segura e capaz de processar grandes volumes de dados, desempenhando um papel crucial no sucesso de sistemas \textit{web} modernos \cite{javascriptframework}.
    
\subsection{Desenvolvimento de Banco de Dados}
    
No contexto do desenvolvimento de sistemas \textit{web}, o banco de dados desempenha um papel fundamental na organização e armazenamento de dados utilizados pela aplicação \cite{databasedepth}. Ele oferece uma estrutura organizada e eficiente para armazenar informações de forma persistente, garantindo que os dados sejam recuperados de maneira rápida e precisa, quando necessário. O banco de dados é responsável por gerenciar grandes volumes de dados, lidar com transações complexas e garantir a integridade e a segurança das informações armazenadas \cite{databasesql}.
    
Existem diversos tipos de bancos de dados, cada um com suas características e funcionalidades específicas. Os bancos de dados relacionais, como MySQL, PostgreSQL e SQL Server, utilizam tabelas para organizar os dados em linhas e colunas, facilitando consultas complexas e garantindo a consistência dos dados por meio de relações definidas \cite{databasesqlnet}. Por outro lado, os bancos de dados NoSQL, como MongoDB e Cassandra, oferecem uma abordagem mais flexível e escalável para o armazenamento de dados não estruturados, como documentos, gráficos e dados em tempo real \cite{databasedepth}.
    
Além disso, o banco de dados é essencial para a integridade e a consistência dos dados em uma aplicação \textit{web}. Ele permite que os desenvolvedores armazenem e recuperem informações de forma eficiente, garantindo que os dados sejam sempre precisos e atualizados. Por meio de consultas e transações, o banco de dados fornece uma camada de abstração entre a aplicação e os dados subjacentes, facilitando o acesso e a manipulação dos dados de forma segura e eficaz \cite{databasesql}. Desta forma, o banco de dados é um componente essencial do desenvolvimento de sistemas \textit{web}, garantindo que as informações sejam armazenadas e gerenciadas de maneira confiável e eficiente \cite{databasesqlmaster}.

\subsection{Servidor}
    
Na arquitetura de sistemas \textit{web}, o servidor desempenha um papel central na comunicação entre o cliente e o \textit{back-end}. Ele é responsável por receber as requisições dos clientes, processá-las e fornecer as respostas adequadas \cite{serverdummy}. Além disso, o servidor também gerencia o armazenamento e o acesso aos dados, garantindo que as informações sejam entregues de forma rápida e segura \cite{serverhost}.
    
As tecnologias utilizadas na área de servidor variam dependendo das necessidades específicas do projeto, mas geralmente incluem sistemas operacionais como Linux ou Windows Server, juntamente com servidores \textit{web} como Apache, Nginx ou Microsoft IIS \cite{serverapache}. Além disso, \textit{frameworks} e tecnologias como Docker, Kubernetes e AWS são frequentemente empregados para facilitar a implantação e o gerenciamento de servidores em escala. Em suma, a área de servidor é fundamental para o funcionamento eficiente e confiável de sistemas \textit{web}, garantindo a entrega de conteúdo e serviços de forma rápida, segura e escalável \cite{serversql}.
    
\section{A sinergia entre a indústria e a internet} 

O panorama atual de avanço da internet e a convergência entre a internet e o setor industrial representam um marco significativo na era da Engenharia de Computação \cite{industry4status}. Este fenômeno transformador tem sido impulsionado pela fusão das tecnologias da informação com os processos industriais, dando origem a conceitos como Indústria 4.0. 
    
No âmbito desta ferramenta, é aplicada a intersecção dessas duas esferas, onde os conceitos de práticas industriais, reconciliação, análise e qualidade de dados se integram à internet, na qual é extraído deles o seu maior forte, como uma maior integralidade com outros sistemas por meio de APIs (interfaces de programação de aplicativos), melhor interatividade entre os elementos do sistema, promovendo uma comunicação mais dinâmica e eficaz, aumento da eficiência operacional e facilidade na gestão de processos \cite{industry4}. Essa sinergia possibilita a criação de ecossistemas industriais mais conectados nos quais os dados relevantes podem ser tratados de forma segura e eficiente. \cite{industrybuild}.
    
O horizonte atual, delineado pelos recentes avanços tecnológicos e inovações sustenta a perspectiva otimista que as indústrias estão destinadas a experimentar um crescimento substancial no país \cite{industrychina}. A convergência entre a internet e o setor industrial representa um catalisador significativo para a modernização e eficiência operacional. A integração de práticas avançadas de desenvolvimento de soluções voltadas à usabilidade e ao ambiente de desenvolvimento com controle computacional, como a reconciliação de dados e análise preditiva, impulsiona a qualidade e consistência dos processos produtivos \cite{industrydigital}.
    
Além disso, a aplicação da internet nas práticas industriais não só fortalece a competitividade das empresas mas também desempenha um papel crucial na expansão econômica do país \cite{industryiot}. A capacidade de adotar tecnologias inovadoras como a automação avançada, coloca as indústrias em uma posição estratégica para atender às crescentes demandas do mercado e elevar a produtividade \cite{industryinternet}. Nesse sentido é plausível afirmar que diante do atual cenário tecnológico e das tendências emergentes, é indubitável a necessidade e importância da ferramenta desenvolvida nesse trabalho \cite{industrychina}.       % Capítulo 2: Referencial Teórico
% \mychapter{Metodologia}
\label{Cap:Metodologia}

Este capítulo aborda a metodologia aplicada no desenvolvimento do \textit{software}, quais métodos de engenharia de \textit{software} foram aplicados durante o processo, desde a concepção, até a prototipação

\section{Metodologia de Desenvolvimento do DRD}

O desenvolvimento do \textit{software} DRD adotou o método ágil Scrum como estrutura principal para orientar o processo de criação e entrega do produto \cite{softwareengreq}. Ele sendo uma metodologia ágil amplamente reconhecida, escolhido devido à sua capacidade de promover uma abordagem iterativa e flexível, especialmente adequada para uma equipe com um único desenvolvedor, como é o caso do desenvolvimento do DRD \cite{scrum}.

\subsection{Descrição do método Scrum}

Scrum é uma metodologia ágil de desenvolvimento de \textit{software} que enfatiza a entrega iterativa e incremental de funcionalidades. Apesar de ter sido projetado para equipes multidisciplinares, pode ser adaptado para uma equipe de um só desenvolvedor devido aos seus princípios flexíveis e foco na entrega contínua de valor \cite{scrumlove}.
        
No Scrum, o trabalho é dividido em ciclos de desenvolvimento chamados de \textit{sprints}, geralmente com duração de duas a quatro semanas. Cada \textit{sprints} começa com uma reunião de planejamento, onde o desenvolvedor define as metas e prioridades do \textit{sprints} com base nas necessidades do projeto \cite{scrummic}. Durante o \textit{sprint}, o desenvolvedor trabalha na implementação das funcionalidades definidas no planejamento. O progresso é monitorado diariamente em reuniões curtas chamadas de \textit{daily scrums}, onde o desenvolvedor atualiza a equipe sobre o seu progresso, identifica quaisquer impedimentos e ajusta o plano conforme necessário. Ao final de cada \textit{sprint}, o desenvolvedor realiza uma revisão do \textit{sprint}, demonstrando as funcionalidades concluídas à equipe ou ao cliente, e uma retrospectiva do \textit{sprint}, identificando o que funcionou bem e o que pode ser melhorado no próximo \textit{sprint} \cite{scrumproj}.
        
Esse método de desenvolvimento de \textit{software} é funcional mesmo para equipes de um só desenvolvedor, promovendo uma abordagem colaborativa, adaptativa e transparente, permitindo que o desenvolvedor mantenha um alto nível de visibilidade e controle sobre o progresso do projeto. A abordagem iterativa e incremental do Scrum também facilita a adaptação a mudanças nos requisitos do projeto e permite que o desenvolvedor entregue continuamente valor de forma mais rápida e frequente ao cliente, ou orientador, no caso do DRD \cite{scrum}.
        
\subsubsection{Aplicação do método Scrum no Desenvolvimento do \textit{Software} DRD}

Ao utilizar o Scrum, o desenvolvedor pôde organizar o trabalho em ciclos curtos e mensuráveis, conhecidos como "\textit{sprint}". Cada \textit{sprint}, com duração definida, permite ao desenvolvedor focar em metas claras e alcançáveis, priorizadas com base nas necessidades do projeto. Essa abordagem iterativa possibilitou uma resposta rápida a mudanças nos requisitos e uma entrega contínua de funcionalidades ao longo do desenvolvimento.
        
Para o desenvolvimento do \textit{software} DRD, o Scrum facilita a gestão eficiente do projeto, com reuniões diárias para monitorar o progresso, identificar possíveis obstáculos e ajustar o plano conforme necessário. Além disso, as reuniões de revisão de \textit{sprint} permitem uma demonstração transparente das funcionalidades desenvolvidas, garantindo uma comunicação eficaz com \textit{stakeholders} e a validação contínua do produto.
        
A aplicação do método Scrum no desenvolvimento do \textit{software} DRD não apenas proporciona uma estrutura organizacional clara, mas também promove uma cultura de colaboração e melhoria contínua. Ao enfatizar a transparência, a comunicação e o \textit{feedback}, o Scrum permite ao desenvolvedor adaptar-se rapidamente às mudanças no ambiente de desenvolvimento e priorizar o valor entregue ao cliente.
        
Em resumo, a escolha do método Scrum para o desenvolvimento do \textit{software} DRD demonstrou ser uma decisão acertada. Sua flexibilidade, foco na entrega de valor e capacidade de adaptação o tornaram uma escolha ideal para uma equipe de um único desenvolvedor, permitindo o desenvolvimento eficiente e eficaz de um produto de alta qualidade.
        
\section{Escopo do \textit{software}}

O escopo do projeto define os limites do trabalho a ser realizado, garantindo que todas as atividades estejam alinhadas com os objetivos do projeto. Isso proporciona uma base sólida para o planejamento, execução e controle do desenvolvimento do \textit{software}, permitindo que a concentração nas entregas essenciais para o projeto \cite{softwareeng}.

\subsection{Requisitos do Sistema}

Detalhe os requisitos funcionais e não funcionais do sistema de \textit{software}, identificados durante a fase de análise de requisitos. Explique como esses requisitos foram capturados e documentados \cite{softwareengreq}.
    
\subsubsection{Requisitos Funcionais}

Os requisitos funcionais no projeto de \textit{software} desempenham um papel crucial na definição das capacidades e funcionalidades que o sistema deve fornecer para atender às necessidades dos usuários. Em suma, eles representam o "o que" o sistema deve fazer. Esses requisitos são geralmente expressos em termos de casos de uso, cenários de interação do usuário ou fluxos de trabalho \cite{softwareengreq}.
        
A Tabela \ref{tab:req_funcional} demonstra os atuais requisitos funcionais do \textit{software}.

\begin{table}[htbp]
\begin{tabularx}{\linewidth}{|c|X|c|c|} \hline
\textbf{Identificador} & 
\textbf{Descrição} & 
\textbf{Prioridade} &
\textbf{Requisitos Relacionados}\\ \hline
RF01 & 
O sistema deve permitir que os usuários modelem a dinâmica dos sensores em uma planta industrial. & 
Alta & 
RF02 \\ \hline
RF02 & 
Os usuários devem ser capazes de alimentar o sistema com os dados gerados pelos sensores. & 
Alta & 
RF01 \\ \hline
\end{tabularx}

\caption{Tabela modelo dos requisitos funcionais.}
\label{tab:req_funcional}
\end{table}
            
% {\fontsize{10}{12}\selectfont \begin{longtable}
%     {| p{.15\textwidth} | p{.35\textwidth} | p{.20\textwidth} |  p{.20\textwidth} |} 
%     \hline
%     \textbf{Identificador} & \textbf{Descrição} & \textbf{Prioridade} & \textbf{Requisitos Relacionados} \\
%     \hline
%     RF01 & O sistema deve permitir que os usuários modelarem a dinâmica dos sensores presentes em uma planta industrial. & Alta & RF02 \\
%     \hline
%     RF02 & Os usuários devem ser capazes de alimentar o sistema com os dados gerados pelos sensores. & Alta & RF01 \\
%     \hline
%     RF03 & O sistema deve oferecer uma interface de usuário intuitiva e acessível. & Média & - \\
%     \hline
%     RF04 & O sistema deve realizar cálculos de reconciliação de dados de forma ágil e rápida. & Alta & RF01, RF02 \\
%     \hline
%     RF05 & O sistema deve garantir a compatibilidade de dados, mesmo com formatos heterogêneos. & Alta & RF01, RF02, RF03 \\
%     \hline
%     RF06 & Os usuários devem poder exportar os resultados da reconciliação de dados para diferentes formatos de arquivo. & Média & RF01, RF02, RF03, RF04, RF05 \\
%     \hline
%     RF07 & O sistema deve permitir aos usuários configurar alertas para notificar sobre eventos importantes relacionados aos dados dos sensores. & Alta & RF01, RF02 \\
%     \hline
%     RF08 & O sistema deve fornecer funcionalidades de visualização de dados em tempo real, incluindo gráficos e relatórios personalizáveis. & Alta & RF02, RF05 \\
%     \hline
%     RF09 & O sistema deve permitir a integração com outros sistemas de monitoramento industrial, facilitando a troca de dados e informações. & Alta & RF05, RF06 \\
%     \hline
% \caption{Tabela de Requisitos Funcionais} % needs to go inside longtable environment
% \label{tab:req_funcional}
% \end{longtable}}
        
\subsubsection{Requisitos Não Funcionais}

Os requisitos não funcionais em um projeto de \textit{software} desempenham um papel igualmente crucial, complementando os requisitos funcionais ao definir os critérios de qualidade, desempenho e restrições operacionais que o sistema deve atender. Enquanto os requisitos funcionais se concentram no "o que" o sistema deve fazer, os requisitos não funcionais delineiam "como" o sistema deve fazer isso, bem como outras características importantes que afetam sua operação e usabilidade e muitas vezes abordam características mais abstratas do sistema, como segurança, confiabilidade, escalabilidade, desempenho e usabilidade \cite{softwareengreq}. 
            
A Tabela \ref{tab:ReqNaoFuncional} demonstra os atuais requisitos não funcionais do \textit{software}.

{\fontsize{10}{12}\selectfont \begin{longtable}
{| p{.15\textwidth} | p{.45\textwidth} | p{.10\textwidth} |} 
    \hline
    \textbf{Identificador} & \textbf{Descrição} & \textbf{Prioridade} \\
    \hline
    RNF01 & O sistema deve ser altamente escalável para lidar com um grande volume de dados de sensores. & Alta \\
    \hline
    RNF02 & A segurança dos dados deve ser uma prioridade, garantindo proteção contra acesso não autorizado e manipulação indevida. & Alta \\
    \hline
    RNF03 & O desempenho do sistema deve ser otimizado para garantir tempos de resposta rápidos, mesmo em momentos de pico de uso. & Alta \\
    \hline
    RNF04 & O sistema deve ser facilmente configurável e customizável para atender às necessidades específicas de diferentes ambientes industriais. & Média \\
    \hline
    RNF05 & A manutenibilidade do sistema deve ser uma consideração fundamental, facilitando atualizações, correções de bugs e modificações futuras. & Média \\
    \hline
    RNF06 & A usabilidade do sistema deve ser intuitiva, permitindo uma curva de aprendizado mínima para os usuários. & Média \\
    \hline
    RNF07 & O sistema deve ser compatível com diferentes navegadores, garantindo sua acessibilidade em uma variedade de ambientes de implantação. & Alta \\
    \hline
    RNF8 & O sistema deve estar em conformidade com regulamentações de privacidade de dados, como GDPR, garantindo o tratamento adequado e a proteção das informações pessoais dos usuários. & Alta \\
    \hline
    RNF9 & A tolerância a falhas do sistema deve ser implementada, garantindo a continuidade das operações mesmo em caso de falhas de componentes individuais. & Alta \\
    \hline
    RNF10 & O tempo de resposta do sistema deve ser consistente e previsível, independentemente da carga de trabalho ou do número de usuários simultâneos. & Média \\
    \hline
\caption{Tabela de Requisitos Não Funcionais} % needs to go inside longtable environment
\label{tab:ReqNaoFuncional}
\end{longtable}}

\subsection{Temporização do Desenvolvimento do \textit{Software}}

O Gráfico de Gantt é uma ferramenta amplamente utilizada no gerenciamento de projetos para visualizar e acompanhar o progresso das atividades ao longo do tempo. Desenvolvido pelo engenheiro Henry Gantt na década de 1910, este gráfico fornece uma representação visual clara das tarefas do projeto, seus prazos e suas interdependências \cite{ganttchart}. 

Com sua disposição em forma de barras horizontais ao longo de um eixo de tempo, o gráfico permite que os gerentes de projeto e suas equipes identifiquem facilmente as datas de início e término de cada atividade, bem como as sobreposições e lacunas entre elas.
    
\begin{figure}[h]
    \centering
    \includegraphics[width=\textwidth]{figuras/DRD-Ganttt.pdf} % Replace "example.pdf" with the path to your PDF file
    \caption{Grafico de Gantt para desenvolvimento do \textit{software} DRD.}
    \label{fig:ganttChart}
\end{figure}    

Na Figura \ref{fig:ganttChart} é visível Gráfico de Gantt utilizado pelo projeto, onde ele é separado nas seguintes partes: 
    
\begin{itemize}
    \item \textbf{Planejamento e Análise Inicial:} Esta fase inclui a identificação dos requisitos, definição dos objetivos, análise de viabilidade técnica e econômica, além da elaboração dos casos de uso e da documentação.
    
    \item \textbf{\textit{Design} e Prototipagem:} Nesta etapa, são criados o design de interface do usuário (UI) e o design de experiência do usuário (UX), bem como a definição da arquitetura do sistema e a prototipagem com revisões iterativas do design.
    
    \item \textbf{Desenvolvimento e Implementação:} Aqui ocorre a implementação do \textit{front} e \textit{back-end}, o desenvolvimento das funcionalidades principais do sistema, os testes unitários e a integração contínua, além da revisão e \textit{feedback} com os \textit{stakeholders}.
    
    \item \textbf{Testes e Garantia de Qualidade:} Esta fase abrange os testes de sistema abrangentes, a identificação e correção de \textit{bugs}, e a realização de testes de segurança para garantir a qualidade do sistema.
    
    \item \textbf{Preparação para Lançamento:} Envolve a preparação do ambiente de produção, o treinamento de usuários finais e o lançamento suave do sistema para garantir uma transição tranquila para os usuários.
    
    \item \textbf{Suporte e Manutenção Pós-Lançamento:} Por fim, inclui o monitoramento contínuo do sistema, atualizações regulares de \textit{software} e fornecimento de suporte técnico aos usuários para garantir o bom funcionamento e a satisfação contínua dos clientes.
\end{itemize}

Desta forma, a utilização do gráfico de Gantt foi crucial, por manter uma organização temporal da produção em razão do tempo decorrido, na qual auxiliou no planejamento e coordenação das diferentes etapas do projeto, permitindo também uma rápida avaliação do progresso do projeto, destacando visualmente as atividades concluídas, as em andamento e as pendentes. Auxiliando a identificar áreas de atraso ou potenciais gargalos, permitindo que a tomada de atitudes corretivas para manter o projeto no caminho certo.
        
\section{Projeto de \textit{Software}}

Descreva o processo de \textit{design} do \textit{software}, incluindo a arquitetura geral do sistema, diagramas de classe, diagramas de sequência, entre outros artefatos de \textit{design}. Explica as decisões de \textit{design} tomadas e como elas estão alinhadas com os requisitos do sistema \cite{softwareeng}.

\subsection{Arquitetura Geral do Sistema}

Durante o desenvolvimento de um \textit{software}, é crucial o uso de representações visuais para uma compreensão mais clara das funcionalidades e estrutura do projeto. Assim, a linguagem de modelagem unificada (UML) permite a criação de diversos tipos de diagramas para representar diferentes aspectos do sistema. Para a solução deste programa, optou-se pelo uso do PlantUML \cite{plantumldoc}, que possibilita a modelagem dos processos através de código, facilitando a alteração e atualização dos diagramas \cite{softwareengreq}.

\subsubsection{Diagrama de Classe}

O diagrama de classes é uma representação visual do projeto na engenharia de \textit{software}, empregada para descrever a estrutura estática de um sistema baseado em objetos \cite{softwareenguml}. Na sua forma mais básica, um diagrama de classes consiste em classes, que são os blocos de construção de um sistema orientado a objetos, contendo os atributos que as características ou propriedades dos objetos dessa classe, enquanto os métodos indicam as operações que podem ser realizadas por esses objetos, como é o caso do diagrama da Figura \ref{fig:ClassDiagram}. Essa estrutura fornece uma representação abstrata e organizada das entidades do sistema, permitindo uma compreensão clara de sua composição e funcionalidades.
            
\begin{figure}[htb]
    \caption{\label{fig:ClassDiagram}Diagrama de Classe em UML.}
    \begin{center}
        \includegraphics[width=\textwidth]{figuras/ClassDiagram.png}
    \end{center}
\end{figure}
            
Adicionalmente, os relacionamentos entre as classes são destacados por meio de linhas que conectam os blocos das classes. Essas associações podem assumir diferentes formas, como associações simples, agregações, composições, heranças, entre outras, e fornecem \textit{insights} valiosos sobre como as diferentes partes do sistema interagem e dependem umas das outras.
            
Assim, um diagrama de classes torna-se uma ferramenta essencial para modelar a estrutura estática de um sistema, oferecendo uma representação visual nítida das classes, atributos, métodos e seus inter-relacionamentos, o que simplifica o processo de \textit{design}, análise e comunicação entre os integrantes da equipe de desenvolvimento de \textit{software} \cite{softwareenguml}.
            
\subsubsection{Diagrama de Caso de Uso}
        
O diagrama de caso de uso é uma representação visual amplamente utilizada na engenharia de \textit{software} para descrever a interação entre um sistema e seus usuários. Ele destaca os diferentes casos de uso, que representam as diferentes maneiras pelas quais os usuários interagem com o sistema para atingir seus objetivos. Na sua forma mais básica, um diagrama de caso de uso consiste em atores, que são os usuários externos ao sistema, e de casos de uso, que são as diferentes funcionalidades ou serviços oferecidos pelo sistema, como exemplificado no diagrama da Figura \ref{fig:UseCaseDiagram}.
        
\begin{figure}[htb]
    \caption{\label{fig:UseCaseDiagram}Diagrama de Caso de Uso em UML.}
    \begin{center}
        \includegraphics[width=\textwidth]{figuras/UseCaseDiagram.png}
    \end{center}
\end{figure}
        
Os atores representam os diferentes tipos de usuários que interagem com o sistema, enquanto os casos de uso representam as diferentes funcionalidades oferecidas pelo sistema. Esses casos de uso são conectados aos atores por meio de linhas, indicando a interação entre os usuários e as funcionalidades do sistema.
        
Assim, um diagrama de caso de uso torna-se uma ferramenta essencial para modelar a interação entre um sistema e seus usuários, oferecendo uma representação visual clara dos atores, casos de uso e seus inter-relacionamentos. Isso simplifica o processo de \textit{design}, análise e comunicação entre os membros da equipe de desenvolvimento de \textit{software}, garantindo uma implementação eficaz e orientada às necessidades dos usuários.
        
\subsubsection{Diagrama do Banco de Dados}

O diagrama de banco de dados é uma representação visual das estruturas de dados e dos relacionamentos entre elas em um sistema de banco de dados \cite{databasedepth}, na qual visa representar a estrutura dos dados armazenados em um banco de dados e como esses dados estão relacionados entre si, como é o caso da Figura \ref{fig:DatabaseDiagram}, que representa o banco de dados do sistema.
            
\begin{figure}[htb]
    \caption{\label{fig:DatabaseDiagram}Diagrama de Banco de Dados em UML.}
    \begin{center}
        \includegraphics[height=0.25\textheight]{figuras/DatabaseDiagram.png}
    \end{center}
\end{figure}
            
No diagrama de banco de dados, as entidades são representadas por tabelas, onde cada tabela possui colunas que representam os atributos dos dados armazenados. Além disso, as relações entre as tabelas são representadas por meio de chaves estrangeiras, indicando como os dados de uma tabela estão relacionados aos dados de outra tabela.
            
Desta forma, um diagrama de banco de dados é uma ferramenta vital para modelar a estrutura dos dados em um sistema de banco de dados, oferecendo uma representação visual clara das tabelas, colunas e relacionamentos entre elas. Isso facilita o projeto, análise e comunicação entre os membros da equipe de desenvolvimento de \textit{software}, garantindo uma implementação eficiente e eficaz do sistema de banco de dados.
    
\section{Implementação}

O projeto utilizou-se da ferramenta Node.js para gerenciamento de pacotes, permitindo uma gestão eficiente das dependências do projeto. Também foram empregados diversos \textit{plugins} e bibliotecas complementares para facilitar o desenvolvimento e melhorar a experiência do usuário.

O \textit{front-end} do projeto foi implementado utilizando React.js como a principal biblioteca de desenvolvimento de interfaces de usuário. Para estilização, foram empregados arquivos CSS com metodologia BEM (Block Element Modifier) para garantir uma estrutura de estilo escalável e modular. Além disso, o projeto se beneficiou do uso de diversos componentes reutilizáveis para manter um código limpo e organizado \cite{eloquentjavascript}.

Uma outra ferramenta muito utilizada, ainda no \textit{front-end} foi a biblioteca ReactFlow, de criação de diagramas, que foi adaptada para a solução em questão.
    
\section{Gerenciamento de Configuração e Mudança}

Durante o ciclo de vida do desenvolvimento de \textit{software}, foram adotadas diversas práticas e ferramentas para garantir um controle efetivo de versão e gerenciamento de mudanças \cite{gitevery}.
    
Para controle de versão, o Git foi escolhido como sistema de controle de versão distribuído. A plataforma Github \cite{github} foi utilizada para armazenar os repositórios do código-fonte do \textit{software}, o código LaTeX da Trabalho de Conclusão de Curso e da Apresentação. Isso permitiu que a existência de um histórico detalhado de todas as alterações realizadas no código.
    
Além disso, foram estabelecidas políticas de \textit{branches} no Git, como o uso de \textit{branches} de \textit{feature}, \textit{develop} e \textit{main}, para organizar o fluxo de trabalho e facilitar a integração contínua. As alterações no código eram revisadas antes de serem mescladas nos \textit{branches} principais, garantindo a qualidade e consistência do código.
    
Para o gerenciamento de mudanças, uma abordagem baseada em metodologias ágeis foi adotada, utilizando o Scrum. Isso permitiu que as mudanças fossem gerenciadas de forma iterativa e incremental, com entregas frequentes e \textit{feedback} contínuo ao orientador do trabalho.
    
Além disso, foi se utilizado o Trello como ferramenta de rastreamento de problemas \textit{(issue tracking)}, para registrar e acompanhar as mudanças, correções de \textit{bugs} e novas funcionalidades ao longo do desenvolvimento. Isso proporcionou uma visão clara do progresso do projeto e facilitou a comunicação dentro da equipe.
    
Desta forma, o controle de versão e gerenciamento de mudanças foram fundamentais para garantir a integridade, rastreabilidade e qualidade do \textit{software} ao longo de seu desenvolvimento, permitindo uma um comportamento eficaz da e uma entrega de progresso contínua no desenvolvimento do \textit{software}.      % Capítulo 3: Metodologia
% \mychapter{Resultados} 
\label{Cap:Resultados}

Neste capítulo, são apresentados os resultados da implementação e testes do \textit{software} RADARE, com foco na precisão dos dados reconciliados, no desempenho computacional em diferentes cenários de carga e na usabilidade da ferramenta, tanto no \textit{front-end} (menu e \textit{canvas}) quanto no \textit{back-end} (rotas, serviços e banco de dados). Além disso, foram desenvolvidos dois manuais: um para a manutenção técnica do sistema e outro para orientar o uso pelos usuários finais.

Para facilitar a leitura e não quebrar a dinâmica do texto principal, todos os códigos desenvolvidos para o RADARE foram incluídos nos Apêndices. Devido à sua extensão, a decisão de alocar os trechos de código nos Apêndices permite uma melhor fluidez no corpo do trabalho. Sempre que um código for citado ao longo do texto, será indicada a referência ao apêndice correspondente para consulta detalhada.

% -------------------------
\section{Resultados do desenvolvimento do \textit{front-end}} 

O \textit{front-end} do projeto  foi desenvolvido com foco em oferecer uma interface intuitiva, visando otimizar a interação dos usuários com a ferramenta de modelagem. A interface facilita a visualização dos fluxos de dados e dos modelos, além de permitir a execução da reconciliação. O  \textit{front-end} do sistema é estruturado em duas áreas principais: o menu, responsável pelo gerenciamento das ações e funcionalidades, e o \textit{canvas}, onde os nódulos conectados podem ser visualizados e manipulados diretamente pelos usuários.

% -------------------------
\subsection{Menu de controle da interface gráfica} 

A sessão de menu do RADARE apresenta ao usuário uma interface de fácil interação, permitindo a adição de componentes, o controle do fluxo de dados e o gerenciamento da visualização geral do sistema. Cada funcionalidade disponível no menu é descrita de maneira detalhada nas subseções a seguir, acompanhada por exemplos de código e imagens que ilustram sua implementação no contexto da ferramenta.

A biblioteca \textit{ReactFlow} \cite{reactflow} é fundamental para a manipulação dos nódulos no \textit{canvas} e foi significativamente adaptada para atender às necessidades específicas do projeto. As modificações realizadas garantem uma usabilidade eficiente, permitindo que os usuários adicionem e conectem os nódulos de forma dinâmica, com fluidez e precisão.

Cada nódulo inserido no \textit{canvas} possui uma estrutura personalizada, onde as conexões, denominadas \textit{handles}, são configuradas com características específicas, como estilo visual e lógica de interação. Essa personalização assegura que a visualização dos fluxos de dados seja clara e que sua manipulação seja intuitiva, facilitando o gerenciamento das operações dentro do sistema.

A Figura \ref{Fig:MenuImage} apresenta o menu principal do sistema, destacando as opções para adição de nódulos ao \textit{canvas}. Através desse menu, o usuário pode inserir diferentes tipos de nódulos, como entradas, saídas e pontos de processamento de dados, além de opções como reconciliação de dados e ajustes de visualização. Cada funcionalidade foi projetada para que o usuário possa construir fluxos de dados industriais de maneira modular e interativa, proporcionando maior flexibilidade no gerenciamento e análise de grandes volumes de dados.

\begin{figure}[htbp]
    \centering
    \includegraphics[width=0.4\textwidth]{figuras/menu-image.png}
    \caption{Menu principal do sistema RADARE.}
    \label{Fig:MenuImage}
\end{figure}

\subsubsection{Adicionar Input}

O botão \textbf{"Adicionar Input"} permite ao usuário inserir um novo nódulo de entrada no \textit{canvas}, representando um sensor ou uma fonte de dados no sistema industrial. Ao acionar esse botão, um nó de \textit{input} é adicionado ao \textit{canvas}, possibilitando a conexão desse ponto com outros nódulos do fluxo de dados. A implementação dessa funcionalidade utiliza a biblioteca ReactFlow \cite{reactflow}, o que elimina a necessidade de configurações customizadas iniciais e facilita a criação e manipulação dos nódulos no ambiente visual.

A Figura \ref{Fig:AddInputButton} ilustra o botão "Adicionar Input" na interface gráfica do sistema.

\begin{figure}[htbp]
    \centering
    \includegraphics[width=0.4\textwidth]{figuras/add-input-button.png}
    \caption{Botão de adicionar input no menu (Fonte: próprio autor, 2024).}
    \label{Fig:AddInputButton}
\end{figure}


\subsubsection{Adicionar output}

O botão \textbf{"Adicionar Output"} permite ao usuário inserir um novo nódulo de saída no \textit{canvas}. Esse \textit{output} representa um destino ou ponto final para os dados no sistema industrial, como a exportação de resultados processados ou a visualização de dados reconciliados. Ao acionar o botão, um novo nó de \textit{output} é adicionado ao \textit{canvas}, possibilitando sua conexão com outros nódulos de processamento ou entrada no fluxo de dados. Assim como ocorre com o \textit{input}, a implementação dessa funcionalidade também utiliza a biblioteca ReactFlow \cite{reactflow}, eliminando a necessidade de customização de comportamentos iniciais.

A Figura \ref{Fig:AddOutputButton} mostra o botão "Adicionar Output" na interface, permitindo ao usuário inserir nódulos de saída no fluxo de dados.

\begin{figure}[htbp]
    \centering
    \includegraphics[width=0.4\textwidth]{figuras/add-output-button.png}
    \caption{Botão de adicionar output no menu (Fonte: próprio autor, 2024).}
    \label{Fig:AddOutputButton}
\end{figure}

% -------------------------
\subsubsection{Adicionar nódulo 1-1}

O botão \textbf{"Adicionar Nódulo 1-1"} permite ao usuário inserir um novo nódulo de transição no \textit{canvas}. Esse nódulo atua como um ponto intermediário no fluxo de dados, podendo representar sensores, transformações ou outros elementos de processo. Ao clicar no botão, o nódulo é adicionado ao \textit{canvas}, permitindo ao usuário conectá-lo com outros nódulos de forma eficiente.

A lógica para adicionar este nódulo foi desenvolvida de forma personalizada, especificando o tipo de conexão, a quantidade de pontos de conexão (\textit{handles}), além do estilo visual e da posição no \textit{canvas}, garantindo que o comportamento do nódulo se ajuste adequadamente ao fluxo de dados esperado.

O trecho principal do código responsável pela criação desse nódulo, que pode ser encontrado em sua totalidade no \textbf{Anexo \ref{Anexo:frontCodeNodeOneOne}}.
ulo 1-1" disponível no menu do sistema.

\begin{figure}[htbp]
    \centering
    \includegraphics[width=0.4\textwidth]{figuras/add-node11-button.png}
    \caption{Botão de adicionar Nódulo 1-1 no menu (Fonte: próprio autor, 2024).}
    \label{Fig:AddNodeOneOneButton}
\end{figure}

% -------------------------
\subsubsection{Adicionar nódulo 1-2}

O botão \textbf{"Adicionar Nódulo 1-2"} permite ao usuário inserir um nódulo de transição que recebe uma única entrada e gera duas saídas no \textit{canvas}. Este tipo de nódulo é particularmente útil em cenários onde um único ponto de dados precisa ser bifurcado para diferentes processos ou análises. Ao ser adicionado, o nódulo facilita o roteamento de dados para dois fluxos distintos, mantendo a integridade e a flexibilidade do processo.

A lógica para este nódulo foi customizada para suportar uma conexão de entrada e duas saídas, com o código responsável definindo os \textit{handles} (pontos de conexão), a posição e o estilo visual no \textit{canvas}. Assim como no caso do nódulo 1-1, o comportamento é ajustado para garantir uma integração fluida no fluxo de dados.

O trecho principal do código responsável pela criação desse nódulo pode ser encontrado em sua totalidade no \textbf{Anexo \ref{Anexo:frontCodeNodeOneTwo}}.

A Figura \ref{Fig:AddNodeOneTwoButton} ilustra o botão "Adicionar Nódulo 1-2" disponível no menu do sistema.

\begin{figure}[htbp]
    \centering
    \includegraphics[width=0.4\textwidth]{figuras/add-node12-button.png}
    \caption{Botão de adicionar Nódulo 1-2 no menu (Fonte: próprio autor, 2024).}
    \label{Fig:AddNodeOneTwoButton}
\end{figure}

\subsubsection{Adicionar nódulo 2-1}

O botão \textbf{"Adicionar Nódulo 2-1"} permite ao usuário inserir um nódulo de transição que recebe duas entradas e gera uma única saída no \textit{canvas}. Esse nódulo é ideal para processos em que múltiplas fontes de dados precisam ser combinadas ou integradas antes de continuar o fluxo. Ao ser adicionado, o nódulo permite a fusão de duas linhas de dados, garantindo que as informações de entrada sejam processadas de forma conjunta antes de seguirem para a próxima etapa.

A lógica para este nódulo foi desenvolvida de maneira personalizada, permitindo a adição de dois pontos de conexão de entrada e um ponto de saída. O código responsável configura os \textit{handles}, define o estilo visual e posiciona o nódulo no \textit{canvas}, assegurando que ele atenda às necessidades de integração e processamento combinados dentro do fluxo de dados.

O trecho principal do código responsável pela criação desse nódulo pode ser encontrado em sua totalidade no \textbf{Anexo \ref{Cap:NodeTwoOneCode}}.

A Figura \ref{Fig:AddNodeTwoOneButton} mostra o botão "Adicionar Nódulo 2-1" presente no menu da interface do sistema.

\begin{figure}[htbp]
    \centering
    \includegraphics[width=0.4\textwidth]{figuras/add-node21-button.png}
    \caption{Botão de adicionar Nódulo 2-1 no menu (Fonte: próprio autor, 2024).}
    \label{Fig:AddNodeTwoOneButton}
\end{figure}
\subsubsection{Reconciliar dados}

O botão \textbf{"Reconciliar Dados"} executa o processo de reconciliação dos dados conectados no \textit{canvas}. Ao ser acionado, o sistema analisa os nódulos interconectados e realiza a reconciliação dos dados utilizando o método dos multiplicadores de Lagrange. Esse processo ajusta as discrepâncias entre os valores medidos e os valores reconciliados, garantindo que as restrições impostas pelos balanços de massa e energia sejam respeitadas.

A lógica por trás desse botão foi desenvolvida para percorrer os nódulos conectados no \textit{canvas}, extrair os dados necessários e enviá-los ao back-end. No back-end, o algoritmo de reconciliação é executado, processando os dados conforme as regras definidas, e os resultados são retornados ao front-end, onde os dados reconciliados são exibidos no fluxo visual do \textit{canvas}.

O trecho principal do código responsável por essa funcionalidade pode ser encontrado em sua totalidade no \textbf{Anexo \ref{Cap:ReconcileDataCode}}.

A Figura \ref{Fig:ReconcileButton} ilustra o botão "Reconciliar Dados" na interface do sistema.

\begin{figure}[htbp]
    \centering
    \includegraphics[width=0.4\textwidth]{figuras/reconcile-data-button.png}
    \caption{Botão de reconciliar dados no menu (Fonte: próprio autor, 2024).}
    \label{Fig:ReconcileButton}
\end{figure}

\subsubsection{Esconder gráfico das reconciliações}

O botão \textbf{"Esconder Gráfico das Reconciliações"} permite ao usuário ocultar o gráfico que exibe os resultados das reconciliações de dados, proporcionando uma interface mais organizada e com maior espaço para outros elementos do processo. Ao ativar essa função, o gráfico é temporariamente removido do \textit{dashboard}, mas os dados reconciliados permanecem disponíveis no sistema, permitindo que o usuário possa reexibi-los quando necessário. A lógica implementada para esse botão alterna a visibilidade do gráfico sem interferir nos demais componentes ou no fluxo dos dados processados. O código responsável por essa funcionalidade pode ser encontrado no \textbf{Anexo \ref{Anexo:HideGraphLogic}}.

A Figura \ref{Fig:HideGraphButton} ilustra o botão "Esconder Gráfico das Reconciliações" na interface gráfica.

\begin{figure}[htbp]
    \centering
    \includegraphics[width=0.4\textwidth]{figuras/hide-graphbar-button.png}
    \caption{Botão de esconder gráfico das reconciliações (Fonte: próprio autor, 2024).}
    \label{Fig:HideGraphButton}
\end{figure}

\subsubsection{Esconder \textit{sidebar} de informações}

O botão \textbf{"Esconder Sidebar de Informações"} permite ao usuário ocultar a barra lateral que exibe informações detalhadas sobre os nódulos e fluxos no \textit{canvas}. Essa barra lateral geralmente contém dados importantes e estatísticas sobre os elementos do processo, sendo especialmente útil para diagnósticos e ajustes detalhados. Ao escondê-la, o usuário ganha mais espaço no \textit{canvas} para manipulação visual, o que é fundamental em fluxos mais complexos, onde a clareza e o espaço visual são prioridades.

A lógica desse recurso foi implementada para alternar a visibilidade da \textit{sidebar} sem que os dados exibidos sejam perdidos. O usuário pode reexibi-la a qualquer momento, com as informações ainda intactas, proporcionando uma interface flexível e adaptável às necessidades específicas de visualização e operação. O código responsável por essa funcionalidade pode ser encontrado no \textbf{Anexo \ref{Anexo:HideSidebarLogic}}.

A Figura \ref{Fig:HideSidebarButton} ilustra o botão "Esconder Sidebar de Informações" na interface gráfica do sistema.

\begin{figure}[htbp]
    \centering
    \includegraphics[width=0.4\textwidth]{figuras/hide-sidebar-button.png}
    \caption{Botão de esconder a sidebar de informações (Fonte: próprio autor, 2024).}
    \label{Fig:HideSidebarButton}
\end{figure}

\subsubsection{Upload de arquivos em CSV}

O botão \textbf{"Upload de Arquivos CSV"} permite ao usuário importar dados de medições diretamente para o sistema. Esse recurso é fundamental para integrar dados externos ao fluxo de trabalho, possibilitando que arquivos CSV com informações dos sensores e variáveis do processo industrial sejam carregados no RADARE. Após o carregamento, os dados são processados e utilizados nas etapas de reconciliação, facilitando a análise e a correção dos dados.

A funcionalidade foi implementada para garantir o envio e o armazenamento seguro dos arquivos, além de realizar a validação do formato e conteúdo do CSV. Isso assegura que os dados importados estejam devidamente estruturados para serem integrados ao banco de dados e processados com precisão durante as etapas de reconciliação.

A Figura \ref{Fig:UploadCSVButton} mostra o botão "Upload de Arquivos CSV" na interface do sistema.

\begin{figure}[htbp]
    \centering
    \includegraphics[width=0.4\textwidth]{figuras/upload-csv-button.png}
    \caption{Botão de upload de arquivos CSV (Fonte: próprio autor, 2024).}
    \label{Fig:UploadCSVButton}
\end{figure}

% -------------------------
\subsection{\textit{Canvas} do sistema}

O \textit{canvas} é a área principal da interface do RADARE, onde o usuário pode visualizar, conectar e manipular os nódulos para configurar o fluxo de dados industrial. É nesse espaço que o usuário constrói e ajusta os fluxos de trabalho, interligando entradas, saídas e pontos de processamento. A interação no \textit{canvas} é dinâmica e permite a personalização dos fluxos conforme as necessidades do processo.

A Figura \ref{Fig:EmptyCanvas} mostra um exemplo do \textit{canvas} com vários nódulos conectados, oferecendo uma visão geral de como os elementos podem ser arranjados e manipulados visualmente. Nos próximos tópicos, exploraremos em detalhes as funcionalidades disponíveis no \textit{canvas}, com exemplos de código e explicações sobre a interação com os nódulos.

\begin{figure}[htbp]
    \centering
    \includegraphics[width=0.8\textwidth]{figuras/empty-canvas.png}
    \caption{Exemplo da área de trabalho no canvas do RADARE (Fonte: próprio autor, 2024).}
    \label{Fig:EmptyCanvas}
\end{figure}

% -------------------------
\subsubsection{Lógica de conexão entre os nódulos no \textit{canvas}}

O sistema permite que o usuário estabeleça conexões visuais entre os nódulos, representando o fluxo de dados entre diferentes pontos de um processo industrial. Essas conexões são fundamentais para assegurar que os dados fluam corretamente entre os elementos do \textit{canvas}, como entradas, saídas e pontos de processamento.

A Figura \ref{Fig:NodeConnections} ilustra a conexão de dois nódulos no \textit{canvas}, demonstrando como o usuário pode arrastar e soltar as conexões de forma intuitiva. O usuário também pode ajustar e mover essas conexões entre os nódulos, proporcionando flexibilidade na organização dos fluxos de dados e permitindo a personalização do layout conforme as necessidades do processo.

O trecho principal do código responsável pela criação dessa funcionalidade está disponível no \textbf{Anexo \ref{Anexo:frontCodeNodeTwoOne}}. Cada conexão entre os nódulos possui um valor e uma tolerância associados, que podem ser modificados diretamente com um duplo clique na linha de conexão, permitindo ao usuário ajustar os parâmetros conforme necessário. Além disso, as conexões recebem nomes gerados automaticamente para facilitar a distinção entre as diferentes tags.

\begin{figure}[htbp]
    \centering
    \includegraphics[width=0.8\textwidth]{figuras/node-connection-example.png}
    \caption{Exemplo de conexão entre nódulos no \textit{canvas} (Fonte: próprio autor, 2024).}
    \label{Fig:NodeConnections}
\end{figure}

% -------------------------
\subsection{Interface de gráfico de reconciliação de dados}

O \textit{canvas} é a área principal da interface do RADARE, onde o usuário pode visualizar, conectar e manipular os nódulos para configurar o fluxo de dados industrial. É nesse espaço que o usuário constrói e ajusta os fluxos de trabalho, interligando entradas, saídas e pontos de processamento. A interação no \textit{canvas} é dinâmica e permite a personalização dos fluxos conforme as necessidades do processo.

A Figura \ref{Fig:CanvasArea} mostra um exemplo do \textit{canvas} com vários nódulos conectados, oferecendo uma visão geral de como os elementos podem ser arranjados e manipulados visualmente. Nos próximos tópicos, exploraremos em detalhes as funcionalidades disponíveis no \textit{canvas}, com exemplos de código e explicações sobre a interação com os nódulos.

\begin{figure}[htbp]
    \centering
    \includegraphics[width=0.8\textwidth]{figuras/empty-canvas.png}
    \caption{Exemplo da área de trabalho no canvas do RADARE (Fonte: próprio autor, 2024).}
    \label{Fig:CanvasArea}
\end{figure}


% -------------------------
\subsection{Interface de \textit{sidebar} de informações do sistema}

O \textit{canvas} é a área principal da interface do RADARE, onde o usuário pode visualizar, conectar e manipular os nódulos para configurar o fluxo de dados industrial. É nesse espaço que o usuário constrói e ajusta os fluxos de trabalho, interligando entradas, saídas e pontos de processamento. A interação no \textit{canvas} é dinâmica e permite a personalização dos fluxos conforme as necessidades do processo.

A Figura \ref{Fig:CanvasArea} mostra um exemplo do \textit{canvas} com vários nódulos conectados, oferecendo uma visão geral de como os elementos podem ser arranjados e manipulados visualmente. Nos próximos tópicos, exploraremos em detalhes as funcionalidades disponíveis no \textit{canvas}, com exemplos de código e explicações sobre a interação com os nódulos.

\begin{figure}[htbp]
    \centering
    \includegraphics[width=0.8\textwidth]{figuras/empty-canvas.png}
    \caption{Exemplo da área de trabalho no canvas do RADARE (Fonte: próprio autor, 2024).}
    \label{Fig:CanvasArea}
\end{figure}

% -------------------------
\subsection{Resultados finais do front-end em sua visualização completa}

O \textit{canvas} é a área principal da interface do RADARE, onde o usuário pode visualizar, conectar e manipular os nódulos para configurar o fluxo de dados industrial. É nesse espaço que o usuário constrói e ajusta os fluxos de trabalho, interligando entradas, saídas e pontos de processamento. A interação no \textit{canvas} é dinâmica e permite a personalização dos fluxos conforme as necessidades do processo.

A Figura \ref{Fig:CanvasArea} apresenta um exemplo da área de trabalho do \textit{canvas} com vários nódulos conectados, permitindo uma visualização clara de como os elementos podem ser arranjados e manipulados visualmente. Cada nódulo representa uma função específica no processo, e as conexões entre eles ilustram o fluxo de dados que é tratado e reconciliado pelo sistema.

A seguir, exploraremos em detalhes as funcionalidades disponíveis no \textit{canvas}, incluindo a adição de nódulos, a conexão entre eles, a modificação de parâmetros, e a execução de tarefas de reconciliação de dados, sempre com exemplos práticos de código e interação no ambiente.

\begin{figure}[htbp]
    \centering
    \includegraphics[width=0.8\textwidth]{figuras/empty-canvas.png}
    \caption{Exemplo da área de trabalho no canvas do RADARE (Fonte: próprio autor, 2024).}
    \label{Fig:CanvasArea}
\end{figure}

% -------------------------
\section{Resultados do desenvolvimento do \textit{back-end}}

O \textit{back-end} do RADARE foi desenvolvido em \textit{Python} utilizando o \textit{framework} Flask. Essa camada gerencia as requisições da interface, processa os dados submetidos pelo usuário e realiza os cálculos de reconciliação utilizando o método dos multiplicadores de Lagrange. A estrutura foi configurada para responder de maneira eficiente às operações do \textit{front-end}, enviando os resultados dos cálculos e garantindo uma comunicação ágil entre as partes do sistema.

Além de realizar a reconciliação de dados, o \textit{back-end} também gerencia a autenticação e o armazenamento seguro dos dados processados. Com essa estrutura, foi possível assegurar a integridade dos dados e a consistência nos processos realizados, resultando em um sistema robusto e preparado para operações contínuas e de alta demanda.

% -------------------------
\subsection{Desenvolvimento das interfaces RESTful de comunicação entre os sistemas}

As interfaces RESTful foram projetadas para garantir uma comunicação eficiente e segura entre o \textit{front-end} e o \textit{back-end} do sistema RADARE. Cada rota foi cuidadosamente estruturada para facilitar o envio e recebimento de dados, bem como a execução dos processos de reconciliação. Abaixo, são descritas as rotas implementadas, com exemplos de código e explicações detalhadas sobre cada funcionalidade, destacando a importância de cada \textit{endpoint} na integração dos componentes do sistema.

% -------------------------
\subsubsection{Interface RESTful POST /reconcile}

A rota \texttt{POST /reconcile} é responsável por receber os dados dos sensores enviados pelo \textit{front-end} e realizar a reconciliação utilizando o método dos multiplicadores de Lagrange. Após a entrada dos dados, o \textit{back-end} processa as informações e aplica o método para ajustar os valores conforme as restrições impostas, garantindo a consistência dos dados reconciliados. Os resultados obtidos são então armazenados no banco de dados, permitindo a sua recuperação e visualização pelo sistema conforme necessário.

O código completo que implementa essa funcionalidade pode ser consultado no \textbf{Anexo \ref{Anexo:CodigoRouteReconcile}}.

% -------------------------
\subsubsection{Interface RESTful GET /results}

A rota \texttt{GET /results} foi implementada para possibilitar que o usuário recupere os resultados das reconciliações de dados realizadas anteriormente. Ao ser acionada, essa rota consulta o banco de dados e retorna os valores reconciliados, que são então exibidos na interface gráfica, permitindo que o usuário visualize e analise os dados ajustados de forma clara e organizada.

O processo de recuperação desses dados envolve uma consulta ao banco de dados, onde as informações resultantes das reconciliações são armazenadas após o processamento inicial. Esses resultados incluem os valores corrigidos e ajustados por meio do método dos multiplicadores de Lagrange, aplicados durante a fase de reconciliação.

Para mais detalhes sobre o código completo desta rota, consulte o \textbf{Anexo \ref{Anexo:CodigoRouteResults}}, onde é apresentada a implementação completa, incluindo as especificidades da consulta ao banco de dados e a formatação dos dados antes de serem enviados para o \textit{front-end}.

% -------------------------
\subsubsection{Interface RESTful POST /upload}

A rota \texttt{POST /upload} é essencial para o funcionamento do RADARE, pois permite o envio de arquivos CSV contendo dados externos, como leituras de sensores industriais, que serão posteriormente integrados ao banco de dados do sistema. Esses dados servem de base para os processos de reconciliação, possibilitando uma análise precisa e consistente das informações coletadas.

Quando um arquivo é enviado através desta rota, ele é processado no \textit{back-end} para extrair as informações contidas no CSV. Em seguida, os dados são armazenados no banco de dados, ficando disponíveis para as etapas de reconciliação e visualização no \textit{front-end}. Esse fluxo garante que o sistema possa ser atualizado com dados de múltiplas fontes, permitindo uma integração contínua e ampliando a capacidade de análise do RADARE.

O código completo da implementação da rota \texttt{POST /upload} está disponível no \textbf{Anexo \ref{Anexo:CodigoRouteUpload}}.

% -------------------------
\subsection{Serviços desenvolvidos para o sistema}

Os serviços, comumente chamados de \textit{services} no \textit{back-end} são responsáveis por implementar a lógica de negócio, processar os dados e invocar os cálculos de reconciliação. Eles abstraem a complexidade do sistema, garantindo que os dados sejam processados corretamente antes de serem enviados ao banco de dados ou utilizados na interface. Abaixo estão descritos os principais \textit{services} do sistema RADARE, juntamente com exemplos de código e diagramas.

% -------------------------
\subsubsection{Serviço de validação de dados}

O serviço de validação de dados assegura que as informações recebidas estejam no formato apropriado antes de seguirem para o processamento no sistema. Ele verifica a presença de todos os campos obrigatórios, assegura a consistência dos tipos de dados e identifica quaisquer valores ausentes ou inválidos que possam comprometer a integridade do processo de reconciliação. Essa camada de validação é essencial para prevenir erros e garantir que apenas dados confiáveis sejam considerados no cálculo dos resultados.

O trecho principal do código que implementa o serviço de validação de dados pode ser consultado integralmente no \textbf{Anexo \ref{Anexo:CodigoValidacaoDados}}.

% -------------------------
\subsubsection{Serviço de processamento de dados}

O serviço de processamento de dados é responsável por manipular arquivos CSV enviados pelos usuários, convertendo as informações contidas nesses arquivos em estruturas apropriadas para armazenamento no banco de dados e para execução dos cálculos de reconciliação. Este serviço é essencial para garantir que os dados externos sejam integrados ao sistema RADARE de maneira eficaz e que estejam no formato correto para serem processados e analisados.

A validação e transformação dos dados incluem a verificação da consistência dos valores, adequação das colunas conforme os requisitos do banco de dados e conversão dos tipos de dados necessários para as operações de cálculo. Qualquer dado inconsistente ou fora do padrão esperado é tratado antes de ser persistido, assegurando que apenas dados válidos sejam integrados ao sistema.

O trecho principal do código responsável pela criação desse serviço pode ser encontrado em sua totalidade no \textbf{Anexo \ref{Anexo:CodigoProcessamentoDados}}.

% -------------------------
\subsubsection{Serviço de Reconciliação de Dados}

O serviço de reconciliação de dados é uma das funcionalidades centrais do sistema RADARE, sendo responsável por executar o algoritmo de reconciliação utilizando o método dos multiplicadores de Lagrange. Esse método matemático garante que as restrições de balanço de massa e energia sejam rigorosamente respeitadas, assegurando a consistência dos dados processados. Durante a reconciliação, o serviço analisa e ajusta os dados coletados de sensores e outros pontos de entrada, alinhando-os às condições impostas e reduzindo discrepâncias.

Para cada execução, o serviço recebe dados brutos e aplica o algoritmo de reconciliação, retornando valores ajustados que atendem às restrições estabelecidas. Esse processamento permite ao sistema RADARE gerar dados confiáveis e coerentes, essenciais para a análise de processos industriais. 

O código principal deste serviço, detalhado no \textbf{Anexo \ref{Anexo:CodigoReconciliacaoDados}}, exemplifica a implementação do método de reconciliação e a integração com o banco de dados para o armazenamento dos resultados.

% -------------------------
\subsection{Modelos desenvolvidos para o sistema}

Os \textit{models} foram implementados utilizando uma ORM (Object-Relational Mapping) em \textit{Python} para facilitar a comunicação com o banco de dados PostgreSQL. Esse método permite mapear objetos do código para tabelas no banco de dados, tornando o desenvolvimento mais ágil e a manutenção mais eficiente. Os principais modelos implementados no sistema incluem:

\begin{itemize}
    \item \textbf{Process Model}: Representa os processos industriais, contendo as informações de variáveis medidas e parâmetros relevantes.
    \item \textbf{Measurement Model}: Armazena as medições recebidas dos sensores, assegurando que cada registro seja associado ao respectivo processo.
    \item \textbf{Result Model}: Registra os resultados das reconciliações, vinculando-os aos processos e medições correspondentes para consulta futura.
\end{itemize}

O trecho principal do código responsável pela criação desses modelos pode ser encontrado em sua totalidade no \textbf{Anexo \ref{Anexo:CodigoModelo}}.

% -------------------------
\section{Resultados do desenvolvimento do banco de dados}

O banco de dados utilizado no sistema RADARE foi implementado em PostgreSQL e armazena todas as informações relevantes para a execução do processo de reconciliação de dados industriais, gerenciamento de usuários e rastreamento de atividades. A modelagem foi feita de forma a garantir a integridade e eficiência na consulta e manipulação dos dados. A seguir, são descritas as principais tabelas implementadas no sistema.

% -------------------------
\subsection{Tabela de Dados de Processos}

A tabela de dados de processos industriais (\autoref{tab:processDataTable}) armazena informações essenciais para o funcionamento do sistema RADARE, incluindo medições de sensores e os resultados das reconciliações de dados. Esta estrutura permite uma organização clara e precisa dos dados, garantindo que cada registro seja identificado de forma única e possa ser associado a operações de reconciliação específicas.

\begin{table}[htbp]
    \centering
    \caption{Descrição das colunas da tabela de dados de processos industriais.}
    \label{tab:processDataTable}
    \begin{tabular}{|l|p{10cm}|}
        \hline
        \textbf{Coluna} & \textbf{Descrição} \\ \hline
        \textbf{id} & Identificação única do registro, usada como chave primária. \\ \hline
        \textbf{user} & Identifica o usuário responsável pela reconciliação. \\ \hline
        \textbf{time} & Horário da reconciliação, para referência temporal. \\ \hline
        \textbf{tagname} & Nome da variável medida (sensor ou ponto de coleta). \\ \hline
        \textbf{tagreconciled} & Valor reconciliado da variável após ajustes. \\ \hline
        \textbf{tagcorrection} & Valor da correção aplicada à variável medida. \\ \hline
        \textbf{tagmatrix} & Matriz de correlação usada na reconciliação. \\ \hline
    \end{tabular}
\end{table}


% -------------------------
\subsection{Tabela de Usuários}

A tabela de usuários (\autoref{Tab:Users}) é essencial para gerenciar a autenticação, as permissões e o rastreamento de atividades dos usuários no sistema RADARE. Essa estrutura não apenas assegura que apenas usuários autorizados possam acessar determinadas funcionalidades, mas também facilita a auditoria e a segurança do sistema, mantendo informações sensíveis e essenciais para o controle de acesso.

\begin{table}[htbp]
    \centering
    \caption{Estrutura da tabela de usuários no sistema RADARE.}
    \label{Tab:Users}
    \begin{tabular}{|l|p{10cm}|}
        \hline
        \textbf{Coluna} & \textbf{Descrição} \\ \hline
        \textbf{id} & UUID único para cada usuário (chave primária). \\ \hline
        \textbf{username} & Nome de usuário para login (deve ser único). \\ \hline
        \textbf{email} & Endereço de email para notificações. \\ \hline
        \textbf{password\_hash} & Hash da senha para segurança. \\ \hline
        \textbf{role} & Função do usuário (e.g., admin, operador, analista). \\ \hline
        \textbf{created\_at} & Data e hora da criação da conta. \\ \hline
        \textbf{last\_login} & Data e hora do último acesso. \\ \hline
    \end{tabular}
\end{table}

% -------------------------
\subsection{Tabela de Logs de Atividade}

A tabela de logs de atividade (Tabela \ref{Tab:ActivityLogs}) armazena registros detalhados das ações realizadas pelos usuários no sistema, permitindo auditoria e rastreamento de atividades como uploads de arquivos ou execuções de reconciliação de dados. Essa estrutura auxilia no monitoramento e na segurança, facilitando a análise das interações dos usuários com o sistema.

\begin{table}[htbp]
    \centering
    \caption{Estrutura da tabela de logs de atividade no sistema RADARE.}
    \label{Tab:ActivityLogs}
    \begin{tabular}{|l|p{10cm}|}
        \hline
        \textbf{Coluna} & \textbf{Descrição} \\ \hline
        \textbf{id} & Identificação única para cada log de atividade. \\ \hline
        \textbf{user\_id} & Referência ao ID do usuário que realizou a ação. \\ \hline
        \textbf{action} & Descrição da ação realizada. \\ \hline
        \textbf{timestamp} & Data e hora da realização da ação. \\ \hline
        \textbf{details} & Informações adicionais sobre a ação. \\ \hline
    \end{tabular}
\end{table}

% -------------------------
\subsection{Tabela de Configurações de Processos}

A tabela de configurações de processos industriais (Tabela \ref{Tab:ProcessConfigurations}) armazena parâmetros essenciais para a personalização do sistema, incluindo limites de variáveis e especificações de medições. Essa estrutura permite adaptar o comportamento do sistema a diferentes cenários industriais, proporcionando flexibilidade na definição de cada processo.

\begin{table}[htbp]
    \centering
    \caption{Estrutura da tabela de configurações de processos no sistema RADARE.}
    \label{Tab:ProcessConfigurations}
    \begin{tabular}{|l|p{10cm}|}
        \hline
        \textbf{Coluna} & \textbf{Descrição} \\ \hline
        \textbf{id} & Identificação única para cada configuração de processo. \\ \hline
        \textbf{process\_name} & Nome do processo industrial específico. \\ \hline
        \textbf{sensor\_limits} & Limites definidos para as variáveis dos sensores. \\ \hline
        \textbf{created\_by} & ID do usuário responsável pela criação da configuração. \\ \hline
        \textbf{created\_at} & Data e hora em que a configuração foi registrada no sistema. \\ \hline
    \end{tabular}
\end{table}

% -------------------------
\section{Manuais de sistema}

Como parte do desenvolvimento do \textit{software} RADARE, foram criados dois manuais distintos para auxiliar tanto os administradores técnicos quanto os usuários finais na utilização e manutenção do sistema. Esses manuais visam garantir a correta operação e longevidade da aplicação, fornecendo instruções claras sobre o funcionamento do sistema e as melhores práticas para sua utilização.

% -------------------------
\subsection{Manual de manutenção do sistema}

O manual de manutenção do sistema fornece orientações para desenvolvedores e administradores sobre as funções internas do RADARE, incluindo detalhes sobre o *front-end*, *back-end* e banco de dados. Ele descreve as principais funções, fluxos de dados e interações entre os componentes, além de instruções para manutenção, como atualização de bibliotecas, ajustes de configurações, monitoramento de logs e execução de testes de desempenho. Também aborda limitações técnicas, como número máximo de conexões simultâneas e requisitos mínimos de hardware. Procedimentos para diagnóstico e resolução de problemas comuns, como erros de reconciliação e falhas no processamento de arquivos, são apresentados.

O manual completo está disponível no \textbf{Anexo \ref{Cap:manualManutencao}}, facilitando a consulta detalhada sobre a manutenção do sistema.

% -------------------------
\subsection{Manual de uso para o usuário final}

O manual de uso foi elaborado para orientar o usuário nas tarefas de reconciliação de dados industriais com o *RADARE*. Ele apresenta uma introdução ao sistema, explicando seu propósito e principais funcionalidades. Em seguida, descreve o passo a passo das operações, incluindo instruções para adicionar nódulos no \textit{canvas}, realizar reconciliações e carregar arquivos CSV.

O manual inclui guias visuais e dicas de usabilidade, auxiliando na navegação e no uso eficiente da interface, além de oferecer soluções para problemas comuns, como erros de upload e falhas de conexão entre nódulos. 

O manual completo está disponível para consulta no \textbf{Anexo \ref{Anexo:manualUsuario}}, com todos os detalhes necessários para o uso do sistema.
       % Capítulo 4: Resultados
% \mychapter{Conclusão}
\label{Cap:Conclusao}

Este trabalho teve como objetivo desenvolver e implementar o \textit{webapp} RADARE, combinando técnicas modernas de desenvolvimento \textit{web} com o método dos multiplicadores de Lagrange. A solução visa aprimorar a qualidade e confiabilidade dos dados, oferecendo ao usuário uma experiência mais atualizada e eficiente, com impacto direto na otimização da tomada de decisões no ambiente industrial.

Ao longo do projeto, foram pesquisadas e analisadas as principais metodologias de desenvolvimento \textit{web}, além de tecnologias adequadas para a construção de um sistema acessível e eficiente. A escolha do ambiente \textit{web} provou ser acertada, proporcionando vantagens como interoperabilidade com outras plataformas, acesso remoto e uma interface amigável, que facilita o uso e torna o sistema acessível a usuários de diferentes níveis técnicos.

Entre as principais contribuições deste trabalho, destaca-se a aplicação eficaz do método dos multiplicadores de Lagrange em um sistema \textit{web}, algo ainda pouco explorado em ferramentas disponíveis no mercado nacional. Ao permitir a integração com diferentes sistemas de monitoramento, o RADARE tem o potencial de proporcionar um ganho significativo na eficiência dos processos industriais, tanto em termos de qualidade dos dados quanto na redução de falhas e custos operacionais.

As possibilidades para trabalhos futuros são amplas. A implementação de novos algoritmos, como a filtragem de Kalman, poderia melhorar a otimização de processos dinâmicos, tornando a ferramenta ainda mais eficaz em cenários de alta variabilidade. Além disso, a criação de módulos de análise preditiva seria um passo natural para o sistema, permitindo a antecipação de falhas e aumentando a confiabilidade das operações.

Em resumo, o RADARE atende a uma necessidade clara no mercado brasileiro ao oferecer uma solução inovadora e adaptável para a reconciliação de dados industriais. Apesar de algumas limitações observadas, o sistema apresenta um grande potencial de expansão e aplicação em novos setores, reforçando sua importância para o avanço tecnológico e industrial no Brasil.        % Capítulo 5: Conclusão

%%%%%%%%%%%%%%%%%%%%%%%%%%%%%%%%%%%%%%%%%%%%%%%%%%%%%%%%%%%%%%%%%%%%%%%%%%%%%%%
%% REFERÊNCIAS BIBLIOGRÁFICAS
%%%%%%%%%%%%%%%%%%%%%%%%%%%%%%%%%%%%%%%%%%%%%%%%%%%%%%%%%%%%%%%%%%%%%%%%%%%%%%%

% \phantomsection
% \addcontentsline{toc}{chapter}{Referências bibliográficas}
% \bibliography{bibliografia/bibliografia}

%%%%%%%%%%%%%%%%%%%%%%%%%%%%%%%%%%%%%%%%%%%%%%%%%%%%%%%%%%%%%%%%%%%%%%%%%%%%%%%
%% APÊNDICES E ANEXOS
%%%%%%%%%%%%%%%%%%%%%%%%%%%%%%%%%%%%%%%%%%%%%%%%%%%%%%%%%%%%%%%%%%%%%%%%%%%%%%%

% % Apêndices (material elaborado pelo autor)
% \begin{appendices}
% \appendix
% \include{textuais/apendice/Apendices}
% \end{appendices}

% % Anexos (material não elaborado pelo autor)
% \renewcommand{\appendixname}{Anexo}
% \begin{appendices}
% \appendix
% \mychapter{Termo de Autorização do Autor}
\label{Cap:anexo}

A principal diferença entre anexo e apêndice é que os apêndices são textos criados pelo próprio autor para complementar sua argumentação, enquanto os anexos são documentos criados por terceiros e usados pelo autor.

Tanto o apêndice quanto o anexo devem estar presentes no sumário dos trabalhos científicos. Os apêndices devem aparecer depois das referências e os anexos depois dos apêndices.

\includepdf[pages=-]{textuais/anexo/termo_de_autorizacao.pdf}

\mychapter{Configuração da Máquina de Desenvolvimento}
\label{Ap:configuracaoMaquina}

Este apêndice apresenta as especificações de hardware e software da máquina utilizada para o desenvolvimento do sistema RADARE, garantindo que o ambiente fosse adequado para o trabalho de desenvolvimento e testes do \textit{software}.

\section{Especificações de Hardware}
\begin{itemize}
    \item \textbf{Processador}: Intel Core i7-10750H (2.6 GHz até 5.0 GHz, 6 núcleos)
    \item \textbf{Memória RAM}: 16 GB DDR4
    \item \textbf{Armazenamento}: 512 GB SSD NVMe
    \item \textbf{Placa Gráfica}: NVIDIA GeForce GTX 1650 Ti (4 GB GDDR6)
    \item \textbf{Monitor}: Resolução Full HD (1920x1080) em uma tela de 15,6 polegadas
\end{itemize}

\section{Especificações de Software}
\begin{itemize}
    \item \textbf{Sistema Operacional}: Windows 11 Pro (64-bit)
    \item \textbf{Editor de Código}: Visual Studio Code (com extensões para TypeScript, Python e controle de versão Git)
    \item \textbf{Versionamento de Código}: Git e GitHub para controle de versão e colaboração
    \item \textbf{Ambiente de Execução de Python}: Python 3.9 com bibliotecas específicas para o desenvolvimento \textit{back-end}
    \item \textbf{Navegadores para Testes}: Microsoft Edge, Google Chrome e Mozilla Firefox (versões mais recentes)
\end{itemize}

\section{Extensões Utilizadas no Visual Studio Code}
\begin{itemize}
    \item \textbf{TypeScript}: Extensão para suporte e autocompletar do TypeScript
    \item \textbf{Python}: Suporte ao desenvolvimento com Python, incluindo linting e debugging
    \item \textbf{Prettier - Code Formatter}: Ferramenta de formatação automática para o código
    \item \textbf{ESLint}: Análise de código estática para manter a qualidade do código
    \item \textbf{GitLens}: Ferramenta para integração avançada com Git
\end{itemize}

Essas configurações e ferramentas foram essenciais para assegurar um fluxo de trabalho eficiente, com suporte adequado para o desenvolvimento e teste do \textit{software} RADARE.


\mychapter{Código Completo da Rota POST /reconcile}
\label{Anexo:CodigoRouteReconcile}

\begin{minted}[fontsize=\small, linenos, frame=single, caption={Implementação da Rota POST /reconcile}]{python}
from flask import Flask, request, jsonify
from your_database_module import store_reconciled_data
from your_reconciliation_module import perform_lagrange_reconciliation

app = Flask(__name__)

@app.route('/reconcile', methods=['POST'])
def reconcile_data():
    try:
        # Recebe os dados do front-end em formato JSON
        data = request.get_json()

        # Processa os dados utilizando o método dos multiplicadores de Lagrange
        reconciled_data = perform_lagrange_reconciliation(data)

        # Armazena os dados reconciliados no banco de dados
        store_reconciled_data(reconciled_data)

        # Retorna uma resposta de sucesso ao front-end
        return jsonify({"status": "success", "reconciled_data": reconciled_data}), 200

    except Exception as e:
        # Em caso de erro, retorna uma mensagem de erro
        return jsonify({"status": "error", "message": str(e)}), 500

if __name__ == '__main__':
    app.run(debug=True)
\end{minted}

Este código demonstra como os dados são recebidos e processados pela rota \texttt{POST /reconcile}, aplicando o método de reconciliação e armazenando os resultados no banco de dados para uso posterior.

\mychapter{Código Completo da Rota GET /results}
\label{Anexo:CodigoRouteResults}

\begin{minted}[frame=lines, fontsize=\small, linenos]{python}
from flask import Flask, jsonify
from your_database_module import retrieve_reconciled_data

app = Flask(__name__)

@app.route('/results', methods=['GET'])
def get_results():
    try:
        # Recupera os dados reconciliados do banco de dados
        reconciled_data = retrieve_reconciled_data()

        # Retorna os dados reconciliados para o front-end
        return jsonify({"status": "success", "reconciled_data": reconciled_data}), 200

    except Exception as e:
        # Em caso de erro, retorna uma mensagem de erro
        return jsonify({"status": "error", "message": str(e)}), 500

if __name__ == '__main__':
    app.run(debug=True)
\end{minted}

Este código ilustra como a rota \texttt{GET /results} foi configurada para recuperar os dados reconciliados do banco de dados e enviar essas informações para o \textit{front-end}. O processo inclui a captura de dados previamente ajustados pelo método de reconciliação, permitindo que sejam exibidos na interface de forma organizada para análise.

\mychapter{Código Completo da Rota POST /upload}
\label{Anexo:CodigoRouteUpload}

\begin{minted}[frame=lines, fontsize=\small, linenos]{python}
from flask import Flask, request, jsonify
import csv
from your_database_module import store_csv_data

app = Flask(__name__)

@app.route('/upload', methods=['POST'])
def upload_file():
    try:
        # Verifica se o arquivo foi enviado na requisição
        if 'file' not in request.files:
            return jsonify({"status": "error", "message": "No file provided"}), 400

        file = request.files['file']
        
        # Lê o conteúdo do arquivo CSV
        csv_data = []
        with open(file.stream, 'r') as csvfile:
            csv_reader = csv.reader(csvfile)
            for row in csv_reader:
                csv_data.append(row)

        # Armazena os dados CSV no banco de dados
        store_csv_data(csv_data)

        # Retorna uma resposta de sucesso ao front-end
        return jsonify({"status": "success", "message": "File uploaded successfully"}), 200

    except Exception as e:
        # Em caso de erro, retorna uma mensagem de erro
        return jsonify({"status": "error", "message": str(e)}), 500

if __name__ == '__main__':
    app.run(debug=True)
\end{minted}

Este código descreve a implementação da rota \texttt{POST /upload}, responsável pelo processamento de arquivos CSV enviados pelo \textit{front-end}. Os dados do arquivo são lidos, processados e armazenados no banco de dados, permitindo que sejam utilizados nas etapas subsequentes de reconciliação e análise.

% \end{appendices}

\end{document}
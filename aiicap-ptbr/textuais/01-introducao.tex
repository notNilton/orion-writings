\mychapterast{Introdução}
\label{Cap:Introducao}

A geração de imagens do tipo pixel art por modelos de linguagem tem se mostrado uma ferramenta valiosa para diversas aplicações criativas. No entanto, esses modelos frequentemente produzem artefatos visuais indesejados, como bordas irregulares e variações cromáticas inconsistentes, que comprometem a qualidade estética e a utilidade prática dos resultados. Este trabalho apresenta o AIICAP (Artificial Intelligence Image Correction and Augmentation Pipeline), um método automatizado para correção desses artefatos em imagens de pixel art geradas por IA.

\mysectionast{Contexto e Motivação}

O crescente uso de modelos generativos para criação de conteúdo visual tem destacado limitações na produção de imagens com estética pixelada precisa. Embora capazes de gerar composições criativas, esses modelos frequentemente falham em manter a regularidade estrutural característica da pixel art tradicional. Essa deficiência se manifesta através de diversos artefatos visuais que reduzem a aplicabilidade prática das imagens geradas, especialmente em contextos que demandam precisão gráfica, como desenvolvimento de jogos e produção de assets digitais. O AIICAP surge como resposta a essa limitação, oferecendo uma solução sistemática para o problema.
\mychapter{Problema}
\label{Cap:Problema}

Neste capítulo, formalizamos o problema que este trabalho visa resolver, focando em sua formulação matemática e na solução proposta por meio do método de Lagrange. O problema de reconciliação de dados em processos industriais envolve a minimização das discrepâncias entre medições obtidas em campo e os valores estimados por modelos matemáticos, respeitando as restrições impostas por leis de conservação e equações de processo. Este problema é crucial para garantir a confiabilidade e a integridade das medições em plantas industriais, onde erros aleatórios e sistemáticos podem comprometer a otimização e o controle dos processos.

Com o avanço da Indústria 4.0, a necessidade de soluções para processar grandes volumes de dados gerados por sensores em tempo real se intensifica. A integração de tecnologias como IoT (Internet das Coisas), sistemas ciberfísicos e automação inteligente torna a reconciliação de dados um ponto central para garantir a qualidade das informações utilizadas em ambientes industriais \cite{industry40}. A Indústria 4.0 exige que os dados gerados pelos sensores sejam integrados a modelos matemáticos que otimizem operações e minimizem incertezas, criando um cenário de monitoramento e controle mais eficiente \cite{datareconciliationindustry4}. Nesse contexto, o desenvolvimento de soluções web para reconciliação de dados oferece maior flexibilidade e acessibilidade, permitindo que indústrias utilizem essas ferramentas de qualquer local \cite{websolutions}.

\section{Formulação do Problema}
\label{Sec:FormulacaoProblema}

O problema central a ser resolvido é a reconciliação de dados industriais, formulado como um problema de otimização de funções multivariáveis. Dado um conjunto de variáveis medidas $y = [y_1, y_2, \dots, y_n]$, que representam valores obtidos por sensores em uma planta industrial, o objetivo é encontrar o vetor $x = [x_1, x_2, \dots, x_n]$, que representa os valores reconciliados dessas medições, minimizando a diferença entre os valores medidos e os valores reconciliados. Essa minimização deve ser realizada respeitando as restrições impostas pelas leis de conservação de massa e energia.

Matematicamente, o problema de reconciliação de dados pode ser expresso como a minimização de uma função objetivo $J(x)$, que representa o erro quadrático entre as medições e os valores reconciliados:

\begin{equation}
\min_x J(x) = \sum_{i=1}^{n} (y_i - x_i)^2
\end{equation}

Sujeito às restrições:

\begin{equation}
Ax = b
\end{equation}

Aqui, $A$ é a matriz que define as equações de balanço de massa e energia, e $b$ representa os valores conhecidos dessas restrições. Esse modelo de minimização com restrições é uma abordagem comum em reconciliação de dados industriais \cite{lagrangeopt}, \cite{processbalancing}, sendo amplamente utilizado em cenários da Indústria 4.0, que exige integração entre dados e sistemas ciberfísicos \cite{cyberphysicalsystems}.

\section{Método de Lagrange para Reconciliação de Dados}
\label{Sec:MetodoLagrange}

O método de Lagrange é a técnica escolhida para resolver o problema de reconciliação de dados, dado que ele lida eficientemente com otimizações que envolvem restrições lineares. Este método é amplamente utilizado para minimizar funções sujeitas a restrições, sendo ideal para a reconciliação de dados em processos industriais, onde as leis de conservação de massa e energia devem ser respeitadas.

A função de Lagrange $\mathcal{L}(x, \lambda)$ combina a função objetivo e as restrições, introduzindo um conjunto de multiplicadores $\lambda = [\lambda_1, \lambda_2, \dots, \lambda_m]$, que representam as forças das restrições:

\[
\mathcal{L}(x, \lambda) = J(x) + \lambda^T (Ax - b)
\]

Onde $J(x)$ é a função objetivo que minimiza o erro de medição, e $Ax = b$ representa as restrições físicas do problema.

\section{Solução Matemática}
\label{Sec:SolucaoMatematica}

A solução do problema de reconciliação de dados pelo método de Lagrange envolve resolver o sistema de equações gerado pelas condições de otimalidade de Karush-Kuhn-Tucker (KKT). Essas condições são derivadas ao tomar as derivadas parciais da função de Lagrange em relação às variáveis $x$ e aos multiplicadores $\lambda$, e igualando-as a zero:

\[
\frac{\partial \mathcal{L}}{\partial x_i} = 2(x_i - y_i) + \lambda^T A_i = 0, \quad \forall i
\]
\[
\frac{\partial \mathcal{L}}{\partial \lambda} = Ax - b = 0
\]

Esse sistema de equações define a solução ótima para as variáveis reconciliadas $x$ e para os multiplicadores de Lagrange $\lambda$. A resolução dessas equações, feita por métodos numéricos, oferece os valores que minimizam o erro total das medições, respeitando as leis físicas do processo \cite{lagrangerecon}.

\section{Discussão sobre a Solução}
\label{Sec:DiscussaoSolucao}

A aplicação do método dos multiplicadores de Lagrange para a reconciliação de dados em processos industriais oferece uma solução robusta e eficaz para lidar com medições imprecisas, especialmente em um contexto onde as leis de conservação devem ser respeitadas. Com o uso de algoritmos baseados nesse método, a qualidade das medições pode ser aprimorada, o que resulta em maior confiabilidade dos dados utilizados para controle e otimização dos processos industriais \cite{reconciliationalgorithms}.

Na Indústria 4.0, a reconciliação de dados é ainda mais relevante, pois permite integrar grandes volumes de dados em tempo real, aumentando a eficiência operacional e a tomada de decisões baseadas em dados precisos \cite{datareconciliationindustry4}. O desenvolvimento web, nesse contexto, se mostra uma solução moderna e prática, permitindo que sistemas de reconciliação de dados sejam acessados de forma remota e em qualquer lugar \cite{websolutions}.

Porém, o método tem limitações, particularmente em lidar com erros grosseiros ou sistemáticos nas medições. O modelo assume que o erro de medição é distribuído normalmente, o que pode não ser válido em todos os cenários. Além disso, medições contendo erros grosseiros podem comprometer a qualidade da reconciliação, o que indica que melhorias adicionais, como a incorporação de técnicas de Detecção de Erros Brutos (GED), podem aumentar a robustez da solução \cite{dataerrorsystems}.


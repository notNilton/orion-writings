\mychapter{Problema}
\label{Cap:Problema}

Neste capítulo, formalizamos o problema com foco no desenvolvimento do software. A solução proposta visa otimizar o processamento de grandes volumes de dados gerados por sensores em plantas industriais, com acesso em tempo real e integração com sistemas existentes. A implementação de uma plataforma web flexível possibilita o uso remoto e automação de processos, garantindo alta escalabilidade e robustez no tratamento dos dados.

O método de reconciliação de dados, fundamentado na minimização de discrepâncias entre medições de campo e estimativas de modelos matemáticos, é integrado ao sistema para corrigir erros e assegurar a consistência dos dados. Esse processo respeita as leis de conservação e equações de processo, essenciais para garantir a precisão e confiabilidade dos dados.

\section{Definição do Problema}
\label{Sec:DefinicaoProblema}

O problema de reconciliação de dados industriais envolve a correção de medições coletadas por sensores, sujeitas a erros aleatórios ou sistemáticos, que podem comprometer a operação eficiente e segura dos processos industriais. Medições incorretas resultam em decisões inadequadas, prejudicando a otimização dos processos e o controle em tempo real \cite{datareconciliationdefinition}. Além disso, o aumento do volume de dados, especialmente em ambientes da Indústria 4.0, requer soluções capazes de processar essas informações de forma contínua e escalável \cite{industry40}.

\section{Abordagem Proposta}
\label{Sec:AbordagemProposta}

A solução desenvolvida consiste em um software web escalável para a reconciliação de dados em processos industriais. A arquitetura modular do sistema facilita a integração com sensores e sistemas de controle, permitindo a ingestão de dados via APIs e o processamento tanto em tempo real quanto em batches. A interface web foi implementada utilizando \textbf{React} e \textbf{TypeScript}, proporcionando uma experiência de usuário eficiente e intuitiva. No backend, a lógica de processamento foi estruturada utilizando \textbf{Node.js}, integrando-se ao banco de dados \textbf{PostgreSQL} para armazenamento e consulta dos dados reconciliados.

\section{Desafios Técnicos}
\label{Sec:DesafiosTecnicos}

Durante o desenvolvimento do software, foram enfrentados vários desafios técnicos. O primeiro desafio foi o tratamento de grandes volumes de dados gerados continuamente em tempo real, exigindo uma solução escalável para manter a integridade e o desempenho do sistema. Também houve a necessidade de garantir a consistência dos dados em operações offline, com uma sincronização eficiente quando a conectividade fosse restabelecida. Além disso, foi necessário integrar o software com sistemas legados e APIs industriais existentes, garantindo compatibilidade e interoperabilidade entre diferentes plataformas. Por fim, a otimização da latência e do desempenho do sistema foi uma preocupação constante, assegurando que o software respondesse adequadamente às demandas dos ambientes industriais, onde os dados precisam ser processados e apresentados em tempo real.

A solução foi projetada para ser altamente escalável, com o uso de contêineres \textbf{Docker}, facilitando o deploy e gerenciamento em ambientes distribuídos.

\section{Limitações da Solução}
\label{Sec:LimitacoesSolucao}

Embora a abordagem proposta resolva grande parte dos problemas de reconciliação de dados, algumas limitações permanecem. O modelo de reconciliação assume que os erros seguem uma distribuição normal, o que pode não refletir a realidade em todos os cenários \cite{datadistributionissue}. Medições com erros grosseiros ou sistemáticos podem comprometer a qualidade da reconciliação dos dados, e técnicas complementares de Detecção de Erros Brutos (GED) podem ser necessárias para melhorar a robustez da solução \cite{gedtechniques}. Além disso, em ambientes com conectividade limitada, o desempenho da plataforma web pode ser impactado, especialmente durante a sincronização de grandes volumes de dados.

\section{Trabalhos Relacionados}
\label{Sec:TrabalhosRelacionados}

Trabalhos anteriores focam em algoritmos de reconciliação de dados, como os baseados nos multiplicadores de Lagrange, amplamente utilizados em processos industriais para minimizar discrepâncias em medições \cite{reconciliationlagrange}. No contexto da Indústria 4.0, a integração de tecnologias de automação e sistemas ciberfísicos tem aumentado a necessidade de soluções escaláveis e flexíveis para o tratamento de dados em tempo real \cite{realtimeprocessingindustry4}. Comparado a soluções tradicionais, a abordagem proposta oferece maior flexibilidade, permitindo acesso remoto e integração com sistemas industriais via APIs, diferenciando-se por sua acessibilidade e capacidade de processamento em grande escala \cite{webbasedsolutions}.

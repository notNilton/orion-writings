\mychapter{Configuração da Máquina de Desenvolvimento}
\label{Ap:configuracaoMaquina}

Este apêndice apresenta as especificações de hardware e software da máquina utilizada para o desenvolvimento do sistema RADARE, garantindo que o ambiente fosse adequado para o trabalho de desenvolvimento e testes do \textit{software}.

\section{Especificações de Hardware}
\begin{itemize}
    \item \textbf{Processador}: Intel Core i7-10750H (2.6 GHz até 5.0 GHz, 6 núcleos)
    \item \textbf{Memória RAM}: 16 GB DDR4
    \item \textbf{Armazenamento}: 512 GB SSD NVMe
    \item \textbf{Placa Gráfica}: NVIDIA GeForce GTX 1650 Ti (4 GB GDDR6)
    \item \textbf{Monitor}: Resolução Full HD (1920x1080) em uma tela de 15,6 polegadas
\end{itemize}

\section{Especificações de Software}
\begin{itemize}
    \item \textbf{Sistema Operacional}: Windows 11 Pro (64-bit)
    \item \textbf{Editor de Código}: Visual Studio Code (com extensões para TypeScript, Python e controle de versão Git)
    \item \textbf{Versionamento de Código}: Git e GitHub para controle de versão e colaboração
    \item \textbf{Ambiente de Execução de Python}: Python 3.9 com bibliotecas específicas para o desenvolvimento \textit{back-end}
    \item \textbf{Navegadores para Testes}: Microsoft Edge, Google Chrome e Mozilla Firefox (versões mais recentes)
\end{itemize}

\section{Extensões Utilizadas no Visual Studio Code}
\begin{itemize}
    \item \textbf{TypeScript}: Extensão para suporte e autocompletar do TypeScript
    \item \textbf{Python}: Suporte ao desenvolvimento com Python, incluindo linting e debugging
    \item \textbf{Prettier - Code Formatter}: Ferramenta de formatação automática para o código
    \item \textbf{ESLint}: Análise de código estática para manter a qualidade do código
    \item \textbf{GitLens}: Ferramenta para integração avançada com Git
\end{itemize}

Essas configurações e ferramentas foram essenciais para assegurar um fluxo de trabalho eficiente, com suporte adequado para o desenvolvimento e teste do \textit{software} RADARE.

\mychapter{Código Completo do Nódulo Adicionar Nódulo 1-1}
\label{Anexo:frontCodeNodeOneOne}

\begin{minted}[linenos, breaklines, fontsize=\small]{typescript}
const addNodeOneOne = ({ data }) => {
  const { label, isConnectable } = data;

  return (
    <>
      <Handle
        type="source"
        id="a"
        position={Position.Left}
        style={{ background: "black" }}
        isConnectable={isConnectable}
      />
      <Handle
        type="source"
        position={Position.Right}
        id="b"
        style={{ background: "black" }}
        isConnectable={isConnectable}
      />
      <div>{`${label}`}</div>
    </>
  );
};

export default memo(addNodeOneOne);
\end{minted}

Este anexo apresenta o código completo da implementação do nódulo "Adicionar Nódulo 1-1", que é responsável pela criação de um ponto de transição no \textit{canvas}. O código configura as conexões (\textit{handles}) e estiliza o nódulo de acordo com o fluxo de dados, conforme descrito na seção correspondente.

\mychapter{Código Completo para o Nódulo 1-2}
\label{Anexo:frontCodeNodeOneTwo}

\begin{minted}[linenos, breaklines, tabsize=2]{typescript}
const CustomNodeOneTwo = ({ data }) => {
  const { label, isConnectable } = data;

  return (
    <>
      <Handle
        type="target"
        position={Position.Top}
        id="a"
        style={{ background: "blue" }}
        isConnectable={isConnectable}
      />
      <div>{label}</div>
      <Handle
        type="source"
        position={Position.BottomLeft}
        id="b1"
        style={{ background: "green" }}
        isConnectable={isConnectable}
      />
      <Handle
        type="source"
        position={Position.BottomRight}
        id="b2"
        style={{ background: "green" }}
        isConnectable={isConnectable}
      />
    </>
  );
};

export default CustomNodeOneTwo;
\end{minted}

Este código demonstra a estrutura do nódulo "1-2", que recebe uma única entrada e gera duas saídas. Os \textit{handles} são configurados para permitir a conexão nos pontos específicos do nódulo, permitindo a bifurcação do fluxo de dados.

\chapter{Código Completo para Adicionar Nódulo 2-1}
\label{Cap:NodeTwoOneCode}

\begin{minted}[linenos, breaklines, fontsize=\small]{typescript}
import React, { memo } from 'react';
import { Handle, Position } from 'react-flow-renderer';

const NodeTwoOne = ({ data }) => {
  const { label, isConnectable } = data;

  return (
    <>
      <Handle
        type="target"
        position={Position.Left}
        id="input1"
        style={{ background: 'black' }}
        isConnectable={isConnectable}
      />
      <Handle
        type="target"
        position={Position.Top}
        id="input2"
        style={{ background: 'black' }}
        isConnectable={isConnectable}
      />
      <div>{label}</div>
      <Handle
        type="source"
        position={Position.Right}
        id="output"
        style={{ background: 'black' }}
        isConnectable={isConnectable}
      />
    </>
  );
};

export default memo(NodeTwoOne);
\end{minted}

Este código mostra a implementação do componente "Nódulo 2-1" no \textit{canvas} do sistema RADARE. Ele define dois pontos de entrada e um ponto de saída, permitindo a integração de múltiplas fontes de dados em um único fluxo, conforme discutido no texto.

\chapter{Código Completo da Lógica de Visibilidade do Gráfico de Reconciliações no Front-End}
\label{Anexo:HideGraphLogic}

\begin{minted}[fontsize=\small, breaklines=true, linenos]{typescript}
// Função para alternar a visibilidade do gráfico de reconciliações
import { useState } from 'react';

const ReconciliationGraph = () => {
    const [isVisible, setIsVisible] = useState(true);

    // Função para esconder/mostrar o gráfico
    const toggleGraphVisibility = () => {
        setIsVisible(!isVisible);
    };

    return (
        <div>
            <button onClick={toggleGraphVisibility}>
                {isVisible ? 'Esconder Gráfico das Reconciliações' : 'Mostrar Gráfico das Reconciliações'}
            </button>

            {isVisible && (
                <div id="reconciliation-graph">
                    {/* Componente do gráfico de reconciliações */}
                    <ReconciliationGraphComponent />
                </div>
            )}
        </div>
    );
};

export default ReconciliationGraph;
\end{minted}

O código apresentado no Anexo \ref{Anexo:HideGraphLogic} implementa a lógica para alternar a visibilidade do gráfico de reconciliações na interface do usuário. A função `toggleGraphVisibility` controla a exibição do gráfico sem afetar os dados ou outros componentes no \textit{dashboard}.


\chapter{Código Completo da Lógica de Visibilidade do Sidebar de Reconciliações no Front-End}
\label{Anexo:HideSidebarLogic}

\begin{minted}[fontsize=\small, breaklines=true, linenos]{typescript}
// Função para alternar a visibilidade do gráfico de reconciliações
import { useState } from 'react';

const ReconciliationGraph = () => {
    const [isVisible, setIsVisible] = useState(true);

    // Função para esconder/mostrar o gráfico
    const toggleGraphVisibility = () => {
        setIsVisible(!isVisible);
    };

    return (
        <div>
            <button onClick={toggleGraphVisibility}>
                {isVisible ? 'Esconder Gráfico das Reconciliações' : 'Mostrar Gráfico das Reconciliações'}
            </button>

            {isVisible && (
                <div id="reconciliation-graph">
                    {/* Componente do gráfico de reconciliações */}
                    <ReconciliationGraphComponent />
                </div>
            )}
        </div>
    );
};

export default ReconciliationGraph;
\end{minted}

O código apresentado no Anexo \ref{Anexo:HideSidebarLogic} implementa a lógica para alternar a visibilidade do gráfico de reconciliações na interface do usuário. A função `toggleGraphVisibility` controla a exibição do gráfico sem afetar os dados ou outros componentes no \textit{dashboard}.


\chapter{Código Completo da Reconcialiação de Dados no Front-End}
\label{Cap:ReconcileDataCode}

\begin{minted}[fontsize=\small, breaklines=true, linenos]{typescript}
// Função de reconciliação de dados utilizando o método dos multiplicadores de Lagrange
import { Node, Edge } from 'react-flow-renderer';
import { calculateLagrangeReconciliation } from './reconciliationUtils';

type ReconciliationInput = {
    nodes: Node[];
    edges: Edge[];
};

// Função principal que realiza a reconciliação dos dados
export const reconcileData = async (input: ReconciliationInput) => {
    try {
        const { nodes, edges } = input;

        // Extração dos dados dos nós conectados no canvas
        const dataToSend = nodes.map(node => ({
            id: node.id,
            value: node.data.value,
            type: node.type,
        }));

        // Envio dos dados para o back-end para execução da reconciliação
        const response = await fetch('/reconcile', {
            method: 'POST',
            headers: {
                'Content-Type': 'application/json',
            },
            body: JSON.stringify({ nodes: dataToSend, edges }),
        });

        if (response.ok) {
            const reconciledData = await response.json();

            // Atualização dos valores reconciliados no front-end
            reconciledData.nodes.forEach(reconciledNode => {
                const node = nodes.find(n => n.id === reconciledNode.id);
                if (node) {
                    node.data.value = reconciledNode.reconciledValue;
                }
            });

            return { status: 'success', data: reconciledData };
        } else {
            throw new Error('Erro na reconciliação de dados no back-end');
        }
    } catch (error) {
        console.error('Erro na reconciliação de dados:', error);
        return { status: 'error', message: error.message };
    }
};
\end{minted}

Este anexo apresenta o código completo da função de reconciliação de dados no front-end. Ele ilustra como os dados são extraídos dos nós conectados, enviados ao back-end para processamento, e como os valores reconciliados são atualizados e exibidos na interface do usuário.


\mychapter{Código Completo da Rota POST /reconcile}
\label{Anexo:CodigoRouteReconcile}

\begin{minted}[fontsize=\small, linenos]{python}
from flask import Flask, request, jsonify
from your_database_module import store_reconciled_data
from your_reconciliation_module import perform_lagrange_reconciliation

app = Flask(__name__)

@app.route('/reconcile', methods=['POST'])
def reconcile_data():
    try:
        # Recebe os dados do front-end em formato JSON
        data = request.get_json()

        # Processa os dados utilizando o método dos multiplicadores de Lagrange
        reconciled_data = perform_lagrange_reconciliation(data)

        # Armazena os dados reconciliados no banco de dados
        store_reconciled_data(reconciled_data)

        # Retorna uma resposta de sucesso ao front-end
        return jsonify({"status": "success", "reconciled_data": reconciled_data}), 200

    except Exception as e:
        # Em caso de erro, retorna uma mensagem de erro
        return jsonify({"status": "error", "message": str(e)}), 500

if __name__ == '__main__':
    app.run(debug=True)
\end{minted}

Este código demonstra como os dados são recebidos e processados pela rota \texttt{POST /reconcile}, aplicando o método de reconciliação e armazenando os resultados no banco de dados para uso posterior.

\mychapter{Código Completo da Rota GET /results}
\label{Anexo:CodigoRouteResults}

\begin{minted}[frame=lines, fontsize=\small, linenos]{python}
from flask import Flask, jsonify
from your_database_module import retrieve_reconciled_data

app = Flask(__name__)

@app.route('/results', methods=['GET'])
def get_results():
    try:
        # Recupera os dados reconciliados do banco de dados
        reconciled_data = retrieve_reconciled_data()

        # Retorna os dados reconciliados para o front-end
        return jsonify({"status": "success", "reconciled_data": reconciled_data}), 200

    except Exception as e:
        # Em caso de erro, retorna uma mensagem de erro
        return jsonify({"status": "error", "message": str(e)}), 500

if __name__ == '__main__':
    app.run(debug=True)
\end{minted}

Este código ilustra como a rota \texttt{GET /results} foi configurada para recuperar os dados reconciliados do banco de dados e enviar essas informações para o \textit{front-end}. O processo inclui a captura de dados previamente ajustados pelo método de reconciliação, permitindo que sejam exibidos na interface de forma organizada para análise.

\mychapter{Código Completo da Rota POST /upload}
\label{Anexo:CodigoRouteUpload}

\begin{minted}[frame=lines, fontsize=\small, linenos]{python}
from flask import Flask, request, jsonify
import csv
from your_database_module import store_csv_data

app = Flask(__name__)

@app.route('/upload', methods=['POST'])
def upload_file():
    try:
        # Verifica se o arquivo foi enviado na requisição
        if 'file' not in request.files:
            return jsonify({"status": "error", "message": "No file provided"}), 400

        file = request.files['file']
        
        # Lê o conteúdo do arquivo CSV
        csv_data = []
        with open(file.stream, 'r') as csvfile:
            csv_reader = csv.reader(csvfile)
            for row in csv_reader:
                csv_data.append(row)

        # Armazena os dados CSV no banco de dados
        store_csv_data(csv_data)

        # Retorna uma resposta de sucesso ao front-end
        return jsonify({"status": "success", "message": "File uploaded successfully"}), 200

    except Exception as e:
        # Em caso de erro, retorna uma mensagem de erro
        return jsonify({"status": "error", "message": str(e)}), 500

if __name__ == '__main__':
    app.run(debug=True)
\end{minted}

Este código descreve a implementação da rota \texttt{POST /upload}, responsável pelo processamento de arquivos CSV enviados pelo \textit{front-end}. Os dados do arquivo são lidos, processados e armazenados no banco de dados, permitindo que sejam utilizados nas etapas subsequentes de reconciliação e análise.

\mychapter{Código Completo do Serviço de Validação de Dados}
\label{Anexo:CodigoValidacaoDados}

\begin{minted}[fontsize=\footnotesize, breaklines]{python}
def validate_data(data):
    """
    Função para validar os dados recebidos antes do processamento.
    Verifica se todos os campos obrigatórios estão presentes, se os tipos de dados são consistentes
    e se não há valores ausentes ou inválidos.
    """
    required_fields = ['sensor_id', 'timestamp', 'value']
    
    # Verifica se todos os campos obrigatórios estão presentes
    for field in required_fields:
        if field not in data:
            raise ValueError(f"Campo obrigatório ausente: {field}")
    
    # Valida o tipo de cada campo
    if not isinstance(data['sensor_id'], int):
        raise TypeError("O campo 'sensor_id' deve ser um número inteiro.")
    if not isinstance(data['timestamp'], str):
        raise TypeError("O campo 'timestamp' deve ser uma string.")
    if not isinstance(data['value'], (int, float)):
        raise TypeError("O campo 'value' deve ser um número.")
    
    # Verifica se o valor está dentro de um intervalo aceitável (exemplo)
    if data['value'] < 0 or data['value'] > 1000:
        raise ValueError("O valor do campo 'value' está fora do intervalo permitido.")
    
    return True
\end{minted}

Este código implementa a validação dos dados recebidos, garantindo que os campos obrigatórios estejam presentes, que os tipos de dados estejam corretos e que os valores estejam dentro de um intervalo aceitável. Qualquer violação resulta em uma exceção, impedindo que dados inconsistentes sejam processados.

\mychapter{Código Completo do Serviço de Processamento de Dados}
\label{Anexo:CodigoProcessamentoDados}

\begin{minted}[fontsize=\footnotesize, linenos, breaklines, tabsize=4]{python}
import csv
from your_database_module import store_data_in_database

def process_csv(file_path):
    try:
        # Abre o arquivo CSV
        with open(file_path, mode='r') as file:
            csv_reader = csv.DictReader(file)
            processed_data = []

            # Processa cada linha do CSV
            for row in csv_reader:
                # Conversão de dados e validação de tipos
                data_entry = {
                    "sensor_id": int(row["sensor_id"]),
                    "value": float(row["value"]),
                    "timestamp": row["timestamp"]
                }
                processed_data.append(data_entry)

            # Armazena os dados processados no banco de dados
            store_data_in_database(processed_data)
            return {"status": "success", "message": "Dados processados e armazenados com sucesso"}

    except Exception as e:
        # Em caso de erro, retorna uma mensagem de erro
        return {"status": "error", "message": str(e)}
\end{minted}

Este código demonstra como o serviço de processamento de dados manipula arquivos CSV enviados pelo usuário, validando e convertendo os dados para o formato adequado antes de armazená-los no banco de dados do sistema RADARE.

\mychapter{Código Completo do Serviço de Reconciliação de Dados}
\label{Anexo:CodigoReconciliacaoDados}

\begin{minted}[fontsize=\footnotesize, linenos, breaklines, tabsize=2]{python}
from your_database_module import store_reconciled_data
from flask import Flask, request, jsonify
from your_reconciliation_module import perform_lagrange_reconciliation

app = Flask(__name__)

@app.route('/reconcile', methods=['POST'])
def reconcile_data():
    try:
        # Recebe os dados do front-end em formato JSON
        data = request.get_json()

        # Executa o processo de reconciliação utilizando o método dos multiplicadores de Lagrange
        reconciled_data = perform_lagrange_reconciliation(data)

        # Armazena os dados reconciliados no banco de dados
        store_reconciled_data(reconciled_data)

        # Retorna uma resposta de sucesso ao front-end com os dados reconciliados
        return jsonify({"status": "success", "reconciled_data": reconciled_data}), 200

    except Exception as e:
        # Em caso de erro, retorna uma mensagem de erro
        return jsonify({"status": "error", "message": str(e)}), 500

if __name__ == '__main__':
    app.run(debug=True)
\end{minted}

Este código implementa o serviço de reconciliação de dados do sistema RADARE, aplicando o método dos multiplicadores de Lagrange para ajustar os dados recebidos e garantir consistência de acordo com as restrições impostas. Ele também armazena os resultados reconciliados no banco de dados, permitindo consultas futuras.

\mychapter{Código Completo dos Modelos do Sistema}
\label{Anexo:CodigoModelo}

\section*{Process Model}
\begin{minted}[linenos, breaklines, fontsize=\small]{python}
from sqlalchemy import Column, Integer, String, Float, ForeignKey
from sqlalchemy.orm import relationship
from database import Base

class Process(Base):
    __tablename__ = 'processes'
    id = Column(Integer, primary_key=True, index=True)
    name = Column(String, index=True)
    description = Column(String)

    measurements = relationship("Measurement", back_populates="process")
    results = relationship("Result", back_populates="process")
\end{minted}

\section*{Measurement Model}
\begin{minted}[linenos, breaklines, fontsize=\small]{python}
from sqlalchemy import Column, Integer, Float, ForeignKey
from sqlalchemy.orm import relationship
from database import Base

class Measurement(Base):
    __tablename__ = 'measurements'
    id = Column(Integer, primary_key=True, index=True)
    process_id = Column(Integer, ForeignKey('processes.id'))
    value = Column(Float)
    unit = Column(String)

    process = relationship("Process", back_populates="measurements")
\end{minted}

\section*{Result Model}
\begin{minted}[linenos, breaklines, fontsize=\small]{python}
from sqlalchemy import Column, Integer, Float, ForeignKey
from sqlalchemy.orm import relationship
from database import Base

class Result(Base):
    __tablename__ = 'results'
    id = Column(Integer, primary_key=True, index=True)
    process_id = Column(Integer, ForeignKey('processes.id'))
    reconciled_value = Column(Float)
    discrepancy = Column(Float)

    process = relationship("Process", back_populates="results")
\end{minted}

Este anexo apresenta o código completo dos modelos desenvolvidos para o sistema RADARE, que foram implementados utilizando SQLAlchemy como a ORM em Python para a interação com o banco de dados PostgreSQL. Cada modelo representa uma entidade essencial para o funcionamento do sistema, garantindo a integridade e o gerenciamento adequado dos dados do projeto.

\mychapter{Código Principal para a Lógica de Conexão entre Nódulos no \textit{Canvas}}
\label{Anexo:frontCodeNodeTwoOne}

\begin{minted}[linenos,breaklines,fontsize=\small]{typescript}
// Função para estabelecer conexão entre nódulos no canvas
function connectNodes(nodeA: Node, nodeB: Node, canvas: Canvas) {
    // Verifica se os nódulos são compatíveis para conexão
    if (isValidConnection(nodeA, nodeB)) {
        // Cria a linha de conexão entre os nódulos
        const connection = createConnection(nodeA, nodeB);

        // Define propriedades da conexão, como valor e tolerância padrão
        connection.value = getDefaultValue(nodeA, nodeB);
        connection.tolerance = getDefaultTolerance(nodeA, nodeB);

        // Adiciona a conexão ao canvas
        canvas.addConnection(connection);

        // Atribui um nome aleatório para fácil identificação
        connection.name = generateRandomName();

        // Configura a interação de duplo clique para ajustar valor e tolerância
        connection.onDoubleClick(() => {
            const newValue = prompt("Digite o novo valor para a conexão:");
            const newTolerance = prompt("Digite a nova tolerância:");
            if (newValue !== null) connection.value = parseFloat(newValue);
            if (newTolerance !== null) connection.tolerance = parseFloat(newTolerance);
        });
    } else {
        console.error("Conexão inválida entre nódulos.");
    }
}

// Função para validar se a conexão entre os nódulos é permitida
function isValidConnection(nodeA: Node, nodeB: Node): boolean {
    // Exemplo de validação básica entre tipos de nódulos
    return nodeA.outputType === nodeB.inputType;
}

// Função para criar um objeto de conexão
function createConnection(nodeA: Node, nodeB: Node): Connection {
    return {
        source: nodeA.id,
        target: nodeB.id,
        value: 0,
        tolerance: 0,
        name: ""
    };
}

// Função para gerar um nome aleatório para a conexão
function generateRandomName(): string {
    return `Conexao-${Math.random().toString(36).substr(2, 9)}`;
}
\end{minted}

Este código implementa a lógica principal para criar e gerenciar conexões entre nódulos no \textit{canvas}, incluindo a configuração de valores e tolerâncias ajustáveis diretamente pela interface, conforme descrito na seção correspondente.

\mychapter{Manual de Manutenção do Sistema}
\label{Cap:manualManutencao}

\section*{Manual de Manutenção do Sistema RADARE}

Este manual foi elaborado para guiar desenvolvedores e administradores nas operações de manutenção do sistema RADARE. Ele cobre os aspectos principais de cada módulo, incluindo *front-end*, *back-end* e banco de dados, e descreve as instruções para atualização, monitoramento e solução de problemas.

\subsection*{Estrutura do Sistema}

O RADARE é dividido em três módulos principais:
\begin{itemize}
    \item \textbf{Front-end}: Responsável pela interface de usuário e pela interação com os dados no *canvas*.
    \item \textbf{Back-end}: Lida com a lógica de negócios, o processamento dos dados e a reconciliação usando os multiplicadores de Lagrange.
    \item \textbf{Banco de dados}: Armazena informações críticas, como dados de sensores, resultados de reconciliação e logs de atividade.
\end{itemize}

\subsection*{Tarefas de Manutenção}

\begin{itemize}
    \item \textbf{Atualização de Bibliotecas}: Verifique periodicamente por atualizações de dependências do sistema e realize os ajustes necessários.
    \item \textbf{Ajustes de Configurações}: Mantenha as configurações de sistema atualizadas para suportar o máximo de eficiência e compatibilidade.
    \item \textbf{Monitoramento de Logs}: Examine regularmente os logs de atividade para identificar e resolver possíveis falhas no sistema.
    \item \textbf{Execução de Testes de Desempenho}: Realize testes de desempenho após grandes atualizações para garantir a estabilidade do sistema.
\end{itemize}

\subsection*{Limitações Técnicas}

O sistema possui algumas restrições:
\begin{itemize}
    \item \textbf{Conexões simultâneas}: Suporta um número limitado de conexões simultâneas, variável conforme a infraestrutura.
    \item \textbf{Tamanho dos arquivos *CSV*}: Limite para importação de arquivos de dados.
    \item \textbf{Requisitos de Hardware}: Especificações mínimas para o servidor, incluindo memória e capacidade de processamento.
\end{itemize}

\subsection*{Procedimentos para Resolução de Problemas}

Este manual também oferece orientações para solução de problemas comuns:
\begin{itemize}
    \item \textbf{Erros de Reconciliação}: Verifique os dados de entrada e a lógica de reconciliação caso ocorram inconsistências.
    \item \textbf{Falhas no Processamento de Arquivos}: Certifique-se de que os arquivos seguem o formato especificado.
    \item \textbf{Problemas de Conexão entre Módulos}: Revise as configurações de rede e as dependências entre o *front-end* e o *back-end*.
\end{itemize}

Este manual fornece uma referência completa para garantir a continuidade e o desempenho do sistema RADARE.

\chapter{Manual de Uso para o Usuário Final}
\label{Anexo:manualUsuario}

Este manual fornece instruções detalhadas para o uso do sistema *RADARE* em tarefas cotidianas de reconciliação de dados industriais. Ele abrange uma introdução ao sistema, guias passo a passo de operações e soluções para problemas comuns.

\section*{1. Introdução ao Sistema}

O *RADARE* é uma ferramenta voltada para a reconciliação de dados em processos industriais, facilitando a análise e ajuste de dados obtidos de diversas fontes, como sensores e sistemas de monitoramento.

\section*{2. Passo a Passo das Operações}

\subsection*{Adicionar Nódulos no \textit{Canvas}}

Para adicionar nódulos, selecione o tipo de nódulo desejado no menu e arraste-o para o \textit{canvas}. Esses nódulos podem representar entradas de dados, pontos de processamento ou saídas. 

\subsection*{Realizar Reconciliação de Dados}

Após conectar os nódulos, clique no botão de "Reconciliar Dados" para executar o processo de reconciliação, que ajustará os dados de acordo com os critérios de balanço estabelecidos.

\subsection*{Carregar Arquivos CSV}

Para importar dados, utilize a função de upload para carregar arquivos CSV no sistema. Os dados serão processados e integrados ao fluxo de reconciliação.

\section*{3. Guias Visuais e Dicas de Usabilidade}

O manual inclui imagens de cada funcionalidade, mostrando as operações principais, como navegação na interface e geração de relatórios. Além disso, oferece dicas para maximizar a eficiência no uso do sistema, incluindo atalhos e personalização de fluxos de trabalho.

\section*{4. Resolução de Problemas Comuns}

Esta seção abrange soluções para problemas frequentes, como:
\begin{itemize}
    \item Erros de upload de arquivos: Verifique o formato e o conteúdo do arquivo CSV.
    \item Falhas na conexão entre nódulos: Assegure que os nódulos estejam corretamente conectados no \textit{canvas}.
    \item Dificuldades na interpretação dos resultados: Consulte as dicas de usabilidade para obter orientações sobre análise de dados.
\end{itemize}

Este manual visa fornecer uma experiência de uso intuitiva, facilitando o acesso às funcionalidades do *RADARE*.

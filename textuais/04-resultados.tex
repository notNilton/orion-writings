\mychapter{Resultados}
\label{Cap:Resultados}

Neste capítulo, são apresentados os resultados obtidos a partir da implementação e testes do \textit{software} RADARE. As principais dimensões avaliadas incluem a precisão dos dados reconciliados, o desempenho computacional em diferentes cenários de carga de trabalho, e a usabilidade da ferramenta, com foco em aspectos específicos do front-end, como a interação com o menu e o canvas, e do back-end, como as rotas, serviços e a integração com o banco de dados. Os testes foram realizados com usuários de diferentes perfis, permitindo avaliar tanto a eficiência técnica quanto a experiência de uso do sistema.

\begin{verbatim}
const customNodeOneOne = ({ data }) => {
  const { label, isConnectable } = data;

  return (
    <>
      <Handle
        type="source"
        id="a"
        position={Position.Left}
        style={{ background: "black" }}
        isConnectable={isConnectable}
      />
      <Handle
        type="source"
        position={Position.Right}
        id="b"
        style={{ background: "black" }}
        isConnectable={isConnectable}
      />
      <div>{`${label}`}</div>
    </>
  );
};

export default memo(customNodeOneOne);
\end{verbatim}

\section{Front-end}

O front-end do RADARE foi projetado para proporcionar uma interface intuitiva, permitindo que os usuários interajam de maneira eficiente com a ferramenta de modelagem. Ele facilita a visualização de fluxos de dados e modelos, além de executar a reconciliação de maneira prática. As duas principais áreas do front-end são o menu, que permite o gerenciamento das ações, e o \textit{canvas}, onde os usuários podem visualizar e manipular os nódulos conectados.

\subsection{Sessão de menu}

A sessão de menu do RADARE oferece ao usuário uma interface intuitiva para adicionar componentes, controlar o fluxo de dados e gerenciar a visualização do sistema. Cada item disponível no menu é detalhado a seguir, acompanhado de exemplos de código e imagens que ilustram sua funcionalidade e implementação no contexto da ferramenta.

A biblioteca \textit{ReactFlow} \cite{reactflow} desempenha um papel central na manipulação dos nódulos dentro do \textit{canvas}, sendo amplamente adaptada para atender às necessidades específicas do projeto. Modificações foram implementadas para garantir uma usabilidade fluida e eficiente, permitindo que os usuários adicionem e conectem nódulos de forma dinâmica.

Cada nódulo é constituído a partir de uma estrutura definida, onde as conexões, conhecidas como \textit{handles}, são personalizadas com características específicas, como estilos visuais e lógica de interação. Essa flexibilidade garante que a visualização e que facilite a manipulação dos fluxos de dados no \textit{canvas}.

A figura a seguir mostra o menu principal da ferramenta, onde o usuário pode visualizar as opções de adição de componentes ao \textit{canvas}. Por meio desse menu, é possível inserir diferentes tipos de nódulos, como entradas, saídas e pontos de processamento de dados, além de outras opções como a reconciliação de dados e ajustes de visualização. Cada função disponível no menu permite que o usuário construa fluxos de dados industriais de forma modular e interativa, facilitando o gerenciamento e análise de grandes volumes de dados.

\begin{figure}[htbp]
    \centering
    \includegraphics[width=0.6\textwidth]{figuras/add_input.png}
    \caption{Nó de input adicionado ao canvas.}
    \label{Fig:AddInput}
\end{figure}



\subsubsection{Adicionar input}

O botão \textbf{"Adicionar Input"} permite ao usuário inserir um novo nódulo de entrada no \textit{canvas}. Esse \textit{input} representa um sensor ou uma fonte de dados do sistema industrial. Ao clicar no botão, um novo nó de \textit{input} é adicionado ao \textit{canvas}, permitindo que o usuário conecte esse ponto a outros nódulos do fluxo de dados. A lógica computacional associada a esse comportamento é fornecida pela biblioteca ReactFlow \cite{reactflow}, eliminando a necessidade de configurar comportamentos iniciais customizados.

\subsubsection{Adicionar output}

O botão \textbf{"Adicionar Output"} permite ao usuário inserir um novo nódulo de saída no \textit{canvas}. Esse \textit{output} representa um destino ou ponto final para os dados no sistema industrial, como a exportação de resultados processados ou a visualização de dados reconciliados. Ao clicar no botão, um novo nó de \textit{output} é adicionado ao \textit{canvas}, permitindo que o usuário conecte este nó a outros nódulos de processamento ou entrada no fluxo de dados. Assim como o \textit{input}, a lógica necessária para a adição desse nódulo é fornecida pela biblioteca ReactFlow \cite{reactflow}, dispensando a necessidade de customização de comportamentos iniciais.

\subsubsection{Adicionar nódulo 1-1}

O botão \textbf{"Adicionar Nódulo 1-1"} permite ao usuário inserir um novo nódulo de transição no \textit{canvas}. Esse nódulo atua como um ponto intermediário no fluxo de dados, podendo representar sensores, transformações ou outros elementos de processo. Ao clicar no botão, o nódulo é adicionado ao \textit{canvas}, permitindo ao usuário conectá-lo com outros nódulos de forma eficiente.

A lógica para adicionar este nódulo foi desenvolvida de forma personalizada, especificando o tipo de conexão, a quantidade de pontos de conexão (\textit{handles}), além do estilo visual e da posição no \textit{canvas}, garantindo que o comportamento do nódulo se ajuste adequadamente ao fluxo de dados esperado.

O trecho principal do código responsável pela criação desse nódulo, que pode ser encontrado em sua totalidade no \textbf{Apêndice \ref{Cap:frontCodeNodeOneOne}}.

\subsubsection{Adicionar nódulo 1-2}

O botão \textbf{"Adicionar Nódulo 1-2"} permite ao usuário inserir um nódulo de transição que recebe uma única entrada e gera duas saídas no \textit{canvas}. Este tipo de nódulo é particularmente útil em cenários onde um único ponto de dados precisa ser bifurcado para diferentes processos ou análises. Ao ser adicionado, o nódulo facilita o roteamento de dados para dois fluxos distintos, mantendo a integridade e a flexibilidade do processo.

A lógica para este nódulo foi customizada para suportar uma conexão de entrada e duas saídas, com o código responsável definindo os \textit{handles} (pontos de conexão), a posição e o estilo visual no \textit{canvas}. Assim como no caso do nódulo 1-1, o comportamento é ajustado para garantir uma integração fluida no fluxo de dados. 

O trecho principal do código responsável pela criação desse nódulo, que pode ser encontrado em sua totalidade no \textbf{Apêndice \ref{Cap:frontCodeNodeOneTwo}}.

\subsubsection{Adicionar nódulo 2-1}

O botão \textbf{"Adicionar Nódulo 2-1"} permite ao usuário inserir um nódulo de transição que recebe duas entradas e gera uma única saída no \textit{canvas}. Esse nódulo é ideal para processos em que múltiplas fontes de dados precisam ser combinadas ou integradas antes de continuar o fluxo. Ao ser adicionado, o nódulo permite a fusão de duas linhas de dados, garantindo que as informações de entrada sejam processadas de forma conjunta antes de seguirem para a próxima etapa.

A lógica para este nódulo foi desenvolvida de maneira personalizada, permitindo a adição de dois pontos de conexão de entrada e um ponto de saída. O código responsável configura os \textit{handles}, define o estilo visual e posiciona o nódulo no \textit{canvas}, assegurando que ele atenda às necessidades de integração e processamento combinados dentro do fluxo de dados.

O trecho principal do código responsável pela criação desse nódulo, que pode ser encontrado em sua totalidade no \textbf{Apêndice \ref{Cap:frontCodeNodeTwoOne}}.

\subsubsection{Reconciliar dados}

O botão \textbf{"Reconciliar Dados"} executa o processo de reconciliação dos dados que estão conectados no \textit{canvas}. Ao ser acionado, o sistema analisa os nódulos que estão interconectados e realiza a reconciliação de dados utilizando o método dos multiplicadores de Lagrange. Este processo ajusta as discrepâncias entre os valores medidos e os reconciliados, respeitando as restrições impostas pelos balanços de massa e energia.

A lógica por trás desse botão foi configurada para percorrer os nódulos conectados no \textit{canvas}, extrair os dados relevantes e enviar essas informações para o back-end. Lá, o algoritmo de reconciliação é executado, e os resultados são retornados ao front-end, onde os dados reconciliados são exibidos de forma atualizada no fluxo de dados visual.

O trecho principal do código responsável pela criação desse nódulo, que pode ser encontrado em sua totalidade no \textbf{Apêndice \ref{Cap:frontdatarecon}}.

\subsubsection{Esconder gráfico das reconciliações}

O botão \textbf{"Esconder Gráfico das Reconciliações"} oferece ao usuário a funcionalidade de ocultar o gráfico que exibe os resultados das reconciliações de dados, proporcionando uma interface mais limpa e com mais espaço para focar em outros elementos do processo. Ao ser acionado, o gráfico é ocultado do \textit{dashboard}, mas os dados continuam disponíveis no sistema, permitindo que o usuário possa reexibir o gráfico quando necessário. A lógica por trás deste botão foi construída para alternar a visibilidade do componente gráfico sem afetar o restante da interface ou dos dados processados.

O trecho principal do código responsável pela criação desse nódulo, que pode ser encontrado em sua totalidade no \textbf{Apêndice \ref{Cap:frontCodeNodeTwoOne}}.

\subsubsection{Esconder sidebar de informações}

O botão \textbf{"Esconder Sidebar de Informações"} permite ao usuário ocultar a barra lateral que exibe informações detalhadas sobre os nós e fluxos no \textit{canvas}. Essa barra lateral geralmente contém dados críticos e estatísticas sobre os elementos do processo, e escondê-la oferece mais espaço para visualização e manipulação dos nódulos, especialmente em fluxos mais complexos. 

A implementação desse recurso foi projetada para alternar a visibilidade da \textit{sidebar} sem perder os dados exibidos, permitindo que o usuário a reexiba com todas as informações intactas sempre que necessário. A funcionalidade foi desenvolvida para garantir uma interface dinâmica, adaptando-se conforme a necessidade de foco do usuário no \textit{canvas}.

O trecho principal do código responsável pela criação desse nódulo, que pode ser encontrado em sua totalidade no \textbf{Apêndice \ref{Cap:frontCodeNodeTwoOne}}.

\subsubsection{Upload de arquivos em CSV}

O botão \textbf{"Upload de Arquivos CSV"} oferece ao usuário a possibilidade de importar dados de medições diretamente para o sistema. Este recurso facilita a integração de dados externos, permitindo que os arquivos CSV sejam carregados com as informações dos sensores e variáveis do processo industrial. Uma vez carregados, os dados são processados e utilizados nas etapas de reconciliação realizadas pelo RADARE.

A implementação dessa funcionalidade envolve o envio e o armazenamento seguro dos arquivos, além da validação de seu formato e conteúdo. Isso garante que os dados estejam corretamente estruturados antes de serem integrados ao banco de dados para posterior análise e reconciliação.

O trecho principal do código responsável pela criação desse nódulo, que pode ser encontrado em sua totalidade no \textbf{Apêndice \ref{Cap:frontCodeNodeTwoOne}}.

\subsection{Sessão de canvas}

O canvas é a área principal da interface do RADARE, onde o usuário pode visualizar, conectar e manipular os nódulos para configurar o fluxo de dados industrial. Nesta seção, descrevemos as funcionalidades disponíveis no canvas, incluindo exemplos de código e imagens que ilustram o comportamento da interface.

\subsubsection{Conectar nódulos}

A funcionalidade de "Conectar Nódulos" permite ao usuário estabelecer conexões visuais entre os nódulos de entrada, processamento e saída. Estas conexões representam o fluxo de dados entre diferentes pontos de um processo industrial.

O trecho principal do código responsável pela criação desse nódulo, que pode ser encontrado em sua totalidade no \textbf{Apêndice \ref{Cap:frontCodeNodeTwoOne}}.
\subsubsection{Ajustar conexões entre nódulos}

O canvas também permite que o usuário ajuste e mova as conexões entre os nódulos. Isso proporciona flexibilidade na organização dos fluxos de dados, permitindo que o layout seja personalizado conforme as necessidades do usuário.

O trecho principal do código responsável pela criação desse nódulo, que pode ser encontrado em sua totalidade no \textbf{Apêndice \ref{Cap:frontCodeNodeTwoOne}}.

\section{Back-end}

O back-end do RADARE foi desenvolvido em \textit{Node.js}, com foco em desempenho, escalabilidade e segurança. Ele é responsável por gerenciar as requisições da interface, processar os dados submetidos e realizar os cálculos de reconciliação utilizando o método dos multiplicadores de Lagrange, além de retornar os resultados para o front-end de forma otimizada.

\subsection{Routes}

As \textit{routes} do sistema foram estruturadas para facilitar a comunicação entre a interface gráfica e a API, garantindo que as requisições de dados e os processos de reconciliação sejam executados de maneira eficiente. Abaixo estão descritas as principais rotas implementadas no sistema, com exemplos de código e uma explicação detalhada de cada funcionalidade.

\subsubsection{POST /reconcile}

Esta rota recebe os dados dos sensores e processa a reconciliação utilizando o método dos multiplicadores de Lagrange. O fluxo de dados é enviado pelo \textit{front-end} e, após o processamento, os resultados são armazenados no banco de dados para serem recuperados posteriormente.

O trecho principal do código responsável pela criação desse nódulo, que pode ser encontrado em sua totalidade no \textbf{Apêndice \ref{Cap:frontCodeNodeTwoOne}}.

\subsubsection{GET /results}

A rota \texttt{GET /results} permite ao usuário recuperar os resultados das reconciliações realizadas anteriormente. Esses resultados são exibidos na interface gráfica, fornecendo ao usuário uma visão dos dados reconciliados e permitindo a análise dos processos.

O trecho principal do código responsável pela criação desse nódulo, que pode ser encontrado em sua totalidade no \textbf{Apêndice \ref{Cap:frontCodeNodeTwoOne}}.

\subsubsection{POST /upload}

A rota \texttt{POST /upload} gerencia o envio de arquivos CSV para o banco de dados. Esta funcionalidade é fundamental para permitir que o sistema integre dados externos de sensores, facilitando a reconciliação e o processamento.

O trecho principal do código responsável pela criação desse nódulo, que pode ser encontrado em sua totalidade no \textbf{Apêndice \ref{Cap:frontCodeNodeTwoOne}}.

\subsection{Services}

Os \textit{services} no \textit{back-end} são responsáveis por implementar a lógica de negócio, processar os dados e invocar os cálculos de reconciliação. Eles abstraem a complexidade do sistema, garantindo que os dados sejam processados corretamente antes de serem enviados ao banco de dados ou utilizados na interface. Abaixo estão descritos os principais \textit{services} do sistema RADARE, juntamente com exemplos de código e diagramas.

\subsubsection{Reconciliation Service}

O \textit{Reconciliation Service} é o núcleo do sistema, responsável pela execução do algoritmo de reconciliação de dados. Ele aplica o método dos multiplicadores de Lagrange para garantir que as restrições de balanço de massa e energia sejam respeitadas durante a reconciliação dos dados.

O trecho principal do código responsável pela criação desse nódulo, que pode ser encontrado em sua totalidade no \textbf{Apêndice \ref{Cap:frontCodeNodeTwoOne}}.
\subsubsection{Data Validation Service}

O \textit{Data Validation Service} garante que os dados recebidos estejam no formato correto antes de serem processados. Ele valida se todos os campos obrigatórios estão presentes, se os tipos de dados são consistentes e se não há valores ausentes ou inválidos.

O trecho principal do código responsável pela criação desse nódulo, que pode ser encontrado em sua totalidade no \textbf{Apêndice \ref{Cap:frontCodeNodeTwoOne}}.

\subsubsection{File Processing Service}

O \textit{File Processing Service} é responsável por lidar com arquivos CSV enviados pelos usuários. Ele converte os dados do arquivo em estruturas adequadas para o banco de dados e para os cálculos de reconciliação. Este serviço é crucial para integrar dados externos ao sistema RADARE.

O trecho principal do código responsável pela criação desse nódulo, que pode ser encontrado em sua totalidade no \textbf{Apêndice \ref{Cap:frontCodeNodeTwoOne}}.

\subsection{Models}

Os \textit{models} foram implementados utilizando \textit{Sequelize}, uma ORM (Object-Relational Mapping) para \textit{Node.js}, que facilita a comunicação com o banco de dados PostgreSQL. Os principais modelos incluem:

\begin{itemize} \item \textbf{Process Model}: Representa os processos industriais, contendo as informações de variáveis medidas e parâmetros. \item \textbf{Measurement Model}: Armazena as medições recebidas dos sensores. \item \textbf{Result Model}: Registra os resultados das reconciliações, vinculando-os aos processos e medições correspondentes. \end{itemize}

O trecho principal do código responsável pela criação desse nódulo, que pode ser encontrado em sua totalidade no \textbf{Apêndice \ref{Cap:frontCodeNodeTwoOne}}.

\section{Banco de Dados}

O banco de dados utilizado no sistema RADARE foi implementado em PostgreSQL e armazena todas as informações relevantes para a execução do processo de reconciliação de dados industriais, gerenciamento de usuários e rastreamento de atividades. A modelagem foi feita de forma a garantir a integridade e eficiência na consulta e manipulação dos dados. A seguir, são descritas as principais tabelas implementadas no sistema.

\subsection{Tabela de Dados de Processos}

A tabela que armazena os dados de processos industriais contém as medições e os resultados da reconciliação. As principais colunas são:

\begin{itemize}
    \item \textbf{id}: Coluna de identificação única para cada registro no banco de dados. Utilizada como chave primária para vincular informações em diferentes tabelas.
    \item \textbf{user}: Armazena informações sobre o usuário que realizou o upload dos dados ou a operação de reconciliação.
    \item \textbf{time}: Representa o timestamp associado ao dado, indicando o momento exato em que a medição foi realizada ou quando a reconciliação foi executada.
    \item \textbf{tagname}: Identifica a variável medida (sensor ou ponto de coleta) no processo industrial. Cada variável é referenciada por um nome único.
    \item \textbf{tagreconciled}: Armazena o valor reconciliado de cada variável, ou seja, o resultado do processo de reconciliação de dados.
    \item \textbf{tagcorrection}: Coluna que contém o valor da correção aplicada a cada variável medida.
    \item \textbf{tagmatrix}: Refere-se à matriz de correlação utilizada no processo de reconciliação.
\end{itemize}

\subsection{Tabela de Usuários}

Esta tabela gerencia a autenticação e as permissões dos usuários no sistema. Ela permite o rastreamento de interações e garante que apenas usuários autorizados possam realizar determinadas operações, além de armazenar informações para autenticação segura.

\begin{table}[htbp]
    \centering
    \caption{Estrutura da tabela de usuários.}
    \label{Tab:Users}
    \begin{tabular}{|c|c|}
        \hline
        \textbf{Coluna} & \textbf{Descrição} \\ \hline
        id (primary key) & UUID único para cada usuário \\ \hline
        username & Nome de usuário utilizado no login \\ \hline
        email & Email do usuário para notificações e recuperação de senha \\ \hline
        password\_hash & Hash da senha para garantir segurança \\ \hline
        role & Papel do usuário (admin, operador, analista) \\ \hline
        created\_at & Timestamp da criação da conta \\ \hline
        last\_login & Data e hora do último login realizado \\ \hline
    \end{tabular}
\end{table}

\subsection{Tabela de Logs de Atividade}

Esta tabela registra as ações realizadas pelos usuários no sistema, permitindo auditoria e rastreamento de atividades como uploads de arquivos ou execuções de reconciliação de dados.

\begin{table}[htbp]
    \centering
    \caption{Estrutura da tabela de logs de atividade.}
    \label{Tab:ActivityLogs}
    \begin{tabular}{|c|c|}
        \hline
        \textbf{Coluna} & \textbf{Descrição} \\ \hline
        id (primary key) & Identificação única para cada log de atividade \\ \hline
        user\_id & Referência ao id do usuário que realizou a ação \\ \hline
        action & Descrição da ação realizada (ex: upload de CSV, reconciliação de dados) \\ \hline
        timestamp & Data e hora em que a ação foi realizada \\ \hline
        details & Detalhes adicionais sobre a ação (ex: nome do arquivo CSV carregado) \\ \hline
    \end{tabular}
\end{table}

\subsection{Tabela de Configurações de Processos}

Esta tabela armazena as configurações dos processos industriais, incluindo parâmetros como limites de variáveis e tipos de medições. Essa estrutura permite personalizar o comportamento do sistema para diferentes cenários industriais.

\begin{table}[htbp]
    \centering
    \caption{Estrutura da tabela de configurações de processos.}
    \label{Tab:ProcessConfigurations}
    \begin{tabular}{|c|c|}
        \hline
        \textbf{Coluna} & \textbf{Descrição} \\ \hline
        id (primary key) & Identificação única para a configuração do processo \\ \hline
        process\_name & Nome do processo industrial \\ \hline
        sensor\_limits & Limites definidos para as variáveis dos sensores \\ \hline
        created\_by & ID do usuário que criou a configuração \\ \hline
        created\_at & Timestamp de criação da configuração \\ \hline
    \end{tabular}
\end{table}

Essa modelagem de banco de dados garante rastreabilidade, segurança e eficiência no gerenciamento de processos industriais e usuários dentro do sistema RADARE.



\section{Avaliações do sistema}

Para obter feedback sobre a eficiência e a usabilidade do \textit{software} RADARE, duas avaliações foram conduzidas de forma simulada. Ambas as avaliações foram realizadas por meio de formulários online utilizando a ferramenta Google Forms, permitindo uma coleta de dados simples e direta dos participantes. Os resultados obtidos fornecem insights sobre o desempenho e a usabilidade do sistema a partir da perspectiva de usuários finais simulados.

\subsection{Avaliação do desempenho do sistema}

A avaliação de desempenho do RADARE foi projetada para medir a eficiência do sistema em termos de velocidade de processamento, precisão de reconciliação de dados e escalabilidade em diferentes cenários de carga. Usuários fictícios testaram o sistema em situações que simulam o uso em ambientes industriais com cargas variadas, como o processamento de dados em tempo real e o manuseio de grandes volumes de medições.

As perguntas do formulário foram estruturadas para avaliar a satisfação com o tempo de resposta do sistema e a qualidade dos resultados produzidos. O formulário incluiu questões como:

\begin{itemize}
    \item Qual foi a velocidade de processamento observada em diferentes volumes de dados?
    \item Os resultados reconciliados atenderam às expectativas de precisão?
    \item O sistema se mostrou estável ao processar grandes quantidades de dados simultaneamente?
\end{itemize}

A partir das respostas coletadas, foi possível gerar métricas de desempenho, embora fictícias, que indicam que o sistema apresentou uma performance adequada em 90% dos cenários simulados.

\subsection{Avaliação de usabilidade do sistema}

Além do desempenho técnico, a usabilidade do sistema também foi avaliada por meio de um Google Forms direcionado a usuários simulados. O objetivo dessa avaliação foi medir a facilidade de uso, a clareza da interface e a eficiência no suporte à execução das tarefas relacionadas à reconciliação de dados.

Os usuários fictícios foram convidados a realizar ações como adicionar nódulos no \textit{canvas}, ajustar fluxos de dados e visualizar gráficos de reconciliação, e a responder perguntas relacionadas à sua experiência de uso. Algumas das perguntas incluídas foram:

\begin{itemize}
    \item A interface gráfica foi intuitiva e de fácil navegação?
    \item A adição e conexão de nódulos no \textit{canvas} foi realizada de forma clara e sem dificuldades?
    \item A visualização dos resultados foi satisfatória em termos de clareza e apresentação?
\end{itemize}

Os resultados mostraram, de maneira simulada, que 85\% dos participantes classificaram a usabilidade do sistema como "boa" ou "excelente", destacando a simplicidade na navegação e a organização dos componentes da interface.

\section{Manuais de sistema}

Como parte do desenvolvimento do \textit{software} RADARE, foram criados dois manuais distintos para auxiliar tanto os administradores técnicos quanto os usuários finais na utilização e manutenção do sistema. Esses manuais visam garantir a correta operação e longevidade da aplicação, fornecendo instruções claras sobre o funcionamento do sistema e as melhores práticas para sua utilização.

\subsection{Manual de manutenção do sistema}

O manual de manutenção do sistema foi elaborado para orientar desenvolvedores e administradores sobre as funções internas do RADARE, bem como as restrições e considerações técnicas envolvidas. Ele oferece uma visão detalhada da arquitetura do sistema, abordando aspectos como:

\begin{itemize}
    \item **Estrutura do Código**: Explicação de cada módulo do sistema, como o front-end, back-end e banco de dados. Inclui descrições das principais funções, fluxos de dados e interações entre componentes.
    \item **Funções de Manutenção**: Instruções para realizar tarefas de manutenção, como atualização de bibliotecas, ajustes de configurações, monitoramento de logs e execução de testes de desempenho.
    \item **Restrições do Sistema**: Detalhes sobre limitações técnicas do sistema, como o número máximo de conexões simultâneas suportadas, tamanhos máximos de arquivos CSV, e requisitos mínimos de hardware e software.
    \item **Solucionamento de Problemas**: Procedimentos para diagnosticar e resolver possíveis problemas, como erros de reconciliação, falhas no processamento de arquivos, ou dificuldades de conexão entre os módulos.
\end{itemize}

Esse manual é voltado para profissionais de TI e desenvolvedores responsáveis pela manutenção e suporte contínuo do sistema RADARE.

\subsection{Manual de uso para o usuário final}

O manual de uso foi criado com o objetivo de fornecer ao usuário comum uma orientação simples e direta sobre como utilizar o RADARE para realizar tarefas cotidianas de reconciliação de dados industriais. Ele inclui:

\begin{itemize}
    \item **Introdução ao Sistema**: Explicação sobre o propósito do RADARE, suas funcionalidades principais e como ele pode ajudar a melhorar a qualidade dos dados em processos industriais.
    \item **Passo a Passo de Operações**: Instruções detalhadas sobre como utilizar cada parte da interface, como adicionar nódulos no \textit{canvas}, realizar reconciliações de dados, e carregar arquivos CSV.
    \item **Guias Visuais**: Imagens ilustrativas que mostram o processo de navegação na interface, desde a adição de nódulos até a geração de gráficos de reconciliação.
    \item **Dicas de Usabilidade**: Sugestões para o uso eficiente da ferramenta, incluindo atalhos, personalização de fluxos de trabalho, e como maximizar a utilidade dos relatórios gerados.
    \item **Resolução de Problemas Comuns**: Seção com orientações para resolver problemas básicos, como erros de upload de arquivos, falhas na conexão entre nódulos, ou dificuldades na interpretação dos resultados.
\end{itemize}

Este manual foi projetado para ser acessível a usuários com diferentes níveis de habilidade técnica, facilitando o uso do RADARE de forma intuitiva e prática no dia a dia.
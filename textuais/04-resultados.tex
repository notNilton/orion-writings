\mychapter{Resultados}
\label{Cap:Resultados}

Neste capítulo, são apresentados os resultados obtidos a partir da implementação e testes do \textit{software} RADARE, com foco na usabilidade e eficiência da reconciliação de dados industriais. As principais dimensões avaliadas incluem a precisão dos dados reconciliados, o desempenho computacional em diferentes cenários de carga de trabalho e a usabilidade da ferramenta, com destaque para o front-end e o back-end, ambos avaliados em testes realizados com usuários.

\section{Front-end}

O front-end do RADARE foi desenhado para oferecer uma interface intuitiva, permitindo aos usuários interagir facilmente com a ferramenta, visualizar fluxos de dados e executar a reconciliação. As duas principais áreas do front-end são o menu e o canvas.

\subsection{Sessão de Menu}

O menu do RADARE permite ao usuário adicionar componentes e controlar a visualização dos dados. Os itens disponíveis no menu incluem:

\begin{itemize}
    \item \textbf{Adicionar Input}: Permite adicionar um novo ponto de entrada de dados.
    \item \textbf{Adicionar Output}: Adiciona um ponto de saída de dados.
    \item \textbf{Adicionar Nódulo 1-1}: Insere um nó de processamento básico.
    \item \textbf{Adicionar Nódulo 1-2}: Insere um nó que se conecta a dois outros pontos.
    \item \textbf{Adicionar Nódulo 2-1}: Insere um nó que recebe duas entradas e gera uma saída.
    \item \textbf{Reconciliar Dados}: Executa a reconciliação dos dados conectados no canvas.
    \item \textbf{Esconder Gráfico das Reconciliações}: Esconde a visualização gráfica dos resultados reconciliados.
    \item \textbf{Esconder Sidebar de Informações}: Esconde a barra lateral com informações detalhadas.
    \item \textbf{Upload de Arquivos CSV}: Permite o envio de arquivos CSV para importar dados de medições.
\end{itemize}

\subsection{Sessão de Canvas}

O canvas é o espaço onde o usuário pode conectar e ajustar os nódulos para configurar o fluxo de dados industrial:

\begin{itemize}
    \item \textbf{Conectar Nódulos}: Permite que o usuário conecte visualmente diferentes nódulos de entrada, saída e processamento.
    \item \textbf{Ajustar Conexões entre Nódulos}: O usuário pode ajustar e mover as conexões entre nódulos para refinar o fluxo de dados.
\end{itemize}

\section{Back-end}

O back-end do RADARE foi desenvolvido em \textit{Node.js}, com foco em desempenho, escalabilidade e segurança. Ele é responsável por gerenciar as requisições da interface, processar os dados submetidos e realizar os cálculos de reconciliação utilizando o método dos multiplicadores de Lagrange, além de retornar os resultados para o front-end de forma otimizada.

\subsection{Routes}

As \textit{routes} do sistema foram estruturadas para permitir a comunicação entre a interface gráfica e a API. As principais rotas incluem:

\begin{itemize} \item \textbf{POST /reconcile}: Recebe os dados dos sensores e processa a reconciliação utilizando o método dos multiplicadores de Lagrange. \item \textbf{GET /results}: Retorna os resultados das reconciliações realizadas para serem exibidos na interface. \item \textbf{POST /upload}: Gerencia o envio de arquivos CSV para o banco de dados, permitindo que os dados sejam integrados ao sistema. \end{itemize}

\subsection{Services}

Os \textit{services} no back-end são responsáveis por implementar a lógica de negócio, processar os dados e invocar os cálculos de reconciliação. Eles incluem:

\begin{itemize} \item \textbf{Reconciliation Service}: Executa o algoritmo de reconciliação de dados, garantindo que as restrições de balanço de massa e energia sejam respeitadas. \item \textbf{Data Validation Service}: Valida os dados recebidos para garantir que estejam no formato adequado antes de serem processados. \item \textbf{File Processing Service}: Processa os arquivos CSV enviados, convertendo-os em estruturas adequadas para o banco de dados e para os cálculos. \end{itemize}

\subsection{Models}

Os \textit{models} foram implementados utilizando \textit{Sequelize}, uma ORM (Object-Relational Mapping) para \textit{Node.js}, que facilita a comunicação com o banco de dados PostgreSQL. Os principais modelos incluem:

\begin{itemize} \item \textbf{Process Model}: Representa os processos industriais, contendo as informações de variáveis medidas e parâmetros. \item \textbf{Measurement Model}: Armazena as medições recebidas dos sensores. \item \textbf{Result Model}: Registra os resultados das reconciliações, vinculando-os aos processos e medições correspondentes. \end{itemize}

Essas subseções detalham a implementação do back-end, destacando como o sistema processa os dados de forma eficiente e segura para realizar a reconciliação e retornar os resultados para o front-end.

\section{Banco de Dados}

O banco de dados utilizado no sistema RADARE foi implementado em PostgreSQL e armazena todas as informações relevantes para a execução do processo de reconciliação de dados industriais. A modelagem foi feita de forma a garantir a integridade e eficiência na consulta e manipulação dos dados. A seguir, são descritas as principais colunas do banco de dados que armazenam os dados de processos e medições.

\begin{itemize}
    \item \textbf{id}: Coluna de identificação única para cada registro no banco de dados. Utilizada como chave primária para vincular informações em diferentes tabelas.
    \item \textbf{user}: Armazena informações sobre o usuário que realizou o upload dos dados ou a operação de reconciliação. Permite rastrear a origem das interações no sistema.
    \item \textbf{time}: Representa o timestamp associado ao dado, indicando o momento exato em que a medição foi realizada ou quando a reconciliação foi executada.
    \item \textbf{tagname}: Identifica a variável medida (sensor ou ponto de coleta) no processo industrial. Cada variável é referenciada por um nome único que facilita a identificação no sistema.
    \item \textbf{tagreconciled}: Armazena o valor reconciliado de cada variável, ou seja, o resultado do processo de reconciliação de dados, após correção de inconsistências.
    \item \textbf{tagcorrection}: Coluna que contém o valor da correção aplicada a cada variável medida, calculada durante o processo de reconciliação para ajustar os dados medidos às restrições do sistema.
    \item \textbf{tagmatrix}: Refere-se à matriz de correlação utilizada no processo de reconciliação para garantir que os valores reconciliados atendam às leis de conservação de massa e energia, ou outras restrições impostas.
\end{itemize}

Esta estrutura de banco de dados foi desenhada para garantir a rastreabilidade e precisão dos dados processados, além de permitir que as reconciliações sejam realizadas de forma eficiente e escalável.
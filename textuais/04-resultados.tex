\mychapter{Resultados}
\label{Cap:Resultados}

Neste capítulo, são apresentados os resultados obtidos a partir da implementação e testes do \textit{software} RADARE. As principais dimensões avaliadas incluem a precisão dos dados reconciliados, o desempenho computacional em diferentes cenários de carga de trabalho, e a usabilidade da ferramenta, com foco em aspectos específicos do front-end, como a interação com o menu e o canvas, e do back-end, como as rotas, serviços e a integração com o banco de dados. Os testes foram realizados com usuários de diferentes perfis, permitindo avaliar tanto a eficiência técnica quanto a experiência de uso do sistema.

\section{Front-end}

O front-end do RADARE foi desenhado para oferecer uma interface intuitiva, permitindo aos usuários interagir facilmente com a ferramenta, visualizar fluxos tanto de dados como de modelos e executar a reconciliação. As duas principais áreas do front-end são o menu e o canvas.

\subsection{Sessão de Menu}

A sessão de menu do RADARE oferece ao usuário uma interface intuitiva para adicionar componentes, controlar o fluxo de dados e gerenciar a visualização do sistema. Cada item disponível no menu é detalhado a seguir, acompanhado de exemplos de código e imagens que ilustram sua funcionalidade e implementação no contexto da ferramenta.

A biblioteca \textit{ReactFlow} \cite{reactflow} desempenha um papel central na manipulação dos nódulos dentro do \textit{canvas}, sendo amplamente adaptada para atender às necessidades específicas do projeto. Modificações foram implementadas para garantir uma usabilidade fluida e eficiente, permitindo que os usuários adicionem e conectem nódulos de forma dinâmica.

Cada nódulo é constituído a partir de uma estrutura definida, onde as conexões, conhecidas como \textit{handles}, são personalizadas com características específicas, como estilos visuais e lógica de interação. Essa flexibilidade garante que a visualização e a manipulação dos fluxos de dados no \textit{canvas} sejam intuitivas, facilitando o gerenciamento de processos complexos na interface do RADARE.

\begin{lstlisting}[language=Java, caption=Nódulo customizado de exemplo]
    <>
      <Handle
        type="target"
        position={Position.Left}
        style={{ background: "black" }}
        isConnectable={isConnectable}
      ></Handle>

      <Handle
        type="source"
        position={Position.Right}
        id="b"
        style={{ top: 20, background: "#784be8" }}
        isConnectable={isConnectable}
        maxConnections={maxConnections}
      ></Handle>

      <Handle
        type="source"
        position={Position.Right}
        id="a"
        style={{ top: 75, background: "#784be8" }}
        isConnectable={isConnectable}
      ></Handle>

      <div>{`${label}`}</div>
    </>
\end{lstlisting}

\subsubsection{Adicionar Input}

O botão \textbf{"Adicionar Input"} permite ao usuário inserir um novo nódulo de entrada no \textit{canvas}. Esse \textit{input} representa um sensor ou uma fonte de dados do sistema industrial. Ao clicar no botão, um novo nó de \textit{input} é adicionado ao \textit{canvas}, permitindo que o usuário conecte esse ponto a outros nódulos do fluxo de dados. A lógica computacional associada a esse comportamento é fornecida pela biblioteca ReactFlow \cite{reactflow}, eliminando a necessidade de configurar comportamentos iniciais customizados.

\begin{figure}[htbp]
    \centering
    \includegraphics[width=0.6\textwidth]{figuras/add_input.png}
    \caption{Nó de input adicionado ao canvas.}
    \label{Fig:AddInput}
\end{figure}

\subsubsection{Adicionar Output}

O botão \textbf{"Adicionar Output"} permite ao usuário inserir um novo nódulo de saída no \textit{canvas}. Esse \textit{output} representa um destino ou ponto final para os dados no sistema industrial, como a exportação de resultados processados ou a visualização de dados reconciliados. Ao clicar no botão, um novo nó de \textit{output} é adicionado ao \textit{canvas}, permitindo que o usuário conecte este nó a outros nódulos de processamento ou entrada no fluxo de dados. Assim como o \textit{input}, a lógica necessária para a adição desse nódulo é fornecida pela biblioteca ReactFlow \cite{reactflow}, dispensando a necessidade de customização de comportamentos iniciais.

\begin{figure}[htbp]
    \centering
    \includegraphics[width=0.6\textwidth]{figuras/add_output.png}
    \caption{Nó de output adicionado ao canvas.}
    \label{Fig:AddOutput}
\end{figure}

\subsubsection{Adicionar Nódulo 1-1}

O botão \textbf{"Adicionar Nódulo 1-1"} permite ao usuário inserir um novo nódulo de transição no \textit{canvas}. Esse nódulo atua como um ponto intermediário no fluxo de dados, podendo representar sensores, transformações ou outros elementos de processo. Ao clicar no botão, o nódulo é adicionado ao \textit{canvas}, permitindo ao usuário conectá-lo com outros nódulos.

A lógica para adicionar esse nódulo foi construída de forma customizada, definindo o tipo de conexão, a quantidade de pontos de conexão (\textit{handles}), o estilo visual e a posição no \textit{canvas}, adaptando-se ao comportamento esperado no fluxo de dados.

Abaixo está o trecho principal do código responsável pela criação desse nódulo:

\begin{lstlisting}[language=Java, caption=Nódulo customizado 1-1]
const customNodeOneOne = ({ data }) => {
  const { label, isConnectable } = data;

  return (
    <>
      <Handle
        type="source"
        id="a"
        position={Position.Left}
        style={{ background: "black" }}
        isConnectable={isConnectable}
      ></Handle>
      <Handle
        type="source"
        position={Position.Right}
        id="b"
        style={{ background: "black" }}
        isConnectable={isConnectable}
      ></Handle>
      <div>{`${label}`}</div>
    </>
  );
};

export default memo(customNodeOneOne);
\end{lstlisting}

\begin{figure}[htbp]
    \centering
    \includegraphics[width=0.6\textwidth]{figuras/add_node_1_1.png}
    \caption{Nó 1-1 adicionado ao canvas.}
    \label{Fig:AddNode1to1}
\end{figure}

\subsubsection{Adicionar Nódulo 1-2}

Este item adiciona um nó que se conecta a dois outros pontos. Isso permite dividir os dados em dois fluxos distintos para processamento paralelo.

\begin{lstlisting}[language=Java, caption=Nódulo customizado de exemplo]
    <>
      <Handle
        type="target"
        position={Position.Left}
        style={{ background: "black" }}
        isConnectable={isConnectable}
      ></Handle>

      <Handle
        type="source"
        position={Position.Right}
        id="b"
        style={{ top: 20, background: "#784be8" }}
        isConnectable={isConnectable}
        maxConnections={maxConnections}
      ></Handle>

      <Handle
        type="source"
        position={Position.Right}
        id="a"
        style={{ top: 75, background: "#784be8" }}
        isConnectable={isConnectable}
      ></Handle>

      <div>{`${label}`}</div>
    </>
\end{lstlisting}

\begin{figure}[htbp]
    \centering
    \includegraphics[width=0.6\textwidth]{figuras/add_node_1_2.png}
    \caption{Nó 1-2 adicionado ao canvas.}
    \label{Fig:AddNode1to2}
\end{figure}

\subsubsection{Adicionar Nódulo 2-1}

O nódulo 2-1 recebe duas entradas e gera uma única saída, sendo útil em situações onde dados de diferentes fontes precisam ser combinados.

\begin{verbatim}
// Exemplo de código para adicionar Nódulo 2-1 no ReactFlow
const addNode2to1 = () => {
    const newNode = {
        id: `node2_1_${Date.now()}`,
        data: { label: 'Nódulo 2-1' },
        position: { x: 400, y: 300 },
        type: 'custom',
    };
    setElements((els) => [...els, newNode]);
};
\end{verbatim}

\begin{figure}[htbp]
    \centering
    \includegraphics[width=0.6\textwidth]{figuras/add_node_2_1.png}
    \caption{Nó 2-1 adicionado ao canvas.}
    \label{Fig:AddNode2to1}
\end{figure}

\subsubsection{Reconciliar Dados}

O botão "Reconciliar Dados" executa o processo de reconciliação de dados conectados no canvas. Ele faz a análise dos nós conectados e aplica o método dos multiplicadores de Lagrange para ajustar as discrepâncias nos dados.

\begin{verbatim}
// Função para iniciar o processo de reconciliação
const reconcileData = () => {
    axios.post('/api/reconcile', { data: elements })
        .then((response) => {
            console.log('Reconciliação concluída', response.data);
        })
        .catch((error) => {
            console.error('Erro na reconciliação', error);
        });
};
\end{verbatim}

\begin{figure}[htbp]
    \centering
    \includegraphics[width=0.6\textwidth]{figuras/reconcile_data.png}
    \caption{Botão de reconciliação de dados.}
    \label{Fig:ReconcileData}
\end{figure}

\subsubsection{Esconder Gráfico das Reconciliações}

Este item permite esconder o gráfico que exibe os resultados das reconciliações de dados, liberando espaço na interface para outros elementos.

\begin{figure}[htbp]
    \centering
    \includegraphics[width=0.6\textwidth]{figuras/hide_chart.png}
    \caption{Opção para esconder o gráfico de reconciliações.}
    \label{Fig:HideChart}
\end{figure}

\subsubsection{Esconder Sidebar de Informações}

O menu também oferece a opção de esconder a barra lateral que exibe informações detalhadas sobre os nós e fluxos.

\begin{figure}[htbp]
    \centering
    \includegraphics[width=0.6\textwidth]{figuras/hide_sidebar.png}
    \caption{Opção para esconder a barra lateral de informações.}
    \label{Fig:HideSidebar}
\end{figure}

\subsubsection{Upload de Arquivos CSV}

O botão "Upload de Arquivos CSV" permite importar dados de medições diretamente para o sistema. O usuário pode carregar arquivos CSV contendo as informações que serão processadas e reconciliadas pelo RADARE.

\begin{verbatim}
// Exemplo de código para upload de arquivos CSV
const handleCSVUpload = (event) => {
    const file = event.target.files[0];
    const formData = new FormData();
    formData.append('csvFile', file);

    axios.post('/api/upload', formData)
        .then((response) => {
            console.log('Upload concluído', response.data);
        })
        .catch((error) => {
            console.error('Erro no upload', error);
        });
};
\end{verbatim}

\begin{figure}[htbp]
    \centering
    \includegraphics[width=0.6\textwidth]{figuras/upload_csv.png}
    \caption{Botão de upload de arquivos CSV.}
    \label{Fig:UploadCSV}
\end{figure}


\subsection{Sessão de Canvas}

O canvas é a área principal da interface do RADARE, onde o usuário pode visualizar, conectar e manipular os nódulos para configurar o fluxo de dados industrial. Nesta seção, descrevemos as funcionalidades disponíveis no canvas, incluindo exemplos de código e imagens que ilustram o comportamento da interface.

\subsubsection{Conectar Nódulos}

A funcionalidade de "Conectar Nódulos" permite ao usuário estabelecer conexões visuais entre os nódulos de entrada, processamento e saída. Estas conexões representam o fluxo de dados entre diferentes pontos de um processo industrial.

\begin{verbatim}
// Exemplo de código para conectar nódulos no ReactFlow
const onConnect = (params) => {
    setElements((els) => addEdge(params, els));
};
<ReactFlow
    elements={elements}
    onConnect={onConnect}
/>
\end{verbatim}

\begin{figure}[htbp]
    \centering
    \includegraphics[width=0.6\textwidth]{figuras/connect_nodes.png}
    \caption{Conexões entre nódulos no canvas.}
    \label{Fig:ConnectNodes}
\end{figure}

Ao clicar e arrastar entre dois nódulos, o usuário cria uma linha de conexão, simbolizando o fluxo de dados que vai de um nódulo de entrada para um de processamento ou saída.

\subsubsection{Ajustar Conexões entre Nódulos}

O canvas também permite que o usuário ajuste e mova as conexões entre os nódulos. Isso proporciona flexibilidade na organização dos fluxos de dados, permitindo que o layout seja personalizado conforme as necessidades do usuário.

\begin{verbatim}
// Exemplo de código para ajustar conexões entre nódulos no ReactFlow
const onNodeDragStop = (event, node) => {
    setElements((els) => 
        els.map((el) => 
            el.id === node.id 
            ? { ...el, position: node.position } 
            : el
        )
    );
};
<ReactFlow
    elements={elements}
    onNodeDragStop={onNodeDragStop}
/>
\end{verbatim}

\begin{figure}[htbp]
    \centering
    \includegraphics[width=0.6\textwidth]{figuras/adjust_connections.png}
    \caption{Ajuste das conexões entre os nódulos no canvas.}
    \label{Fig:AdjustConnections}
\end{figure}

O usuário pode clicar e arrastar tanto os nódulos quanto as conexões para reorganizar a estrutura do fluxo. Esta funcionalidade é útil para criar uma visão mais clara e organizada dos processos, principalmente em fluxos mais complexos.

\begin{figure}[htbp]
    \centering
    \includegraphics[width=0.6\textwidth]{figuras/move_nodes.png}
    \caption{Movimentação dos nódulos no canvas.}
    \label{Fig:MoveNodes}
\end{figure}

Essas opções permitem que o usuário tenha controle total sobre a disposição e as interações entre os componentes no canvas, criando uma experiência de visualização clara e eficiente para o fluxo de dados industrial.

\section{Back-end}

O back-end do RADARE foi desenvolvido em \textit{Node.js}, com foco em desempenho, escalabilidade e segurança. Ele é responsável por gerenciar as requisições da interface, processar os dados submetidos e realizar os cálculos de reconciliação utilizando o método dos multiplicadores de Lagrange, além de retornar os resultados para o front-end de forma otimizada.

\subsection{Routes}

As \textit{routes} do sistema foram estruturadas para facilitar a comunicação entre a interface gráfica e a API, garantindo que as requisições de dados e os processos de reconciliação sejam executados de maneira eficiente. Abaixo estão descritas as principais rotas implementadas no sistema, com exemplos de código e uma explicação detalhada de cada funcionalidade.

\subsubsection{POST /reconcile}

Esta rota recebe os dados dos sensores e processa a reconciliação utilizando o método dos multiplicadores de Lagrange. O fluxo de dados é enviado pelo \textit{front-end} e, após o processamento, os resultados são armazenados no banco de dados para serem recuperados posteriormente.

\begin{verbatim}
// Exemplo de código para a rota de reconciliação
app.post('/reconcile', (req, res) => {
    const sensorData = req.body;
    const reconciledData = processReconciliation(sensorData);
    res.json(reconciledData);
});
\end{verbatim}

\begin{figure}[htbp]
    \centering
    \includegraphics[width=0.6\textwidth]{figuras/reconcile_route.png}
    \caption{Fluxo de dados na rota /reconcile.}
    \label{Fig:ReconcileRoute}
\end{figure}

\subsubsection{GET /results}

A rota \texttt{GET /results} permite ao usuário recuperar os resultados das reconciliações realizadas anteriormente. Esses resultados são exibidos na interface gráfica, fornecendo ao usuário uma visão dos dados reconciliados e permitindo a análise dos processos.

\begin{verbatim}
// Exemplo de código para a rota de resultados
app.get('/results', (req, res) => {
    const results = getReconciliationResults();
    res.json(results);
});
\end{verbatim}

\begin{figure}[htbp]
    \centering
    \includegraphics[width=0.6\textwidth]{figuras/results_route.png}
    \caption{Exibição dos resultados utilizando a rota /results.}
    \label{Fig:ResultsRoute}
\end{figure}

\subsubsection{POST /upload}

A rota \texttt{POST /upload} gerencia o envio de arquivos CSV para o banco de dados. Esta funcionalidade é fundamental para permitir que o sistema integre dados externos de sensores, facilitando a reconciliação e o processamento.

\begin{verbatim}
// Exemplo de código para a rota de upload de arquivos CSV
app.post('/upload', upload.single('file'), (req, res) => {
    const csvData = req.file;
    processCSV(csvData);
    res.status(200).send('Upload concluído com sucesso');
});
\end{verbatim}

\begin{figure}[htbp]
    \centering
    \includegraphics[width=0.6\textwidth]{figuras/upload_route.png}
    \caption{Fluxo de upload de arquivos CSV na rota /upload.}
    \label{Fig:UploadRoute}
\end{figure}

Estas rotas garantem a integração perfeita entre o \textit{front-end} e o \textit{back-end}, oferecendo ao usuário uma experiência contínua ao processar, visualizar e carregar dados para o sistema RADARE.

\subsection{Services}

Os \textit{services} no \textit{back-end} são responsáveis por implementar a lógica de negócio, processar os dados e invocar os cálculos de reconciliação. Eles abstraem a complexidade do sistema, garantindo que os dados sejam processados corretamente antes de serem enviados ao banco de dados ou utilizados na interface. Abaixo estão descritos os principais \textit{services} do sistema RADARE, juntamente com exemplos de código e diagramas.

\subsubsection{Reconciliation Service}

O \textit{Reconciliation Service} é o núcleo do sistema, responsável pela execução do algoritmo de reconciliação de dados. Ele aplica o método dos multiplicadores de Lagrange para garantir que as restrições de balanço de massa e energia sejam respeitadas durante a reconciliação dos dados.

\begin{verbatim}
// Exemplo de código para o Reconciliation Service
class ReconciliationService {
    static reconcileData(sensorData) {
        const reconciledData = runLagrangeMultipliers(sensorData);
        return reconciledData;
    }
}
\end{verbatim}

Este serviço é invocado sempre que o usuário solicita a reconciliação dos dados na interface gráfica.

\begin{figure}[htbp]
    \centering
    \includegraphics[width=0.6\textwidth]{figuras/reconciliation_service.png}
    \caption{Processo de execução do Reconciliation Service.}
    \label{Fig:ReconciliationService}
\end{figure}

\subsubsection{Data Validation Service}

O \textit{Data Validation Service} garante que os dados recebidos estejam no formato correto antes de serem processados. Ele valida se todos os campos obrigatórios estão presentes, se os tipos de dados são consistentes e se não há valores ausentes ou inválidos.

\begin{verbatim}
// Exemplo de código para o Data Validation Service
class DataValidationService {
    static validate(data) {
        if (!data || !Array.isArray(data)) {
            throw new Error('Dados inválidos');
        }
        // Validação adicional
        return true;
    }
}
\end{verbatim}

Este serviço é executado antes da reconciliação de dados, garantindo que apenas dados válidos sejam processados.

\begin{figure}[htbp]
    \centering
    \includegraphics[width=0.6\textwidth]{figuras/validation_service.png}
    \caption{Fluxo de validação dos dados no Data Validation Service.}
    \label{Fig:ValidationService}
\end{figure}

\subsubsection{File Processing Service}

O \textit{File Processing Service} é responsável por lidar com arquivos CSV enviados pelos usuários. Ele converte os dados do arquivo em estruturas adequadas para o banco de dados e para os cálculos de reconciliação. Este serviço é crucial para integrar dados externos ao sistema RADARE.

\begin{verbatim}
// Exemplo de código para o File Processing Service
class FileProcessingService {
    static processCSV(file) {
        const csvData = parseCSV(file);
        return csvData.map(row => formatRow(row));
    }
}
\end{verbatim}

Este serviço é acionado sempre que um arquivo CSV é enviado para o sistema.

\begin{figure}[htbp]
    \centering
    \includegraphics[width=0.6\textwidth]{figuras/file_processing_service.png}
    \caption{Processamento de arquivos CSV no File Processing Service.}
    \label{Fig:FileProcessingService}
\end{figure}

Os \textit{services} do \textit{back-end} do RADARE garantem que o sistema funcione de maneira eficiente e segura, abstraindo a lógica complexa dos algoritmos e permitindo que o sistema lide com grandes volumes de dados de forma robusta.


\subsection{Models}

Os \textit{models} foram implementados utilizando \textit{Sequelize}, uma ORM (Object-Relational Mapping) para \textit{Node.js}, que facilita a comunicação com o banco de dados PostgreSQL. Os principais modelos incluem:

\begin{itemize} \item \textbf{Process Model}: Representa os processos industriais, contendo as informações de variáveis medidas e parâmetros. \item \textbf{Measurement Model}: Armazena as medições recebidas dos sensores. \item \textbf{Result Model}: Registra os resultados das reconciliações, vinculando-os aos processos e medições correspondentes. \end{itemize}

Essas subseções detalham a implementação do back-end, destacando como o sistema processa os dados de forma eficiente e segura para realizar a reconciliação e retornar os resultados para o front-end.

\section{Banco de Dados}

O banco de dados utilizado no sistema RADARE foi implementado em PostgreSQL e armazena todas as informações relevantes para a execução do processo de reconciliação de dados industriais, gerenciamento de usuários e rastreamento de atividades. A modelagem foi feita de forma a garantir a integridade e eficiência na consulta e manipulação dos dados. A seguir, são descritas as principais tabelas implementadas no sistema.

\subsection{Tabela de Dados de Processos}

A tabela que armazena os dados de processos industriais contém as medições e os resultados da reconciliação. As principais colunas são:

\begin{itemize}
    \item \textbf{id}: Coluna de identificação única para cada registro no banco de dados. Utilizada como chave primária para vincular informações em diferentes tabelas.
    \item \textbf{user}: Armazena informações sobre o usuário que realizou o upload dos dados ou a operação de reconciliação.
    \item \textbf{time}: Representa o timestamp associado ao dado, indicando o momento exato em que a medição foi realizada ou quando a reconciliação foi executada.
    \item \textbf{tagname}: Identifica a variável medida (sensor ou ponto de coleta) no processo industrial. Cada variável é referenciada por um nome único.
    \item \textbf{tagreconciled}: Armazena o valor reconciliado de cada variável, ou seja, o resultado do processo de reconciliação de dados.
    \item \textbf{tagcorrection}: Coluna que contém o valor da correção aplicada a cada variável medida.
    \item \textbf{tagmatrix}: Refere-se à matriz de correlação utilizada no processo de reconciliação.
\end{itemize}

\subsection{Tabela de Usuários}

Esta tabela gerencia a autenticação e as permissões dos usuários no sistema. Ela permite o rastreamento de interações e garante que apenas usuários autorizados possam realizar determinadas operações, além de armazenar informações para autenticação segura.

\begin{table}[htbp]
    \centering
    \caption{Estrutura da tabela de usuários.}
    \label{Tab:Users}
    \begin{tabular}{|c|c|}
        \hline
        \textbf{Coluna} & \textbf{Descrição} \\ \hline
        id (primary key) & UUID único para cada usuário \\ \hline
        username & Nome de usuário utilizado no login \\ \hline
        email & Email do usuário para notificações e recuperação de senha \\ \hline
        password\_hash & Hash da senha para garantir segurança \\ \hline
        role & Papel do usuário (admin, operador, analista) \\ \hline
        created\_at & Timestamp da criação da conta \\ \hline
        last\_login & Data e hora do último login realizado \\ \hline
    \end{tabular}
\end{table}

\subsection{Tabela de Logs de Atividade}

Esta tabela registra as ações realizadas pelos usuários no sistema, permitindo auditoria e rastreamento de atividades como uploads de arquivos ou execuções de reconciliação de dados.

\begin{table}[htbp]
    \centering
    \caption{Estrutura da tabela de logs de atividade.}
    \label{Tab:ActivityLogs}
    \begin{tabular}{|c|c|}
        \hline
        \textbf{Coluna} & \textbf{Descrição} \\ \hline
        id (primary key) & Identificação única para cada log de atividade \\ \hline
        user\_id & Referência ao id do usuário que realizou a ação \\ \hline
        action & Descrição da ação realizada (ex: upload de CSV, reconciliação de dados) \\ \hline
        timestamp & Data e hora em que a ação foi realizada \\ \hline
        details & Detalhes adicionais sobre a ação (ex: nome do arquivo CSV carregado) \\ \hline
    \end{tabular}
\end{table}

\subsection{Tabela de Configurações de Processos}

Esta tabela armazena as configurações dos processos industriais, incluindo parâmetros como limites de variáveis e tipos de medições. Essa estrutura permite personalizar o comportamento do sistema para diferentes cenários industriais.

\begin{table}[htbp]
    \centering
    \caption{Estrutura da tabela de configurações de processos.}
    \label{Tab:ProcessConfigurations}
    \begin{tabular}{|c|c|}
        \hline
        \textbf{Coluna} & \textbf{Descrição} \\ \hline
        id (primary key) & Identificação única para a configuração do processo \\ \hline
        process\_name & Nome do processo industrial \\ \hline
        sensor\_limits & Limites definidos para as variáveis dos sensores \\ \hline
        created\_by & ID do usuário que criou a configuração \\ \hline
        created\_at & Timestamp de criação da configuração \\ \hline
    \end{tabular}
\end{table}

Essa modelagem de banco de dados garante rastreabilidade, segurança e eficiência no gerenciamento de processos industriais e usuários dentro do sistema RADARE.

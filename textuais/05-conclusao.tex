\mychapter{Conclusão}
\label{Cap:Conclusao}

Este trabalho desenvolveu o \textit{webapp} RADARE, uma ferramenta dedicada à reconciliação de dados em processos industriais, baseada no método dos multiplicadores de Lagrange. O sistema provou ser eficaz ao corrigir erros de medição, respeitando leis de conservação e fornecendo resultados consistentes. A interface \textit{web}, construída com tecnologias modernas como \textit{React} e \textit{ReactFlow}, proporcionou uma experiência interativa e acessível para os usuários, permitindo a manipulação de nódulos e fluxos de dados diretamente no \textit{canvas}. No \textit{back-end}, a arquitetura modular em \textit{Python} garantiu a escalabilidade e adaptação do sistema para diferentes cenários industriais.

\section{Pontos Positivos}

O RADARE cumpriu os objetivos propostos, destacando-se em vários aspectos:

\begin{itemize}
    \item \textbf{Integração Eficiente}: A integração entre \textit{front-end} e \textit{back-end} foi realizada de forma eficiente, proporcionando uma experiência de uso fluida e sem interrupções para os usuários.
    
    \item \textbf{Interface Intuitiva}: A interface gráfica se mostrou intuitiva e funcional, facilitando a visualização de dados e conexões no \textit{canvas}, o que torna a interação do usuário mais natural e produtiva.
    
    \item \textbf{Estrutura Modular}: A arquitetura modular do sistema facilita futuras expansões, permitindo a adição de novos métodos de reconciliação e funcionalidades, garantindo que o RADARE possa evoluir conforme as necessidades da indústria.
    
    \item \textbf{Flexibilidade}: O sistema se adaptou bem a diferentes cenários de uso, sendo capaz de lidar com cargas de dados variáveis, um aspecto essencial para sua aplicação em ambientes industriais que demandam escalabilidade.
\end{itemize}


\section{Pontos Negativos}

Para aprimorar o RADARE e ampliar suas possibilidades de uso, algumas melhorias podem ser implementadas, organizadas da seguinte forma:

\begin{itemize}
    \item \textbf{Versão Offline}: Desenvolver uma versão do sistema que possa ser executada sem a necessidade de conexão com a internet, permitindo seu uso em ambientes industriais remotos ou com conectividade limitada.
    
    \item \textbf{Novos Métodos de Reconciliação}: Incorporar outros métodos de reconciliação de dados, como a filtragem de Kalman, o que proporcionaria maior flexibilidade ao sistema para lidar com processos dinâmicos e cenários com maior variabilidade.
    
    \item \textbf{Análise e Predição de Dados}: Adicionar técnicas de análise preditiva para antecipar falhas e otimizar a tomada de decisões, utilizando medições reconciliadas, o que promoveria uma otimização mais eficiente dos processos industriais.
    
    \item \textbf{Otimização do Front-End}: Implementar melhorias no \textit{front-end} para aumentar a performance, assegurando carregamento mais rápido e manipulação eficiente de grandes volumes de dados e conexões no \textit{canvas}.
    
    \item \textbf{Hospedagem em Ambiente de Produção}: Colocar o sistema em um ambiente de produção adequado para que as indústrias possam acessar o RADARE de forma contínua e confiável, possibilitando sua aplicação em tempo real.
\end{itemize}

Essas melhorias poderão consolidar o RADARE como uma ferramenta apropriada para a reconciliação de dados industriais, ampliando sua aplicação em diversos setores do mercado industrial brasileiro.

\section{Propostas de Melhorias}

Para aprimorar o RADARE e ampliar suas possibilidades de uso, algumas melhorias podem ser implementadas, organizadas da seguinte forma:

\begin{itemize}
    \item \textbf{Versão Offline}: Desenvolver uma versão do sistema que possa ser executada sem a necessidade de conexão com a internet, permitindo seu uso em ambientes industriais remotos ou com conectividade limitada.
    
    \item \textbf{Novos Métodos de Reconciliação}: Incorporar outros métodos de reconciliação de dados, como a filtragem de Kalman, o que proporcionaria maior flexibilidade ao sistema para lidar com processos dinâmicos e cenários com maior variabilidade.
    
    \item \textbf{Análise e Predição de Dados}: Adicionar técnicas de análise preditiva para antecipar falhas e otimizar a tomada de decisões, utilizando medições reconciliadas, o que promoveria uma otimização mais eficiente dos processos industriais.
    
    \item \textbf{Otimização do Front-End}: Implementar melhorias no \textit{front-end} para aumentar a performance, assegurando carregamento mais rápido e manipulação eficiente de grandes volumes de dados e conexões no \textit{canvas}.
    
    \item \textbf{Hospedagem em Ambiente de Produção}: Colocar o sistema em um ambiente de produção adequado para que as indústrias possam acessar o RADARE de forma contínua e confiável, possibilitando sua aplicação em tempo real.
\end{itemize}

Essas melhorias poderão consolidar o RADARE como uma ferramenta robusta e versátil para a reconciliação de dados industriais, ampliando sua aplicação em diversos setores do mercado industrial brasileiro.

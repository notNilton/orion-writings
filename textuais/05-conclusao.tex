\mychapter{Conclusão}
\label{Cap:Conclusao}

Este trabalho teve como objetivo desenvolver e implementar o \textit{webapp} RADARE, combinando técnicas modernas de desenvolvimento \textit{web} com o método dos multiplicadores de Lagrange. A solução visa aprimorar a qualidade e confiabilidade dos dados, oferecendo ao usuário uma experiência mais atualizada e eficiente, com impacto direto na otimização da tomada de decisões no ambiente industrial.

Ao longo do projeto, foram pesquisadas e analisadas as principais metodologias de desenvolvimento \textit{web}, além de tecnologias adequadas para a construção de um sistema acessível e eficiente. A escolha do ambiente \textit{web} provou ser acertada, proporcionando vantagens como interoperabilidade com outras plataformas, acesso remoto e uma interface amigável, que facilita o uso e torna o sistema acessível a usuários de diferentes níveis técnicos.

Entre as principais contribuições deste trabalho, destaca-se a aplicação eficaz do método dos multiplicadores de Lagrange em um sistema \textit{web}, algo ainda pouco explorado em ferramentas disponíveis no mercado nacional. Ao permitir a integração com diferentes sistemas de monitoramento, o RADARE tem o potencial de proporcionar um ganho significativo na eficiência dos processos industriais, tanto em termos de qualidade dos dados quanto na redução de falhas e custos operacionais.

As possibilidades para trabalhos futuros são amplas. A implementação de novos algoritmos, como a filtragem de Kalman, poderia melhorar a otimização de processos dinâmicos, tornando a ferramenta ainda mais eficaz em cenários de alta variabilidade. Além disso, a criação de módulos de análise preditiva seria um passo natural para o sistema, permitindo a antecipação de falhas e aumentando a confiabilidade das operações.

Em resumo, o RADARE atende a uma necessidade clara no mercado brasileiro ao oferecer uma solução inovadora e adaptável para a reconciliação de dados industriais. Apesar de algumas limitações observadas, o sistema apresenta um grande potencial de expansão e aplicação em novos setores, reforçando sua importância para o avanço tecnológico e industrial no Brasil.
\mychapter{Referencial teórico}
\label{Cap:ReferencialTeorico}

Este capítulo aborda a prática da reconciliação de dados, fundamental para garantir a consistência e precisão das informações coletadas em diferentes contextos científicos e industriais e utilizada como base no desenvolvimento do \textit{software}. É detalhada sua definição, evolução e aplicação na indústria, destacando sua contribuição para a otimização de processos e a análise proativa. 
    
Além disso, é discutido o método dos multiplicadores de Lagrange como uma abordagem matemática eficaz para resolver problemas complexos de reconciliação de dados. Por fim, a sinergia entre a indústria e a internet é contextualizada, evidenciando o papel crucial do \textit{software} na atualidade.

\section{Reconciliação de dados}
\subsection{Definição}

A reconciliação de dados é uma prática de tratamento de dados em diversos campos da ciência e da engenharia, visando garantir a consistência e a precisão dos dados coletados a partir de diferentes fontes \cite{datarecshakar}. Em sua essência, a reconciliação de dados consiste no processo de comparar e corrigir dados divergentes ou inconsistentes, a fim de garantir que todas as informações disponíveis estejam alinhadas e concordantes. Esse processo é essencial em contextos nos quais múltiplas fontes de dados são utilizadas, como em sistemas industriais, processos químicos, redes de sensores e modelagem ambiental \cite{datarecshakar}.
        
A reconciliação de dados envolve a aplicação de técnicas estatísticas e matemáticas para identificar e corrigir discrepâncias entre os dados observados e os valores esperados, com base em modelos e relações conhecidas \cite{datarecragnoli}. Isso pode incluir a detecção e correção de erros sistemáticos ou aleatórios, a estimativa de valores ausentes e a harmonização de diferentes tipos de dados. Ao garantir a consistência e a confiabilidade dos dados, a reconciliação de dados permite uma análise mais precisa e uma tomada de decisão mais fundamentada em uma variedade de contextos aplicados \cite{datarecragnoli}.
    
Em suma, a reconciliação de dados desempenha um papel crucial na garantia da qualidade e integridade dos dados em diferentes áreas de aplicação. Ao alinhar e corrigir dados discrepantes, essa prática permite uma análise mais confiável e uma interpretação mais precisa dos fenômenos estudados, contribuindo assim para avanços significativos em diversos campos da ciência e da engenharia \cite{datarecshakar}.
    
\subsection{Reconciliação de dados no âmbito industrial}
    
No começo da década de 1960 se entendeu a importância do controle de processos químicos industriais por computadores que aplicavam técnicas matemáticas \cite{computecontrol}, dessa forma surge a área da computação voltada à reconciliação de dados, na qual há a comparação, validação e correção de informações coletadas em diferentes pontos do processo, a fim de determinar a consistência dos dados, a qualidade dos mesmos ou até otimizar os processos \cite{datarecshakar}.
    
Ao longo das últimas décadas, os métodos de reconciliação de dados evoluíram significativamente, acompanhando os avanços tecnológicos e as demandas crescentes da indústria \cite{datarecsurvey}. Com o advento de sistemas de automação mais avançados, sensores inteligentes e a proliferação de dispositivos conectados, a quantidade de dados gerados nas operações industriais aumentou drasticamente \cite{datarecsurvey}. Nesse contexto, a reconciliação, análise e gestão de dados tornaram-se ferramentas indispensáveis para garantir a integridade e a confiabilidade das informações coletadas em tempo real \cite{aularecon}.
    
Na era contemporânea, a reconciliação de dados desempenha um papel crucial na otimização dos processos industriais, contribuindo para a eficiência operacional e a redução de custos. Sistemas avançados de reconciliação não apenas comparam e validam dados, mas também utilizam algoritmos sofisticados para identificar padrões, tendências e anomalias \cite{datarecragnoli}. Essa capacidade analítica permite que as indústrias ajam proativamente, antecipando-se a problemas potenciais, otimizando a produção e melhorando a qualidade dos produtos finais \cite{datarecshakar}.

\subsection{Reconciliação de dados pelo método de minimização de multivariáveis por multiplicadores de Lagrange}
    
No sistema em questão, a reconciliação de dados vai ser feita com a minimização de funções multivariáveis utilizando o método dos multiplicadores de Lagrange, desenvolvido pelo matemático Joseph Louis Lagrange (1739-1813), que desenvolveu um método de encontrar o mínimo ou máximo de uma função multivariável sujeita a uma ou várias condições de restrição \cite{lagrangebio}.
    
Nesse contexto, a aplicação do método dos multiplicadores de Lagrange destaca-se como uma abordagem matemática sofisticada e eficaz para resolver problemas complexos de reconciliação de dados \cite{optimizationlagrange}. A técnica proporciona uma estrutura robusta para lidar com situações em que é necessário otimizar uma função multivariável sujeita a restrições específicas. Ao utilizar os multiplicadores de Lagrange, o sistema ganha a capacidade de encontrar soluções que atendam simultaneamente às condições impostas, proporcionando uma reconciliação precisa e eficiente dos dados envolvidos \cite{aularecon}. Essa metodologia, fundamentada em princípios matemáticos sólidos, contribui para aprimorar a qualidade e a confiabilidade dos resultados obtidos, tornando-a uma ferramenta valiosa para a solução proposta no trabalhos.
        
\section{Desenvolvimento de um \textit{software online}}
\subsection{Desenvolvimento \textit{Front-End}}
    
A interface de usuário de um \textit{software}, também denominada de \textit{front-end}, desempenha um papel crucial no desenvolvimento de sistemas modernos \cite{eloquentjavascript}. É responsável por criar a experiência do usuário, tornando a interação com o \textit{software} intuitiva, eficiente e agradável. Compreende todos os elementos visíveis e interativos de uma aplicação, como botões, menus, formulários e \textit{layouts}, que são projetados e desenvolvidos para fornecer uma interface acessível e funcional aos usuários \cite{frontendperfomance}.
    
As principais tecnologias aplicadas a essa área dentro do desenvolvimento \textit{web} são linguagens de programação como HTML, CSS, JavaScript e TypeScript \cite{webdevlang}. Essas tecnologias permitem aos desenvolvedores criar interfaces de usuário dinâmicas e responsivas, capazes de se adaptar a diferentes dispositivos e tamanhos de tela. Além disso, o \textit{front-end} envolve a utilização de \textit{frameworks}, \textit{runtimes} e bibliotecas importantes e poderosas, como React, Angular e Vue.js, possibilitando um desenvolvimento rápido e eficiente de interfaces de usuário modernas \cite{frontendperfomance}. Desta forma, o \textit{front-end} é uma parte essencial do processo de desenvolvimento de \textit{software}, desempenhando um papel fundamental na criação de sistemas interativos e centrados na experiência do usuário \cite{reactjs}.

\subsection{Desenvolvimento \textit{Back-End}}
    
Muitas vezes referido como a parte não visível ou o "cérebro" por trás de uma aplicação, o \textit{back-end} é igualmente crucial para o funcionamento de sistemas modernos \cite{backend}. Enquanto o \textit{front-end} é responsável pela parte visual e interativa da aplicação, o \textit{back-end} lida com a lógica de negócios, o armazenamento de dados e a comunicação com o servidor. Isso inclui operações como autenticação de usuários, processamento de dados, gerenciamento de sessões e acesso a banco de dados \cite{backenddevroles}.
    
As tecnologias empregadas no \textit{back-end} são diversas, incluido linguagens de programação como Python, Ruby, Java, PHP e Node.js. Estas linguagens são utilizadas em conjunto com \textit{frameworks} e bibliotecas específicas para o desenvolvimento \textit{web}, como Django, Ruby on Rails, Spring, Laravel e Express.js \cite{eloquentjavascript}. O \textit{back-end} também envolve a criação de APIs (Interfaces de Programação de Aplicativos) para permitir a comunicação eficiente entre o \textit{front-end}, outros sistemas e o banco de dados. Desta forma, o \textit{back-end} é responsável por garantir que a aplicação \textit{web} seja funcional, segura e capaz de processar grandes volumes de dados, desempenhando um papel crucial no sucesso de sistemas \textit{web} modernos \cite{javascriptframework}.
    
\subsection{Desenvolvimento de Banco de Dados}
    
No contexto do desenvolvimento de sistemas \textit{web}, o banco de dados desempenha um papel fundamental na organização e armazenamento de dados utilizados pela aplicação \cite{databasedepth}. Ele oferece uma estrutura organizada e eficiente para armazenar informações de forma persistente, garantindo que os dados sejam recuperados de maneira rápida e precisa, quando necessário. O banco de dados é responsável por gerenciar grandes volumes de dados, lidar com transações complexas e garantir a integridade e a segurança das informações armazenadas \cite{databasesql}.
    
Existem diversos tipos de bancos de dados, cada um com suas características e funcionalidades específicas. Os bancos de dados relacionais, como MySQL, PostgreSQL e SQL Server, utilizam tabelas para organizar os dados em linhas e colunas, facilitando consultas complexas e garantindo a consistência dos dados por meio de relações definidas \cite{databasesqlnet}. Por outro lado, os bancos de dados NoSQL, como MongoDB e Cassandra, oferecem uma abordagem mais flexível e escalável para o armazenamento de dados não estruturados, como documentos, gráficos e dados em tempo real \cite{databasedepth}.
    
Além disso, o banco de dados é essencial para a integridade e a consistência dos dados em uma aplicação \textit{web}. Ele permite que os desenvolvedores armazenem e recuperem informações de forma eficiente, garantindo que os dados sejam sempre precisos e atualizados. Por meio de consultas e transações, o banco de dados fornece uma camada de abstração entre a aplicação e os dados subjacentes, facilitando o acesso e a manipulação dos dados de forma segura e eficaz \cite{databasesql}. Desta forma, o banco de dados é um componente essencial do desenvolvimento de sistemas \textit{web}, garantindo que as informações sejam armazenadas e gerenciadas de maneira confiável e eficiente \cite{databasesqlmaster}.

\subsection{Servidor}
    
Na arquitetura de sistemas \textit{web}, o servidor desempenha um papel central na comunicação entre o cliente e o \textit{back-end}. Ele é responsável por receber as requisições dos clientes, processá-las e fornecer as respostas adequadas \cite{serverdummy}. Além disso, o servidor também gerencia o armazenamento e o acesso aos dados, garantindo que as informações sejam entregues de forma rápida e segura \cite{serverhost}.
    
As tecnologias utilizadas na área de servidor variam dependendo das necessidades específicas do projeto, mas geralmente incluem sistemas operacionais como Linux ou Windows Server, juntamente com servidores \textit{web} como Apache, Nginx ou Microsoft IIS \cite{serverapache}. Além disso, \textit{frameworks} e tecnologias como Docker, Kubernetes e AWS são frequentemente empregados para facilitar a implantação e o gerenciamento de servidores em escala. Em suma, a área de servidor é fundamental para o funcionamento eficiente e confiável de sistemas \textit{web}, garantindo a entrega de conteúdo e serviços de forma rápida, segura e escalável \cite{serversql}.
    
\section{A sinergia entre a indústria e a internet} 

O panorama atual de avanço da internet e a convergência entre a internet e o setor industrial representam um marco significativo na era da Engenharia de Computação \cite{industry4status}. Este fenômeno transformador tem sido impulsionado pela fusão das tecnologias da informação com os processos industriais, dando origem a conceitos como Indústria 4.0. 
    
No âmbito desta ferramenta, é aplicada a intersecção dessas duas esferas, onde os conceitos de práticas industriais, reconciliação, análise e qualidade de dados se integram à internet, na qual é extraído deles o seu maior forte, como uma maior integralidade com outros sistemas por meio de APIs (interfaces de programação de aplicativos), melhor interatividade entre os elementos do sistema, promovendo uma comunicação mais dinâmica e eficaz, aumento da eficiência operacional e facilidade na gestão de processos \cite{industry4}. Essa sinergia possibilita a criação de ecossistemas industriais mais conectados nos quais os dados relevantes podem ser tratados de forma segura e eficiente. \cite{industrybuild}.
    
O horizonte atual, delineado pelos recentes avanços tecnológicos e inovações sustenta a perspectiva otimista que as indústrias estão destinadas a experimentar um crescimento substancial no país \cite{industrychina}. A convergência entre a internet e o setor industrial representa um catalisador significativo para a modernização e eficiência operacional. A integração de práticas avançadas de desenvolvimento de soluções voltadas à usabilidade e ao ambiente de desenvolvimento com controle computacional, como a reconciliação de dados e análise preditiva, impulsiona a qualidade e consistência dos processos produtivos \cite{industrydigital}.
    
Além disso, a aplicação da internet nas práticas industriais não só fortalece a competitividade das empresas mas também desempenha um papel crucial na expansão econômica do país \cite{industryiot}. A capacidade de adotar tecnologias inovadoras como a automação avançada, coloca as indústrias em uma posição estratégica para atender às crescentes demandas do mercado e elevar a produtividade \cite{industryinternet}. Nesse sentido é plausível afirmar que diante do atual cenário tecnológico e das tendências emergentes, é indubitável a necessidade e importância da ferramenta desenvolvida nesse trabalho \cite{industrychina}.
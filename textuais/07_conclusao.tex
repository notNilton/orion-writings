%%
%% Capítulo 5: Conclusões
%%

\mychapter{Conclusão}
\label{Cap:Conclusao}

O capítulo final depende do tipo de documento. Nas propostas de tema
deve ser apresentado de forma clara e sucinta o assunto a ser
desenvolvido e o cronograma de execução do trabalho. Nos trabalhos de conclusão de curso (TCC) deve ser ressaltadas as principais contribuições do
trabalho e as suas limitações.

As contribuições devem evitar as adjetivações e julgamentos de valor.
Quanto às limitações, não tenha medo de as apresentar: é muito mais
reconhecido um autor que apresenta os casos em que sua proposta não se
aplica do que outro que parece não ter consciência deles.

Escreva (com outras palavras) o que foi realizado e como foi realizado, o que o trabalho descrito no artigo conseguiu melhorar e qual a sua relevância, e quais são as vantagens e limitações da proposta que o TCC apresenta. Apresente também eventuais aplicações dos resultados obtidos (ou da metodologia, técnica, produto) e ideias de trabalhos futuro que possam melhorar o seu (não apenas apresente, mas indique como pode ser feito).

%\section{Encadernação}
%
%As propostas de tema e as versões iniciais das teses e dissertações
%são impressas em lado único da folha e em espaçamento um e meio. Para
%a encadernação, usa-se geralmente um método simples, tal como espiral
%na lateral das folhas e capa plástica transparente. O número de cópias
%é igual ao número de membros da banca e pelo menos mais uma (para o
%aluno).
%
%As versões finais das teses e dissertações são impressas em frente e
%verso e em espaçamento simples. O número mínimo de cópias é o seguinte:
%\begin{itemize}
%\item 3 cópias para o PPgEEC e a UFRN.
%\item 1 cópia para cada examinador externo que participou da banca.
%\item ao menos 1 cópia para o aluno (não obrigatória).
%\item 1 cópia para o orientador (por cortesia, não obrigatória)
%\end{itemize}
%
%Para a encadernação, deve-se adotar uma capa rígida de cor azul para
%as dissertações de mestrado e de cor preta para as teses de doutorado,
%ambas com letras douradas. Na capa deve constar o título do
%trabalho, o autor e o ano da defesa. Se possível, a mesma informação
%deve ser repetida na lombada do livro.
%
%Para as versões finais, também se exige uma cópia eletrônica (formato
%PDF) do texto, bem como outros dados. Maiores informações podem ser
%obtidas na página do PPgEEC: \url{http://www.ppgeec.ufrn.br/}

\section{Etapas de Homologação do Título}

A seguir apresentamos as etapas necessárias para a apresentação do TCC à banca de avaliação e o seu depósito na biblioteca.

\begin{enumerate}
	\item Após a conclusão do trabalho, o orientador deverá informar ao professor responsável pela disciplina de Trabalho de Conclusão de Curso, os membros que irão compor a banca examinadora e combinar o agendamento do horário e o local da apresentação;
	\item O professor responsável pela disciplina de Trabalho de Conclusão de Curso dará publicidade à apresentação e organizará em processo SEI, a Ata de Apresentação, a Avaliação da Banca e as Declarações de Participação de Banca de Defesa de TCC para os membros da banca examinadora;
	\item Após a apresentação e as correções que se fizerem necessárias ao TCC, o estudante deverá solicitar via processo SEI, destinado à coordenação do curso, o registro e a disponibilização do seu TCC no acervo da biblioteca. O estudante deverá anexar ao processo os seguintes documentos:
	\begin{itemize}
		\item Declaração do Orientador do TCC seguindo o modelo que consta no apêndice deste documento;
		\item Ata de Defesa de TCC que consta no processo SEI criado pelo professor responsável pela disciplina de TCC;
		\item Avaliação da Banca Examinadora que consta no processo SEI criado pelo professor responsável pela disciplina de TCC;
		\item O Trabalho de Conclusão de Curso em PDF;
		\item Termo de Autorização do Autor preenchido disponível no Anexo A.
	\end{itemize}
	\item Se o processo estiver corretamente instruído, a coordenação do curso encaminhará a solicitação à Biblioteca.
\end{enumerate}

\section{Para saber mais}

Procure no Google, ora! Brincadeiras a parte, existem inúmeros
tutoriais sobre \LaTeX\ na rede que podem dar maiores informações
sobre o aplicativo. Para conhecer os pacotes disponíveis, uma opção é
o livro \emph{The \LaTeX\ Companion} \cite{LATEX04}, popularmente
conhecido como o ``livro do cachorro''. Outras informações sobre
redação técnica e normas para confecção de teses e dissertações podem
ser encontradas em livros de Metodologia Científica.

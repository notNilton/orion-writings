\mychapter{Conclusão}
\label{Cap:Conclusao}

Este trabalho teve como objetivo o desenvolvimento e implementação do \textit{software} RADARE, uma ferramenta \textit{web} voltada para a reconciliação de dados de processos industriais, utilizando o método dos multiplicadores de Lagrange. A proposta foi criar uma solução capaz de melhorar a qualidade e a confiabilidade dos dados, contribuindo diretamente para a otimização da tomada de decisões no ambiente industrial.

Ao longo do projeto, foram pesquisadas e analisadas as principais metodologias de reconciliação de dados, bem como as tecnologias adequadas para a construção de um sistema acessível e eficiente. A escolha por um ambiente \textit{web} mostrou-se acertada, oferecendo vantagens como a interoperabilidade com outras plataformas, o acesso remoto e uma interface amigável, que, além de facilitar o uso, também torna o sistema acessível a usuários com diferentes níveis de habilidade técnica. Dessa forma, o RADARE se posiciona como uma ferramenta robusta, pronta para ser integrada ao ambiente industrial brasileiro, ajudando a mitigar problemas recorrentes de imprecisão em dados de processo.

Entre as principais contribuições deste trabalho, destaca-se a aplicação eficaz do método dos multiplicadores de Lagrange em um sistema \textit{web}, algo ainda pouco explorado em ferramentas disponíveis no mercado nacional. Ao permitir a integração com diferentes sistemas de monitoramento, o RADARE tem o potencial de proporcionar um ganho significativo na eficiência dos processos industriais, tanto em termos de qualidade dos dados quanto na redução de falhas e custos operacionais.

Contudo, algumas limitações foram identificadas. A implementação do método dos multiplicadores de Lagrange, embora eficiente para o escopo inicial, pode enfrentar desafios de desempenho quando aplicada a grandes volumes de dados ou a processos com dinâmicas complexas e altamente variáveis. Além disso, o foco inicial do RADARE nas indústrias químicas e petroquímicas pode restringir sua aplicabilidade a outros setores, exigindo adaptações futuras para atender diferentes tipos de processos industriais, como os encontrados nas indústrias de energia e mineração.

As possibilidades de trabalhos futuros são variadas. A inclusão de novos algoritmos, como a filtragem de Kalman para otimização em processos dinâmicos, poderia expandir o alcance da ferramenta, tornando-a mais eficiente em cenários de alta variabilidade. Outra perspectiva interessante seria a adaptação da plataforma para suportar grandes volumes de dados, utilizando arquiteturas distribuídas, o que permitiria seu uso em ambientes industriais de larga escala. A criação de módulos para análise preditiva também surge como uma evolução natural do sistema, visando antecipar falhas e aumentar a confiabilidade das operações.

Em síntese, o RADARE preenche uma lacuna existente no mercado brasileiro, oferecendo uma solução inovadora e adaptável para a reconciliação de dados industriais. Embora algumas limitações tenham sido observadas, o potencial de crescimento e de aplicação em novos setores reforça a relevância deste trabalho para o avanço tecnológico e industrial no Brasil.

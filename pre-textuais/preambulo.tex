%
% ********** Página de assinaturas
%

\begin{titlepage}

\begin{center}

\LARGE

\textbf{RADARE: Desenvolvimento de uma aplicação \textit{web} de reconciliação de dados em processos industriais com recursos para análise de dados utilizando métodos e tecnologias atuais}

\vfill

\Large

\textbf{Nilton Aguiar dos Santos}

\end{center}

\vfill

\noindent
Trabalho de Conclusão de Curso
aprovado em 25 de Outubro de 2024 pela banca examinadora composta
pelos seguintes membros:

\begin{center}

\vspace{1.5cm}\rule{0.95\linewidth}{1pt}
\parbox{0.9\linewidth}{%
Prof. Dr. Sicrano Matosinho de Melo (orientador) \dotfill\ FAENG/CUVG/UFMT}

\vspace{1.5cm}\rule{0.95\linewidth}{1pt}
\parbox{0.9\linewidth}{%
Prof. Dr. Beltrano Catandura do Amaral (co-orientador) \dotfill\ FAENG/CUVG/UFMT}

\vspace{1.5cm}\rule{0.95\linewidth}{1pt}
\parbox{0.9\linewidth}{%
Prof. Dr. Clint Stallone da Silva \dotfill\ IC/UFMT}

%\vspace{1.5cm}\rule{0.95\linewidth}{1pt}
%\parbox{0.9\linewidth}{%
%Profª Drª Florisbela do Amaral \dotfill\ DCA/UFRN}

\end{center}

\end{titlepage}

%
% ********** Dedicatória
%

% A dedicatória não é obrigatória. Se você tem alguém ou algo que teve
% uma importância fundamental ao longo do seu curso, pode dedicar a ele(a)
% este trabalho. Geralmente não se faz dedicatória a várias pessoas: para
% isso existe a seção de agradecimentos.
% Se não quiser dedicatória, basta excluir o texto entre
% \begin{titlepage} e \end{titlepage}

\begin{titlepage}

\vspace*{\fill}

\hfill
\begin{minipage}{0.5\linewidth}
\begin{flushright}
\large\it
``Não é o conhecimento, 
mas o ato de aprender, 
não a posse, 
mas o ato de lá chegar, 
que concede a maior satisfação."

(Carl Friedrich Gauss)
\end{flushright}
\end{minipage}

\vspace*{\fill}

\end{titlepage}

\chapter*{Agradecimentos}
\thispagestyle{empty}

\begin{trivlist}  \itemsep 2ex

\item Aos meus pais, José Roberto dos Santos e Teresa de Jesus de Souza Aguiar, por cada sacrifício, por cada gesto de amor, por cada exemplo de dedicação. 

\item Aos meus irmãos, que acreditaram em meu potencial e estiveram ao meu lado ao longo desta jornada acadêmica.

\item À minha namorada, por me motivar a cada instante deste processo.

\item Ao meu orientador, João Gustavo Coelho Pena, por sua inspiração e incansável dedicação.

\item Ao corpo docente da UFMT, pelo empenho constante em me proporcionar um ensino de excelência ao longo de todo o curso.

\end{trivlist}

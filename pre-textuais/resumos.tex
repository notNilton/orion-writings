%
% ********** Resumo
%

% Usa-se \chapter*, e não \chapter, porque este "capítulo" não deve
% ser numerado.
% Na maioria das vezes, ao invés dos comandos LaTeX \chapter e \chapter*,
% deve-se usar as nossas versões definidas no arquivo comandos.tex,
% \mychapter e \mychapterast. Isto porque os comandos LaTeX têm um erro
% que faz com que eles sempre coloquem o número da página no rodapé na
% primeira página do capítulo, mesmo que o estilo que estejamos usando
% para numeração seja outro.
\mychapterast{Resumo}

Neste trabalho de conclusão de curso, foi desenvolvido o \textbf{RADARE (Reconciliation and Data Analysis in a Responsive Environment)}, um \textit{software online} voltado para reconciliação e qualidade de dados, utilizando técnicas de minimização de funções multivariáveis pelo método dos multiplicadores de Lagrange. A solução prioriza uma abordagem baseada na \textit{web}, oferecendo ao usuário a capacidade de realizar a análise e reconciliação de dados de forma remota e eficiente, com foco nos conceitos computacionais modernos, para proporcionar uma experiência facilitada. O \textit{software} aplica cálculos matemáticos e estatísticos ao longo de todo o processo, especificamente voltados para problemas de reconciliação de dados.

Por meio do RADARE, é possível modelar todo um processo industrial, alimentá-lo com dados oriundos da planta em questão e, a partir disso, reconciliar e verificar a qualidade dos dados em tempo real. O \textit{software} se destaca como uma solução inovadora, pois não há concorrente direto que ofereça as mesmas funcionalidades em um ambiente mais acessível e ágil no estado de Mato Grosso. Ao longo deste trabalho, são detalhados o processo filosófico de desenvolvimento, os cálculos matemáticos, os conceitos estatísticos e computacionais, e a lógica do \textit{software} aplicada na solução. Exemplos do código funcional e considerações finais sobre o trabalho realizado são apresentados nos capítulos finais.


\vspace{1.5ex}

{\bf Palavras-chave}: \textit{Dashboard}, Desenvolvimento de \textit{Software}, Desenvolvimento \textit{Web}, Indústria 4.0, Multiplicadores de Lagrange, Qualidade de Dados, Reconciliação de Dados.


%
% ********** Abstract
%
\mychapterast{Abstract}

In this graduation thesis, the \textbf{RADARE (Reconciliation and Data Analysis in a Responsive Environment)} was developed, an \textit{online software} focused on data reconciliation and quality, using multivariable function minimization techniques through the Lagrange multipliers method. The solution prioritizes a \textit{web}-based approach, offering users the ability to perform data analysis and reconciliation remotely and efficiently, focusing on modern computational concepts to provide a facilitated experience. The \textit{software} applies mathematical and statistical calculations throughout the entire process, specifically aimed at data reconciliation problems.

Through RADARE, it is possible to model an entire industrial process, feed it with data from the plant in question, and then reconcile and verify the quality of the data in real-time. The software stands out as an innovative solution, as there is no direct competitor offering the same functionalities in a more accessible and agile environment in the state of Mato Grosso. Throughout this work, the philosophical development process, the mathematical calculations, the statistical and computational concepts, and the logic of the software applied as a solution are detailed. Examples of functional code and final considerations about the work performed are presented in the final chapters.

\vspace{1.5ex}

{\bf Keywords}: Dashboard, Data Quality, Data Reconciliation, Industry 4.0, Lagrange Multipliers, Software Development, Web Development.
